%%%%%%%%%%%%%%%%%%%%%%%%%%%%%%%%%%%%%%%%%%%%%%%%%%%%%%%%%%%%%%%%%%%%%%%%%%%%%%%
%%%%%%%%%%%%%%%%%%%       ADVANCED PROGRAMMING         %%%%%%%%%%%%%%%%%%%%%%%
%%%%%%%%%%%%%%%%%%%%%%%%%%%%%%%%%%%%%%%%%%%%%%%%%%%%%%%%%%%%%%%%%%%%%%%%%%%%%%%
% !TeX program = lualatex
\documentclass[a4paper,10pt,twoside,english]{book}
%\documentclass[a4paper,twoside,english]{book}


\usepackage[
paperwidth=16.99cm,
paperheight=24.40cm,
left=2.4cm,
right=2cm,
top=2.95cm,
bottom=2.2cm,
twoside
]{geometry}


\usepackage[width=16.99truecm,height=24.40truecm,center]{crop}
%\usepackage[frame,width=16.99truecm,height=24.40truecm,center]{crop}
%\usepackage[a4,frame,center]{crop}


\usepackage{epic}
\include{./doc/format/packages}
\include{./doc/format/listings}
\include{./doc/format/luamacros}


%\usepackage{parskip}
%\setlength{\parskip}{1cm plus 4mm minus 3mm}
%\setlength\parindent{18pt}



\graphicspath{ {images/} }
\newcommand{\folder}{./libraries}

\usepackage{amsfonts}

%MIGUEL
\usepackage{rotating}
\newcommand\myfunc[5]{%
    \begingroup
    \setlength\arraycolsep{0pt}
    #1\colon\begin{array}[t]{c >{{}}c<{{}} c}
        #2 & \to & #3 \\ #4 & \to & #5 
    \end{array}%
    \endgroup}


\usepackage{amsmath}
%\DeclareMathOperator{\Tr}{Tr}



%LINEAS NUEVAS
\newlength\myheight
\newlength\mydepth
\settototalheight\myheight{Xygp}
\settodepth\mydepth{Xygp}
\setlength\fboxsep{0pt}


% Replace the title of the table of contents 
\addto\captionsenglish{
    \renewcommand{\contentsname}%
    {Contents}% Here it goes the new title
}


%_______________________________________________________________________________
\usepackage{titling} % Allows custom title configuration

\bibliographystyle{abbrvnat}
\addto\captionsenglish{\renewcommand{\bibname}{References}}		%It changes name of bibliography to References using natbib

\definecolor{DarkGoldenrod}{rgb}{0.8,0.6,0.1}
\definecolor{DarkRed}{rgb}{0.5,0.1,0.1}
\definecolor{DarkRedPart}{rgb}{0.3,0.1,0.2}

% Defines the gold horizontal rule around the title
%\newcommand{\HorRule}{\color{DarkGoldenrod} \rule{\linewidth}{1pt}}

\renewcommand*{\bibfont}{\footnotesize} %Makes the font of the bibliography a little bit smaller to adjust it to one page.

% Horizontal rule before the title
\pretitle{\vspace{-150pt} \begin{flushleft} \HorRule \fontsize{34}{34}
		\usefont{OT1}{phv}{b}{n} \color{DarkRed} \selectfont}
	% Your article title	
	\title{{\fontsize{27}{33}\selectfont Advanced Programming for \\ Numerical Calculations:  \\  \vspace{0.8cm} } 
	 {\fontsize{27}{33}\selectfont Python \& Fortran\\ }
	 }
	
	% Whitespace under the title	
	\posttitle{\par\end{flushleft}\vskip 0.5em} 

% Author font configuration
\preauthor{	
\begin{flushleft} 
    \large \lineskip 0.5em \usefont{OT1}{phv}{b}{sl} \color{DarkRed}}	 	
	\author{ Juan Antonio Hern\'andez Ramos\\ 	
  	         Miguel \'Angel Rapado Tamarit\\
            \vspace{5cm} } % Your name. En función de los nombres poner más espacio en el vspace o menos, para dejar la línea abajo del todo
         
        % Configuration for the institution name
        \postauthor{\footnotesize \usefont{OT1}{phv}{m}{sl} \color{Black}
        Department of Applied Mathematics \\
        School of Aeronautical and Space Engineering \\
        Technical University of Madrid (UPM)		
	\par 
\end{flushleft}
\HorRule % Horizontal rule after the title
}

\date{  } % Add a date here if you would like one to appear underneath the title block




%%%%%%%%%%%%%%%%%%%%%%%%%%%%%%%%%%%%%%%%%%%%%%%%%%%%%%%%%%%%%%%%%%%%%%%%%%%%%%%%%%%%%%%%%%%%%%%%%%%%%%%%%%
%%%%%%%%%%%%%%%%%%%%%%%%%%%%%%%%%%%%%%%%%%%%%%%%%%%%%%%%%%%%%%%%%%%%%%%%%%%%%%%%%%%%%%%%%%%%%%%%%%%%%%%%%%
%\pagestyle{empty}
\begin{document}
 
  
%  \verb|pepito|
  
%  pepito 
  
% \end{document} 
   
    \maketitle
    \thispagestyle{empty}   %Le quita el número de página y las líneas arriba y abajo a la parte de atrás de la portada. 
    %\cleardoublepage       %Deja en blanco la página de detrás de la portada y no se numera
    
    %______________________________________________________________________________________________________
    %Para la página de detrás de la portada, la hacemos a mano que es más sencillo y da menos problemas.  
    
    \noindent  Cover: 
    
    \noindent If you want something, go for it. Cover design Belén Moreno Santamaría
   
 
    \vfill  
    \begin{minipage}[b]{0.9\textwidth}
     %   This manual is aimed at the Aerospace Engineering students of the Technical University of Madrid who may start programming different languages with Microsoft Visual Studio. It is also for anyone interested in start using this complete tool from scratch. Additional content can be downloaded from \url{https://github.com/jahrWork}. The manual has been written with the idea of increasing all the information presented and improving the contents. Ideas for something to be added, deleted or changed would be kindly appreciated. In addition, corrections and feedback about anything that could be improved or better explained are welcomed. All contributions can be made in:
        
        \vspace*{2\baselineskip}
        Miguel Ángel Rapado Tamarit \newline
        \textcolor{Blue}{marapadotamarit@gmail.com}
        \vspace*{3\baselineskip}
        
        %\footnotesize\raggedright
        
    \end{minipage}
    
    \noindent
    All rights reserved.
    No part of this publication may be reproduced or transmitted in any form or by any means, electronic or mechanical, including photocopy, recording, or any information storage and retrieval system, without permission in writing from the authors. 
    \vspace{2\baselineskip}
    
 %   \noindent \copyright   \ \ 2020 Miguel Ángel Rapado Tamarit \\ Belén Moreno Santamaría \\ Juan Antonio Hernández Ramos \\
       \noindent \copyright  \ \ 2022 Juan Antonio Hernández Ramos 
       
    
             
       \hspace{0.7cm}     Miguel Ángel Rapado Tamarit
    
    \vspace{0.5cm}            
    \noindent ISBN 978-1727581539 
    %\noindent Depósito Legal: M-3036-2001
    \clearpage
    
  
    %______________________________________________________________________________________________________
    \frontmatter
    
    \pagenumbering{roman}
    
    \parskip = \baselineskip %  plus 4mm minus 3mm}  
   % \input{./doc/Chapters/Preface.tex}
    
    \newpage
    \setlength{\parskip}{0pt}
    \tableofcontents
    ~\clearpage


    \pagenumbering{arabic}
    
    \parskip = \baselineskip
    
    %______________________________________________________________________________________________________
    \mainmatter
  
    \newcommand\home{./Fortran_project/sources/IEEE}
  
   \chapter*{Introduction}
%addcontentsline{toc}{chapter}{Introduction}
%\thispagestyle{empty}





\vspace{0.5cm}
 \renewcommand{\home}{./Fortran/sources} 
  \listings{\home/main.f90}{Welcome}{techniques}{main.f90}



    
  \part{Foundations}\label{PartI}
         \chapter{Python and Fortran foundations through examples} 
 
As a first approach to the use of Fortran to learn applied mathematics 
a chapter including basic operations will be presented. 
In it the reader can become familiar to the use of basic sentences in order to 
perform simple mathematical operations.

For each concept treated a Fortran code example guides the explanation and at the 
end of each chapter the corresponding Python codes are presented. Thus, the following
can be also used as a comparison between the foundations of both programming languages. 
 
The following menu schematizes the concepts treated in the following chapters, by choosing 
different options you can execute all the examples presented. 
 
    \newpage 
    \vspace{0.5cm}
    \renewcommand{\home}{./Fortran/sources/Foundations} 
  \listings{\home/Foundations.f90}{do while}{end select}{Foundations.f90}
   
    \newpage
    \vspace{0.5cm}
    \renewcommand{\home}{./Python/sources/Foundations} 
  \listings{\home/Foundations.py}{while}{Test_load_matrix()}{Foundations.py}  
   
 
 
         \chapter{Basic operations} 

\newpage 
\section{Roots of a second degree equation} 
In this section, a program to obtain the roots of a second order equation is presented:  
$$
a x^2 + b x + c = 0, \qquad \forall \ a, b, c \in \mathbb{R}.
$$
The fundamental theorem of algebra states that every  Nth order polynomial has N complex roots. 
If the coefficients are real, then the roots are complex conjugate.
Dividing the above equation by $ a $ an looking for a perfect square, 
the following equation is obtained: 
$$
\left( x + \frac{b}{2a} \right)^2 - \frac{b^2 }{ 4 a^2} + \frac{c}{a} = 0. 
$$
Solving the unknown $x$, the well known formula for the roots is obtained: 
\begin{equation}
 x_{1,2} = \frac{ - b \pm \sqrt{ b^2 - 4 a c }  }{ 2 a  }  
 \label{x12}
\end{equation} 
If the discriminant $ d = b^2 - 4 a c $ is less than zero, roots become complex. 
In the following code, complex solutions given by (\ref{x12}) are implemented. 
Note that the discriminant $ d $ was defined as a complex variable to avoid math problems 
when the discriminant is negative. Whereas the root of a real negative number is not defined,
the root of a negative number has a value for complex numbers. 
 
\vspace{0.5cm}
\listings{\home/Roots.f90}{subroutine Roots_2th}{end subroutine}{Roots.f90}



\newpage 
\subsection*{Python code}
The same function is presented below coded with Python. Essentially both codes are quite similar, 
however, now data types for variables are not explicitly declared because 
Python automatically declares them. In addition, indentation rules must be strictly followed. 
\vspace{0.5cm} 
\lstpython
\renewcommand{\home}{./Python/sources/Foundations/Roots} 
\listings{\home/Roots.py}{from}{output}{Roots.py}
\lstfor
  
  
 
         \input{./doc/Chapters/part_I/Imperative_declarative.tex}
         %___________________________________________________________________________________________________
\chapter{Operations with vectors and matrices}    \label{chap:matrices}


\vspace{-.7cm}
\section{Introduction}

This chapter covers two independent topics through examples. 
In the first place, some essential operations of matrix arithmetic are introduced, 
an operation like the product of two matrices becomes extremely simple 
in programming languages oriented to vector programming. 
In the second place, the concepts of static and dynamic data objects are covered. 
Notice that all the examples are written according to a declarative programming style, 
consider reproducing the same results with an imperative programming style to compare.
    
    \vspace{-.3cm}
    \subsection*{Matrix arithmetic}
    \vspace{-.2cm}

An array programming language or vector language has the possibility 
to operate a whole group of values at once.
The common operations that range over scalar numbers can then be applied to vectors, 
matrices and higher dimension arrays in a closer to maths style. 
Hence, mathematical expressions at the time of programming are simplified.

Languages like Fortran, MATLAB, R or the NumPy extension of Python support array programming.
Matrix arithmetic are built-in in these languages and expressing the mathematical language in a natural way is feasible.

\newpage
The following array operations are covered:
\begin{enumerate}
    \item Addition. 
    \item Dot product.
    \item Matrix multiplications. 
    \item Hadamard product. 
\end{enumerate}  


%Take note of the difference between array programming and array processors. The first makes reference to how the programmer 
%codes the mathematical operations in its program (a big advantage is obtained from this feature for Scientific Computing as mentioned before). 
%The second feature is related to how the processor operates that group of numbers, by performing
%all the operations together under the same instruction given to the processor in a considerably increase of speed.
%Both features suppose an increase of performance for the coding and executing of scientific programs, however, this chapter is 
%oriented to the aspects of the first feature mentioned. 



    \vspace{-.5cm}
    \subsection*{Static/Dynamic data objects}

Each data object declared in a program (variables, constants, pointers, arrays, etc.) are either static or dynamic, 
this means that the memory storage to hold that piece of data is reserved during compilation time or 
during the execution of the program respectively. 

Consider for example the Fortran variable \texttt{real :: A(N, N)} declared at the beginning of a code, 
its memory storage is reserved when the program is compiled and this space is not liberated 
until the program has finished the execution. 

Another option to declare and manage the memory storage of an object involves using dynamic allocation. 
In this case, the storage of the variable is not reserved until the program orders it (which happens during the execution of the code) and can be changed or freed at any moment. 
In Fortran for example, this is done through a code like \texttt{real, allocatable :: Ak(:, :, :)}.

This chapter also includes a review of the main memories that a program uses: Static, Heap and Stack.
All variables, program instructions, constants, etc. are stored in these memories and their use is intimately related 
to the different ways of allocating space in the code.



%___________________________________________________________________________________________________
\newpage 
\section{Static size vectors and matrices} 

Consider the same vectors $V, W \in \mathbb{R}^N$ and matrices $A, B \in { \cal{M}}_{N \times N} (\mathbb{R})$ used in previous chapters (with $N=10$): 
$$
V = \left[ v_i =\frac{1}{i^2}, \ \ i = 1 \ldots  N \right],
$$
$$
W = \left[ w_i = \frac{(-1)^{i+1}}{2i+1}, \ \ i = 1 \ldots  N \right].
$$
$$
A = \left[ a_{ij} = \left( \frac{i}{N} \right)^{j-1}, \ \ i = 1 \ldots  N, \ \ j = 1 \ldots  N \right].
$$
$$
B = A^T
$$

and let's compute the following operations:
$$
1.\;\sum_{i=1}^N v_i  \qquad \qquad 2.\;\sum_{i,j=1}^N a_{ij}   \qquad \qquad   3.\;\sum_{v_i>0} v_i   \qquad \qquad 4.\;\sum_{\scriptstyle i,j=1 \atop \scriptstyle a_{ij} > 0.1}^N a_{ij}   
$$
$$
5.\; V\cdot W^T    \quad \quad      6.\; V\cdot (a_{ij})_{\scriptstyle 1\leq i\leq N \atop \scriptstyle j = N}    \quad  \quad  7.\; AV     \quad \quad   8.\;\sum_{i,j=1}^N (A^2)_{ij}   \quad \quad    9.\; B = A^T
$$

$$
10.\;\max_{1\leq i,j\leq N} \{ a_{ij} \}       \qquad \qquad     11.\;\arg\max_{1\leq i,j\leq N} \{ a_{ij} \}   
$$

Since the sizes of these arrays are known at compile-time, the memory storage can be declared statically. 
The main properties of a static allocation are:

\begin{enumerate}
    \item The memory address and the size are assigned during the compilation of the code in the executable image of the program.
    \item These address and size can not be changed during execution.
    \item Once the program finishes the execution the space is freed.
    \item It is a simple and quick allocation process.
\end{enumerate}







        \newpage
        \subsection*{Fortran code}
The following code computes these operations using Fortran. 
Notice that these examples are limited to numeric arrays: integers, reals or complex,
however, arrays can be constructed with a different data type. 
Furthermore, new operations could be defined for other data types. 
    
Consult the specifications of the following matrix operations to become familiarized with the purpose, inputs and arguments.
Notice that the rank and dimension of the arguments must agree with the mathematical definitions of the operations.
As a quick reference: 
\begin{itemize}[noitemsep]
    \item \texttt{sum} adds all elements of an array: across one dimension, all the dimensions 
    or only the elements that accomplish with some condition (e.g. $v_i > 0$).
    \item \texttt{dot\_product} calculates the scalar product of two vectors.
    \item \texttt{matmul} computes a matrix multiplication. 
    \item \texttt{transpose} calculates the transpose of a matrix or vector. 
    \item \texttt{maxval} and \texttt{maxloc} return the maximum value from all the elements of an array and its position respectively.
    It can compute the result across one dimension, the whole array or only for those values that accomplish with a specified condition.
\end{itemize}

\lstfor
\renewcommand{\home}{./Fortran/sources/Foundations/Algebra} 
\listings{\home/Matrix_operations.f90}{subroutine Matrix_operation_examples}
      {end subroutine}{Matrix_operations.f90}

Let's remark some interesting notes of this code. 
First of all, \texttt{N} (which declares the size of the arrays) is declared as a named constant thanks to the \texttt{parameter} attribute. 
Hence, its value is fixed during the whole execution and a try to change it returns a compilation error. 

Secondly, to initialize both \texttt{V} and \texttt{W} their definition is written by means of an implicit loop (declared with parentheses). 
To define the matrix \texttt{A} two implicit loops are used: 
the nested loop \texttt{i} iterates in the rows while 
\texttt{j} jumps from one column to the next one. 
Both loops define a $N*N$ rank-one vector that needs to be reshaped to a ${\mathbb{R}}^{N\times N}$ matrix. 
The function \texttt{reshape} organizes the components of the vector into columns. 

Finally, notice how an implicit loop with a colon symbol \texttt{:} is also used
in the scalar product of the vector \texttt{V} with the $N$-th column of \texttt{A}. 
The implicit loop in \texttt{A} iterates in the \texttt{N} elements of the column. 
Another example would be how the matrix \texttt{B} is written in the screen,
it uses two loops (one explicit and one implicit) so each row of the matrix is represented element by element.
Compare this way with the output of the line \texttt{write(*,*) B} where all the elements are represented by columns and not by rows. 
This is due to the fact multidimensional arrays are stored in a linear storage and Fortran,
which is a \textit{column-major order} language, organizes the consecutive elements of a column one next to each other.






 
%Thirdly, take into account the difference between these operations and the element wise operations that can performed with 
%vectors and matrices. Some operations like the addition of two matrices are performed by adding the corresponding elements of both, also, 
%many elemental operations on numbers can also be applied to matrices, so the result is an equal shaped array 
%with the results of applying the operation to each element. Test the following expressions for matrix addition, Hadamard product (multiplying the corresponding elements of both matrices), cosine or square root of all elements of \texttt{A}:
%
%\begin{verbatim} 
%A + B
%A*B
%cos( A )
%SQRT( A )
%\end{verbatim}




%my matmul







         \newpage 
        \subsection*{Python code}
The same function can be written almost identically with Python. 
Notice from the following code how all vectors and matrices are constructed by using similar implicit loops. 
In addition, the same intrinsic functions are implemented so the use of mathematical notation when programming becomes natural.
Some particularities in this examples would be the indentation or the use of implicit typing. 
\vspace{0.5cm} 
\renewcommand{\home}{./Python/sources/Foundations/Algebra} 
\lstpython
\listings{\home/Matrix_operations.py}{def Matrix_operation_examples}
      {WARNING}{Matrix_operations.py} 








%_______________________________________________________________________________________________
\newpage  
\section{Dynamic allocation of vectors and matrices} 

Now let's use the square Vandermonde matrix  of $M\times M$:
$$
A_M =  \left[ a_{ M_{ij} } = \left( \frac{i}{M} \right)^{j-1}, \ \ i = 0, \ldots M-1, \ \  j=0, \ldots M-1 \right],  
$$
to compute the following operations:
$$ 
S = \sum_{M=1} ^{10} \Tr(A_M)  \qquad
S = \sum_{M=1} ^{5} \Tr(A_M^2) \qquad
S = \Tr \left( \sum_{k=0} ^{5} A_M^k  \right)  \qquad  M=8 
$$

Imagine now that the size of the matrix involved in your code is not known at compile-time, maybe it comes from 
an user input, 
an external file or 
it is the result of a previous operation. 
In this case, dynamic data objects are used so its memory storage (address, size, etc.) is allocated or modified during the execution of the program.
In Fortran for example the essential statements used to manage the memory are 
\texttt{allocate} and \texttt{deallocate} while the attribute for the data object is \texttt{allocatable}. 
The main properties of dynamic data objects are:

\begin{enumerate}
    \item The program decides how much memory reserve for a data objects so it can accommodate any size with no need of re-compiling the code.
    \item Modifications for the memory size are allowed. 
    \item It becomes the programmer responsibility to liberate the memory reserved for non-used arrays in order not to run out-of-memory.
    \item It is generally slower than static allocation. 
\end{enumerate}

The following examples, whether coded in Fortran or Python, are based on the following functions: 
\begin{itemize}[noitemsep]
    \item \texttt{Vandermonde( M )} which computes the Vandermonde matrix of a given dimension \texttt{M},
    \item \texttt{trace( A )} to compute the trace of a matrix and
    \item \texttt{power( a, k )}, a recursive function to obtain the $k$-th power of a matrix.
\end{itemize}
Once each function is coded, more general operations are easy to implement. 
In addition, remember that each abstraction can now be used in any program where this algebra is needed from now on. 




    \newpage 
    \subsection*{Fortran code}

The subroutine \texttt{Matrices\_allocation()} computes the mentioned operations in Fortran:
\lstfor
\renewcommand{\home}{./Fortran/sources/Foundations/Algebra} 
\listings{\home/Dynamic_allocation.f90}{subroutine Matrices_allocation}
{end subroutine}{Dynamic_allocation.f90}

Let's take a deeper look into the program. 
Notice that the 3-rank object \texttt{Ak} is declared with the attribute \texttt{allocatable} and colons \texttt{:} instead of the dimension specifications. 
Later, when the dimensions are known, the \texttt{allocate} function is used to reserve the appropriate memory for the variable.

For the first operation, the expression \texttt{trace( Vandermonde(M) )} is coded inside an implicit loop from 1 to 10. 
These results are constructed in a vector (using squared brackets) and 
\texttt{sum} performs the summation of the components of that vector. 
In addition, notice the use of few variables thanks to a declarative style. 
In an imperative style each Vandermonde matrix would be explicitly stored in a variable \texttt{AM(M,M)} and
their traces in a real vector variable used as an input for \texttt{sum} function. 
%Do not think that these objects are not used now, the compiler still needs them. 
%However, it automatically reserves memory, stores the intermediate results, returns only the single result needed \texttt{S} and, 
%when the calculation is finished, frees all that temporary memory used in an efficient way. 

The second operation is similar with the exception that now the trace is calculated on the product of two matrices. 
%Fortran, like any common scientific language already has implemented the product between two matrices of reals. 
A vector programming style has advantages to code algebra expressions as we know, 
but it also has a better performance when computing multiple operations if vector processors are used.
This is due to an increase in efficiency when the same operation is performed over a bunch of numbers. 

\newpage
For the third operation the variable \texttt{Ak} is dynamically allocated with bounds 0 to 5 in the third dimension. 
While normally the indices of any array starts in 1, the programmer may decide to modify it 
so the index automatically responds to a mathematical sense.
In this case, starting in 0, we allow \texttt{Ak(:,:,k)} to be a reference to the $k$-th power.
%with no need of remembering in which index starts and how many powers are we calculating. 
%If the needed powers of Vandermonde would only have been 4, 5 and 6, we could have allocated the matrix like: \texttt{allocate( Ak(8, 8, 4:6) )}. 
Finally, notice the complete array assignation performed with \texttt{Ak}. 
Once it is properly allocated, writing \texttt{Ak(:,:,k) =} is enough to store the result since it is previously known that the result is an $8\times8$ matrix. 
%Also, it is mandatory to specify to the function \texttt{sum} that the summation is only performed in the third dimension of \texttt{Ak} so it is adding one matrix to the next one. 

\vspace{0.5cm}
\listings{\home/Dynamic_allocation.f90}{function trace}
{end function}{Dynamic_allocation.f90}

\listings{\home/Dynamic_allocation.f90}{function power}
{end function}{Dynamic_allocation.f90}
For the functions \texttt{power( A, k )} and \texttt{trace( A )} two concepts should be extracted. 
Firstly, both must be used only with square matrices, where both mathematical operations are defined. 
Secondly, the concept of recursion is used to calculate the $k$-th power of a matrix. 
Essentially, the $k$-th power of a matrix is the multiplication of that matrix with his $k-1$-th power. 
A \texttt{recursive} statement is used in the declaration so the compiler knows that this function can call to itself to compute smaller instances of itself.
In this example it is going to compute the $k-1$-th power when trying to compute the $k$-th, 
the $k-2$-th power when trying to compute the $k-1$-th and so on until the identity matrix is reached (power 0). 



    \newpage
    \subsection*{Python code}
The same functions are written with Python in a similar style:
the use of implicit loops to construct arrays, 
the use of similar intrinsic functions or 
the need to define those functions not built-in in the language.
In this sense, the function \texttt{trace} belongs to the \texttt{numpy} library so there is no need to code it.

\vspace{0.5cm}
\lstpython
\renewcommand{\home}{./Python/sources/Foundations/Algebra} 
\listingsp{\home/Dynamic_allocation.py}{def Matrices_allocation}
{return}{Dynamic_allocation.py}

\listingsp{\home/Dynamic_allocation.py}{def power}
{matmul}{Dynamic_allocation.py}



\newpage
\section{Memories: Static, Heap, Stack}

It is clear that during the execution of the program certain amount of memory is needed. Data objects or the source code 
must be stored somewhere. The compiler, the linker and the operating system of the machine decides where each piece of data is stored. 
Three regions of the memory can be distinguished for a program (see Figure \ref{fig:Memories}): 
static, heap and stack, each one related to one type of allocation. Notice that the last two memories are dynamic in nature.

Static and dynamic allocation have been already treated. A third way to allocate memory size for data objects is used: 
stack (or automatic) allocation. The compiler is constantly using it 
in order to store temporary arrays used 
inside subroutines/functions or needed for array expressions.


\begin{figure}[h]
    \centering
    \includegraphics[width= \textwidth]{./doc/Figures/Memories.png}
    \caption{Three memory regions used by a program.}
    \label{fig:Memories}
\end{figure}

Let's revise some properties, advantages and problems associated to these memories:

\begin{itemize}
    \item \textbf{Static:} it is usually located in the low end of the memory reserved for the application. The compiler creates a list of variables to be allocated during compilation and gives them a fixed address 
    so the whole program can use these variables at any time and faster. 
    
    This static allocation requires knowing the amount of memory needed at compile-time.
    
    \item \textbf{Heap:} when a data object is allocated, the application requests an amount of memory to the system, if there is space available the system answers with the 
    starting address where storing the data. Once that memory is not needed any more, the program can ask the system to free that memory under the \texttt{deallocate} statement (in the case of Fortran) so
    that space becomes available for the next time. Notice that the amount of space for a program is bigger than the stack, but not unlimited, here concepts like virtual memory and swap play an important role. 
    
    For the heap a memory leak can happen if no deallocation is performed (the memory continue being used with not useful data) and it is normally slower than static allocation. 
    However, the programmer manages this space and the program decides how much memory to use for each purpose. 
    
    \item \textbf{Stack:} it is composed by a limited amount of memory and a pointer that holds the current position where routines can store local variables. This space is filled from the top to the bottom of the memory so any time that a subroutine or function (let's say \texttt{function A}) needs from temporary storage this space is used, the pointer is decremented and the stack space is reduced. If this function calls a nested \texttt{function B}, then its local variables are stored below variables of \texttt{A} and the pointer is decremented once again. Once the \texttt{routine B} finishes its operations, the pointer is incremented again below the spaced filled by \texttt{routine A} so that memory is available for the next routine. This structure is called LIFO; 'Last-In,-First-Out'.  
    
    The process is efficient and fast and the space is automatically freed when the routine returns to its host. This allocation is performed by the machine but the programmer usually has the option to declare some variables as \texttt{AUTOMATIC} in subprograms so they reside in the stack area. On the contrary, the amount of space is limited by the linker in the case of Windows and the programmer must be careful with allocating more space than it is available in order to avoid stack overflow. 
        
\end{itemize}


Stack overflow, when the memory allocated on the stack overflows into other memory regions, usually happens with two situations; extremely deep (or infinite) recursion and large array variables. 

%Primer ejemplo de stack overflow


The second example dynamically allocates a $400\times 400\times 400$ of 4-bytes reals array in the heap (\texttt{R})  but then the function \texttt{StackOverflow\_size( n )} tries to locate a similar automatic array (\texttt{S}) in the stack leading to a Stack Overflow error. 

\begin{verbatim}
real, allocatable :: R(:,:,:)
integer :: n

n = 400
allocate( R(n, n, n) )

R = StackOverflow_size( n ) 
\end{verbatim}

with the function:

\begin{verbatim}
function StackOverflow_size( n ) result(R)
integer, intent(in) :: n
real :: R(n, n, n)    

    R(:, :, :) = 1.

end function
\end{verbatim}

Some strategies can be used to avoid this error:

\begin{enumerate}
    \item Try to reduce the abuse of stack; allocate automatic arrays that usually goes to stack so they are located in heap. Then, they are automatically deallocated at the end of the routine. 
    
    \item The size of the Stack memory can be increased through a linker option. In Visual Studio for example it can be written: \texttt{/STACK:100000000} specifying the size in bytes desired.
    
    \item A compiler option can be used to change the default storing place for automatic arrays and temporary arrays so they are automatically located in the heap: \texttt{/heap-arrays}. If a kilobytes size is specified, only larger arrays are allocated to heap (i.e. \texttt{/heap-arrays100} to only affect arrays larger than 100 kilobytes). Use the value \texttt{0} to apply the behaviour to all arrays.  
    
\end{enumerate}


Now try to execute the second example with the option \texttt{/heap-arrays0} specified in the compilation options. For the case of Visual Studio it can be done in the project properties by clicking on Configuration Properties/Fortran/Command Line and adding that line.


%-------------------------------------------------------------------------------------------------------------
%MAYBE INTERESTING IN FUTURE:

%When using AUTOMATIC, SAVE, STATIC, etc.
%\item How to manage it in your program and recommendations.
%\item How to reserve more space in each of them.




%-------------------------------------------------------------------------------------------------------------
%Revise the following concepts:
%
%\begin{enumerate}
%    \item Who manages the memory.
%    \item Where are physically stored (RAM, swap).
%    \item Which is faster to allocate and use.
%    \item Common problems related to the use of that memory.
%    \item How to manage it in your program and recommendations.
%    \item How to reserve more space in each of them.
%    \item etc.
%\end{enumerate}


%  Fortran for example uses this memory to create space for local arrays (those based on arguments of routines) or for temporary copies in array expressions. 
%     the stack is managed by the CPU, there is no ability to modify it
% variables are allocated and freed automatically
%  
  
         \input{./doc/Chapters/part_I/integrals_derivatives.tex}
         %___________________________________________________________________________________________________
\chapter{Series expansion} 


%___________________________________________________________________________________________________
\section{Introduction} 

One of the main focus of numerical methods is to approximate functions by means of different expansions: 

\begin{enumerate} 
\setlength\itemsep{-0.1cm}
	\item Polynomial expansion: Taylor, Lagrange.
	\item Trigonometric series: sine and cosine expansions. 
	\item Expansion by means of known basis: Chebyshev, Legendre, etc. 
\end{enumerate} 

The same function can be expanded or approximated with different basis.
It goes without saying that the computer only allows to sum a finite number $ N $  of terms. 

Hence, the best election is related to the rate of convergence 
of the expansion. In other words, given a tolerance error between the approximation and the exact function, 
the best expansion allows to obtain the approximation with a minimum number of terms $N$. 




%___________________________________________________________________________________________________
\newpage
\section{Expansions of functions} 
   
One of the most used series expansion of a function is its Taylor series, without going any further, in section \ref{sec:intder} we have used a Taylor expansion to estimate the truncation error committed when approximating the derivative of a function by its finite difference expression.

In general, as already mentioned, a series expansion is a very useful way to approximate the value of a given function in a domain. More specifically, the Taylor series, which uses power series, allows us to estimate the value of a function at one point from its value and the value of its derivatives at a different point. Polynomials are extensively known and they are easy to derive, integrate, multiply, etc. so treating a function as its power series approximation can have many advantages when it comes to operating with that function.

A power series of the function $f(x)$ can be expressed as:
%\[  f(x) = \sum_{k=0} ^N a_k \  x^k, \qquad \qquad a_k = \frac{  f^{(k)} (0)  }{ k! },  \]  
 
\[  f(x) = \sum_{k=0} ^N a_k \  x^k,  \] 

where $a_k$ are real numbers that depend only on $k\in \mathbb{N}$ and not on $x$. For the case of a Taylor series around the point $a = 0$, these coefficients are calculated as:

$$
a_k = \frac{  f^{(k)} (0)  }{ k! }  
$$

being $f^{(k)} (0)$ the \textit{k}-th derivative of $f(x)$ in the point $a = 0$.
 
For a Taylor expansion around a generic point $a$ the shape of the polynomial series and the coefficient expressions are:

\[  f(x) = \sum_{k=0} ^N a_k \  (x - a)^k, \qquad a_k = \frac{  f^{(k)} (a)  }{ k! }   \] 




  
\renewcommand{\home}{./Fortran/sources/Foundations/Calculus} 
   \listings{\home/Taylor_expansions.f90}{function TaylorE}
   {end function}{Taylor_expansions.f90}
   
   
   \listings{\home/Taylor_expansions.f90}{interface}
   {end interface}{Taylor_expansions.f90}
   
   
   
   \listings{\home/Taylor_expansions.f90}{function TaylorN}
      {end function}{Taylor_expansions.f90}
  
  
  
   \listings{\home/Taylor_expansions.f90}{interface Taylor}
         {end interface}{Taylor_expansions.f90}
   
   
   
   
   
   

%___________________________________________________________________________________________________
\section{Parseval identity} 





\begin{equation} 
	f ( x)  =  \sum_{k=-\infty} ^{\infty}  \hat{c}_k  e^{ i k x }
\end{equation} 

\begin{equation} 
	f ( x)  =  \sum_{k=0} ^{\infty} \  \hat{a}_k  \ \cos k x + \sum_{k=1} ^{\infty} \  \hat{b}_k  \sin k x
\end{equation} 	

\begin{equation} 
	\hat{c}_k  =  \frac{1}{2} \ ( \ \hat{a}_k  - i \ \hat{b}_k \ ),  \quad k=1, \ldots, N/2. 
\end{equation}	

\begin{equation} 
	\hat{c}_{-k}  =  \overline{ \hat{c}} _{k}  , \quad k=1, \ldots, N/2. 
\end{equation} 	


\begin{equation} 
	\int _{-\pi} ^{\pi} f(x)^2 \ dx = 2 \pi \left(    | \hat{c}_0 |^2 + 2 \sum_{k=1} ^{\infty} |  \hat{c}_k  |^2 \right) 
\end{equation} 

Let $ f(x) = x $ $ \forall x \in [-\pi, +\pi ] $ and extend $ f(x) $  periodically to the right and to the left. Once $ f(x) $ is defined in 
this way, it can be approximated by a sine expansion: 
\begin{equation} 
	f ( x)  =  \sum_{k=1} ^{N/2} \   \hat{b}_k  \sin k x, 
\end{equation} 
where the coefficients $ \hat{b}_k $ are given by:
\begin{equation} 
	\hat{b}_k   = \frac{ (-1)^{k+1} }{ \pi k }, \qquad k=1, \ldots, N/2. 
\end{equation} 

Using Parseval identify with $ f(x)$, the following  summation is obtained: 
\begin{equation} 
	\sum_{k=1} ^{\infty} \  \frac{1}{n^2}  \  = \frac{\pi^2}{6}
\end{equation} 



         
\newpage   
\chapter{Reading and writing external files}    \label{chap:readwrite}



Reading and writing data from/to external files is usually an essential part of every software project. Maybe you want to process data returned by a different software, share information between two programs written in different languages, process the results from your program in an specialised plotting software or just reading the inputs of your program through a data sheet. 

The aim of this chapter is to propose an orderly way to encapsulate and use a matrix reading function. As a result, this function hides the particularities of Fortran (or any other programming language where imitating this function) related to reading/writing information from files and builds an useful tool that can be easily recycled for future projects. A one-time effort to create a powerful and general tool is worthwhile if you are using a programming language regularly. Additionally, encapsulating and separately validating this tool reduces failures and helps to standardize the way of proceeding each time that a reading operation is performed.

The following subroutine loads two matrices by declaring the relative path of the \texttt{.csv} file where information is stored and then writes them in the screen by rows. Notice that the dynamic object \texttt{A} in the subroutine does not need to be allocated since the compiler automatically takes care of that when the output of the reading function is available. Also, notice that the matrix is written using an explicit loop for rows and an implicit loop for columns. This technique is preferred since Fortran writes by default the data by columns. 

\newpage 
\vspace{0.5cm} 
\renewcommand{\home}{./Fortran/sources/Foundations/read files} 
\listings{\home/read_files.f90}{subroutine Test_load_matrix}
{end subroutine}{read_files.f90}


%-------------------------------
The function \texttt{load\_matrix()} is in charge of reading the data and assigning it to a local variable \texttt{A}. Contrary to the subroutine, here the dynamic object \texttt{A} is allocated by hand. 

In summary, this function opens the requested file (input argument), reads the first line and saves the information in the variable \texttt{header}. This header determines the number of columns in which the data is ordered (using commas as separators). From now on, all rows are read until the total number is determined and the size of array \texttt{A} can be defined. Finally, the whole file can be reread again, saving the data in \texttt{A} this time. This function makes use of an auxiliary function called \texttt{columns( string, s )} which reads how many columns are declared in the header of the file using some specific separator. 

Encapsulating some tools like this one allows the programmer to save time and reduce possible sources of error in future projects but does not eliminate the need to know the peculiarities of reading/writing data in files or screen, as is the case with any programming language. For the needs of each project some adjustments could be done.

\newpage
\vspace{0.5cm} 
\renewcommand{\home}{./Fortran/sources/Foundations/read files}
\listings{\home/read_files.f90}{function load_matrix}
{end function}{read_files.f90}

\renewcommand{\home}{./Fortran/sources/Foundations/read files}
\listings{\home/read_files.f90}{function columns}
{end function}{read_files.f90}

  
  
  
  
\newpage 
\subsection*{Python code} 
The following examples reproduce the same behaviour with Python, notice that many functions are already implemented in different packages of Python. 
In this case the function \texttt{load\_matrix()} encapsulates the load of the specific file through \texttt{read\_csv()} (included in Pandas package), prints the header and returns the needed values. As mentioned before, encapsulating tools like this one allows you to save time and reduce possible sources of error in future projects by creating an standard for all your programs. 


\vspace{0.5cm}
\renewcommand{\home}{./Python/sources/Foundations/Read_files}
\lstpython 
   \listingsp{\home/Read_files.py}{def Test_load_matrix}
    {end}{Read_files.py}
  
 \renewcommand{\home}{./Python/sources/Foundations/Read_files} 
    \listingsp{\home/Read_files.py}{def load_matrix}
     {return}{Read_files.py} 
  \lstfor
  
  
  
 %   A batch file named: \mycap{Preliminary_examples.bat} allows to compile and to run this main program.  
 
 
 % \newpage  
 
 
 
 
 %   \subsection{Windows}
 
 
 % _______________ RUN EXAMPLES_________________________________________________
 
 %   >compile_Cauchy_Problem_examples.bat
 %   
 %   >compile_BVP_examples.bat 
 %   
 %   >
 %   
 %   >
 %   
 %   
 %   
 %   Examples of use compile.bat 
 %   
 % %  _________BASIC USAGE___________________________________________________________ 
 %   
 % %  >compile main 
 %   
 %   The file main.f90 contains the main program to be executed
 %   
 %%   e.g. to compile and to execute the "hello world" first example 
 %   
 %   >compile main    
 %   
 %   
 % %  ___________ADVANCED USER______________________________________________________
 %   
 %   If the user develops a new module, or he wants to use existing examples,
 %   the compilation procedure is as follows: 
 %   
 %%   > compile module_name main 
 %   
 %   1) module_name is a fortran module file which is located in sources. 
 %   
 %   2) main file is the main program.
 %   It contains a use setence to access module_name and it calls some subrutine of  the module 
 %   
 %   
 %   e.g. 
 %   > compile 
 %   
 
 
 
 
 
 
 %   
 %   
 %   
 %   
 %   Informatics 2017/2018 Aerospace Engineering (UPM) 
 %   
 %   How to install FORTRAN and graphical windows in windows 10 
 %   
 %   Author: Juan A. Hernandez:  juanantonio.hernandez@upm.es 
 %   
 %   
 %   
 %   %   A) Copy Informatica\_2018 to Desktop 
 %   
 %   B) Check the compiler 
 %   1)open command prompt and type >compile Hello\_world
 %   2)open  command prompt and type >compile plot\_graph
 %   
 %   
 %   \subsection{Ubuntu}
 %   
 %   
 %   
 %   Informatics 2017/2018 Aerospace Engineering (UPM) 
 %   
 %   How to install FORTRAN and graphical windows in Ubuntu from windows 10 
 %   
 %   Author: Juan A. Hernandez:  juanantonio.hernandez@upm.es 
 %   
 %   
 %   
 %   
 %   A) Install UBUNTU from Windows 10 (skip this step if your machine is UBUNTU) 
 %   
 %   1) Setting/Security/Developer ( allow to create and modify programs ) 
 %   
 %   2) control panel/programs/Windows characteristics allow UBUNTU 
 %   
 %   3) cmd> bash . Create user and password UBUNTU machine 
 %   
 %   4) Install X11 server for graphical windows ( XMing ) to run graphical linux binaries from shell.
 %   Capability to start 64-bit GUI-programs from the desktop file manager 
 %   
 %   5) Install graphical vim editor to check: 
 %   
 %   \$sudo apt-get install vim-gtk
 %   
 %   Now, you?ll need to set the ?DISPLAY? environment variable to point at the X server running on your Windows 10 PC. 
 %   If you don?t do this, graphical applications will simply fail to launch.
 %   To do this, run the following command in the Bash environment:
 %   
 %   \$export DISPLAY=:0
 %   \$gvim
 %   
 %   
 %   B) Install gfortran 
 %   
 %   1) \$ sudo apt-get install gfortran 
 %   
 %   
 %   
 %   C) Install DISLIN to plot with UBUNTU 
 %   
 %   1)  should install OpenMotif on Ubuntu 
 %   
 %   
 %   \$ sudo add-apt-repository deb http://archive.ubuntu.com/ubuntu lsb\_release -sc universe multiverse
 %   
 %   \$ sudo apt-get update 
 %   
 %   \$ sudo apt-get install libxm4
 %   
 %   \$ sudo apt-get install libmotif4* libmotif-dev
 %   
 %   \$sudo apt-get install libx11-dev libxt-dev libgl1-mesa-dev  WARNING:(l1 ele and then one )
 %   
 %   
 %   2) Check the compiler 
 %   
 %   1)\$ ./compile.sh Hello\_world
 %   2)\$ ./compile.sh plot\_graph
 %   
 %   
 %   
 %   \subsection{MAC} 
 %   
 %   Informatics 2017/2018 Aerospace Engineering (UPM) 
 %   
 %   How to install FORTRAN and graphical windows in MAC OS 
 %   
 %   Author: Juan A. Hernandez:  juanantonio.hernandez@upm.es 
 %   
 %   
 %   1) How to execute a Terminal 
 %   
 %   Applications/ Utilities/ Terminal
 %   
 %   2) How to execute a Terminal in a specific folder
 %   
 %   System Preferences/ Keyboard/ Speed functions/ Services - scroll and select "Terminal in the folder"
 %   
 %   
 %   
 %   3) Install gFortran to compile
 %   
 %   3.1) Install manager HOMEBREW (https://brew.sh)
 %   3.2) \$ brew install gcc
 %   
 %   
 %   4) Install Openmotiff
 %   
 %   4.1) Install server X11 through XQuartz (https://www.xquartz.org)
 %   Capability to start 64-bit GUI-programs using x-windows from the shell 
 %   
 %   4.1)\$ brew install openmotif
 %   
   
         
 
  \part{Computer operations with integers and reals}\label{PartII} 
       %--------------------------------------------------------------------------------------------------------------------------------------
\chapter*{Menu for Integers and Reals} 

%\vspace{-1.3cm}
Being the base of any numerical calculation that a computer scientist wants to perform, 
both, integers and reals, must be a mastered subject.  
This part of the book covers the particularities of these two data types giving special
attention to the common errors that appear when programming maths in the computer. 

How the numbers are treated by computers is similar for most machines in the world. Although 
Fortran is used for the examples, any programming language will carry with the same issues 
in a similar way. Here you will find some useful notions to acquire a better understand of 
the behaviour of integers in a computer, the arithmetic behind different data types or the 
phenomena associated to the approximation of numbers by a finite precision. 

Make use during the reading of the chapter of the attached programs, there you will find the 
same examples explained here and the possibility to change and write your own programs 
using the existing codes. The following is the menu implemented in the program and also
serves as an overview of the topics covered in the following pages. 

\vspace{0.3cm}
 \renewcommand{\home}{./Fortran/sources/IEEE} 
  \listings{\home/integers_and_reals.f90}{select an option}{exception}{integers_and_reals.f90}


       %--------------------------------------------------------------------------------------------------------------------------------------
 \renewcommand\home{./Fortran/sources/IEEE}

\chapter{Integers representation} 

    \addcontentsline{toc}{section}{Overview}
    \section*{Overview} 
    
It is well known that the programs are written with numbers in decimal form. However, the computer, 
in order to work with these constants and variables, is going to convert numbers to binary form. 
Using two voltage levels (representing 0`s and 1`s) as unique states the computer can perform calculations that, 
in order to show to the user the results obtained, will be translated again to decimal form. 

The computer translates decimal values to binary similarly to the way humans typically do (with some exceptions to be noticed). For 
positive integers the way to translate a value uses the principles of positional numeral systems, the value of each digit is totally determined by the 
digit itself, the position in the number and the base of the numeral system (10 and 2 for decimal and binary). In the case of negative integer values, 
while a person could write a sign (-) before the number, the machine uses only 0`s and 1`s and how this is done is treated in the chapter.

On one hand, not all the numbers that can be written in a program (actually, 
much less numbers than the infinite possibilities) are exactly represented 
in the binary translation that the computer performs. It is
essential for a programmer to understand this issue.
On the other hand, integers do have an exact representation in binary form 
with an integer variable  
but its maximum and minimum value depends on the memory size of this variable.  
The main problematic that motivates the study of the integer representation in 
computers is related to overflow, values out of the representable range. 

The following topics are covered in this chapter: 

\begin{enumerate} 
\setlength\itemsep{-0.1cm}
    \item Integer overflow example.
    \item Two's complement integer representation.
    \item Understanding integer overflows.
    \item Overflow of constants.
    \item Overflow by incorrect assignment.
    \item Declaring kind
    \item Two's complement converter
\end{enumerate} 
    
Throughout this chapter the reader will be able to understand and justify the corrupted values stored in a variable when an out-of-range integer is assigned by mistake. When this error occurs during compilation time, in many cases, the compiler does not abort the compilation, it only prints a warning. In the worst situation, the error appears during execution, the program is not working properly and this source of error could be a nightmare to trace.
 
 
 
   
\newpage
    \section{Integer overflow example}


The range of integer variables is limited by its memory size representation. 
When programming and by mistake is possible to assign a value out of the
range of some integer variable.
It is very important to understand how the programming language treats this exception 
to understand what is going on in our execution. 
In the following example, the integer variable \texttt{i} is stored in 1-byte 
 of memory 
whereas  \texttt{k} is stored in 2-bytes.

\vspace{0.5cm}

\listings{\home/Integer_representation.f90}{subroutine Integer_overflow}
{end subroutine}{Example of Integer overflow. Integer_representation.f90}



A single loop prints on screen 
the successive values of \texttt{k} and \texttt{i}. Once the value +127 
of \texttt{i} is reached, negative numbers of \texttt{i} appear even when we are
incrementing step by step the variable \texttt{k}.  
The successive  values of \texttt{i} after $+127$ are $-128, -127,  \dots$ 

To understand this fact, we should consider that integer numbers of 1-byte size 
are treated as numbers 
modulo $ 2^8 $ which means that if integer numbers were unsigned, their range would be 
from 0 to 255. 
If this unsigned integer variable were assigned with a number \texttt{k}
greater than 255, the resulting value would be k mod 255 (remainder of k divided by 255). 
Additionally, 
to hold signed integer variables (positive and negative values), 
a two's complement notation is generally used 
that restrict the integer values from -128 to +127 in case of 1-byte size variables. 
 



Another classical example that disconcerts to the novel programmer is the overflow with classical 
mathematical operations such as the  factorial of an integer value: 
$$
 {\displaystyle n \ !=n\cdot (n-1)\cdot (n-2)\cdot (n-3)\cdot \cdots \cdot 3\cdot 2\cdot 1\ }  
$$
In the following code, 
the factorial of 20 is obtained by multiplying the integer variable \texttt{f} 
sequentially from 1 to 20. The partial results are shown on screen. After the variable \texttt{k} 
has reached 13, the variable \texttt{f} shows wrong results. 
\vspace{0.5cm}
\listings{\home/Integer_representation.f90}{subroutine Factorial_overflow}
{end subroutine}{Factorial overflow. Integer_representation.f90}
Once the result has reached the maximum value the integer variable \texttt{f} can hold, even negative 
numbers appears being the result of the modular arithmetic with two's complement. 
It is important to notice, that in this case, the integer variable \texttt{f} is stored 
in 4 -bytes which is the default storage of an integer value. 


When an integer overflow occurs, the execution is not aborted 
and the programmer is not alerted of the malfunction. 
This is the main problems of integer overflow occurrences.  
Generally, and integer overflow is associated to the following situations: 
\begin{enumerate}
\setlength\itemsep{-0.1cm}
\item Counters to measure time or repetitive processes. 
     If these counters are used in a perpetual control system, they are  eventually overflowed.
\item Mathematical operations in the set of integers $ \mathbb{Z} $ such as the factorial of a number. 
\item Careless assignments between reals and integers.
\end{enumerate}

When program codes are used as controlling units in embedding systems, 
the overflow could origin terrible accidents such as the crash of the 1996 maiden 
flight of the Ariane 5 rocket. 

On May 2015, the European Aviation Safety Agency discovered that the embedded 
software of the Boeing 787 suffered after 248 days an integer overflow  of a  counter of 
a signed variable of 32-bits. 
This overflow required to reset periodically the system to avoid 
loss of electrical power and ram air turbine deployment. 



The two's complement integer representation is explained in the next section. 

 
\newpage
    \section{Two's complement integer representation} \label{sec:TwosComp}

An integer variable declared as 4-bytes (32 bits) size has $ 2^{32} $ possible different values. 
Some languages allow to work with unsigned integers. In that case, 
all bits are reserved for positive values and its range goes from 0 to $ 2^N - 1$. 
When specifying signed integers, one bit has to be reserved for the sign and then the range is divided by two.
Hence, an integer variable stored in 4-bytes has the range  [-2 147 483 648, 2 147 483 647]. 
The Table \ref{tab:propertiesint} summarizes the main parameters for signed integers.

\begin{table}[h]
    \centering
    \begin{tabular}{| r | c | c |}
        
        \hline
        Name  & Minimum ($-2^{N-1}$) & Maximum ($2^{N-1} - 1$)  \\ \hline
        
        \texttt{ Kind = 1}                         & $-128$ & $127$ \\ \hline
        \texttt{ Kind = 2}                      & $-32 768$ & $32 767$ \\ \hline
        \texttt{ Kind = 4}               & $-2 147 483 648$ & $2 147 483 647$ \\ \hline
        \texttt{ Kind = 8}   & $-9 223 372 036 854 775 808$ & $9 223 372 036 854 775 807$ \\ \hline
        
    \end{tabular}                                                       
    \caption{Main properties of the different signed integer kinds used (two's complement used for negative values).}
    \label{tab:propertiesint}
\end{table}

In general, the range of an integer variable stored in N bits is:   
\begin{equation}
    [ -2^{N-1},\: 2^{N-1}-1 ].  
\end{equation}



The Table \ref{bits} represents an integer storage with N-bits where $ b_i $ denotes the possible values 0 or 1. 

\begin{table}[h]
    \centering
    \begin{tabular}{| c | c | c | c | c | c | c | c |}
        \hline
        $ b_{N-1} $  & $ b_{N-2} $ &  $ b_{N-3} $ &\ldots  & \ldots & $ b_{2} $  & $ b_{1} $ & $ b_{0} $ \\ \hline 
    \end{tabular}                                                       
    \caption{Binary representation of an integer number stored in N bits.}
    \label{bits}
\end{table}

The general expression to convert this binary number to decimal when it is expressed in two's complement is given by: 
\begin{equation} 
 x = - b_{N-1} \ 2 ^{ N-1 } \ + \ \sum_{i=0} ^{ N-2} \ b_i \ 2^i. 
 \label{reconstruction}
\end{equation} 

As an example, let's consider an integer value equal to $425_{10}$ (subscripts $_{10}$ and $_2$ means decimal and binary value respectively) stored in 2-bytes of memory.
The binary form comprises 16 different bits to hold the integer number. As a signed positive integer 
it is represented with common binary numeral system: 
\begin{multline*}
    \texttt{ 0000000110101001}_{2} = 1\cdot 2^8 +1\cdot 2^7+ 0\cdot 2^6+ 1\cdot 2^5+ 0\cdot 2^4+ 1\cdot 2^3+  0\cdot 2^2+ \\ + 0\cdot 2^1+ 1\cdot 
    2^0 = 2^8 + 2^7 + 2^5 + 2^3 + 2 = 425_{10}
\end{multline*}
where the leading zeroes are not displayed when printing that number on your screen. 
The same number stored in a variable of 4-bytes (32bits) of memory size
has the same binary representation but with more leading zeroes. 
Notice that a signed integer variable of 1-byte size can not store this number 
since its range is  $[-128, 127]$. 




The following two questions arise:
\begin{enumerate} 
\setlength\itemsep{0cm}
\item How are negative integer numbers stored in the computer? 
\item How is overflow treated with integer numbers? 
\end{enumerate} 

To answer these two questions, we should understand how signed integer  
numbers are stored in the computer. 
There are two main ways to store signed integer variables in memory: 

\begin{enumerate}
\setlength\itemsep{0cm}
\item One's complement integer representation. 
\item Two's complement integer representation.
\end{enumerate}

In the following table, a signed integer variable of 3-bits of memory size is represented in binary form, 
unsigned integer, two's complement and one's complement. 

\begin{center} 
    \begin{tabular}{|c|c|c|c|}
        \hline
        Binary  & unsigned & two's comp.& one's comp.  \\
        \hline
        \texttt{000} & 0 & 0 & 0 \\
        \hline
        \texttt{001} &  1& 1 & 1 \\
        \hline
        \texttt{010} & 2 & 2 & 2 \\
        \hline
        \texttt{011} & 3 & 3 & 3 \\
        \hline
        \texttt{100} & 4 & -4 & -3 \\
        \hline
        \texttt{101} &  5 & -3 & -2 \\
        \hline
        \texttt{110} &  6 & -2 & -1 \\
        \hline
        \texttt{111} &  7 & -1 & -0 \\
        \hline
    \end{tabular}
\end{center} 


The two's complement notation is, generally, used in computers. 
If the leftmost bit is \texttt{1}, the integer variable holds a negative number.
From the above table, a procedure is deduced to obtain the binary form of a negative number. 
For example, the binary form of -2 is obtained from the binary form of 2 which is \texttt{010}. Then, 
these bits are inverted to give \texttt{101}. An finally, \texttt{001} is added to the resulting number to give: 
\texttt{110} which is -2 in the two's complement.

\begin{IN}
   To obtain the binary form in two's complement of a negative number, carry out the following steps: 
   \begin{enumerate}
       \item Obtain the binary form of the integer number without sign. 
       \item Invert all bits by using the bitwise NOT operation. 
       \item Add 1 to the resulting value. 
   \end{enumerate}
\end{IN}

The signed negative number $-425_{10}$, using two's complement, is translated in the machine like:

\begin{equation}
    \texttt{1111111001010111}_{2} = -425_{10}
\end{equation}

Notice that this technique to encode signed numbers defines the 'two's complement' of a specific number 
with N bits as its complement with respect to $2^N$ which means that adding a number to its complement 
results in $2^N$. From the different ways to represent signed numbers, this method performs addition, subtraction, and multiplication 
in the same way for signed and unsigned binary numbers and it lacks from the negative zero. 




Once the treatment of negative numbers is understood, the next step is to understand how the computer 
treats an overflow with integers.  
The most common result of an overflow is that the least significant representable digits 
of the result are stored. Imagine that integer value 5 is stored in a 4-bits x-variable:
$$
       x = 0101.
$$
If the content of this variable is assigned to a y-variable of 3-bits (which can only store the range [-4,3]),
only the least significant bits are stored and the 
resulting variable will store the number
$$
       y = 101,
$$
which represents the value -3 in a two's complement representation with 3 bits. The original value of x has value 5.  
Since the overflow of an integer variable never shows as an error, 
it is very important to avoid overflows of integer variables not to have mysterious 
programming problems very difficult to be identified. 


\begin{IN}
Overflow of integer variables are not usually identified as explicit errors and they 
can originate strange behaviors very difficult to discover in a programming code. 
\end{IN}




    \newpage 
    \section{Understanding integer overflows}  \label{sec:UndersIntOver}

Now, the examples at the beginning of this chapter are easily interpreted. In the first case, the 
values $+128, +129, \dots$ stored in \texttt{k} (2-bytes memory) size are assigned to the variable 
\texttt{i} with 1-byte memory size.
Then, the binary forms of $+128_{2-bytes}$ or $+129_{2-bytes}$:

\begin{equation*}
    +128_{2-bytes} = \texttt{00000000 10000000} ,\:  +129_{2-bytes} = \texttt{00000000 10000001}
\end{equation*}

are stored keeping the 8 least significant bits like:

\begin{equation*}
    \texttt{10000000} ,\:  \texttt{10000001}
\end{equation*}

which are the binary two's complement form of $-128_{1-byte}$ and $-127_{1-byte}$ respectively. 

In the second example the variable \texttt{f} does not overflow in one or two units but in a much 
higher quantity. 
The real result of $13!$ is $6227020800$ whose binary representation with enough memory size to store 
it (40 bits for example) is: 

\begin{equation*}
    6227020800_{40-bits} = \texttt{00000001 01110011 00101000 11001100 00000000} 
\end{equation*}

Once this variable is stored with the 32 least significant bits the result is:

\begin{equation*}
    \texttt{01110011 00101000 11001100 00000000} 
\end{equation*}

which is the binary two's complement representation of $+1932053504_{4-bytes}$, the result printed by the computer. 
From that value, all the following results of the factorial are wrong because of the overflow of the 4-bytes variable. 




In the following example, 
an integer variable of 1-byte size is overflowed with the value 130 and with 
-130. The resulting values are compared with an integer variable of 2-bytes size.  
The internal representation  in binary form according to two's complement is also shown. 
The following subroutine determines the internal binary form of an integer of any kind. 
Depending on the \texttt{kind\_type} of the integer variable, 
the maximum and minimum range is calculated by means of the subroutine \texttt{integer\_parameters}.
Later, the integer variable \texttt{xr} is written in binary form by means of a \texttt{write} sentence. 
And finally, the integer variable \texttt{xr} is reconstructed by means of equation  (\ref{reconstruction})
to show on screen and to check the binary form representation. 

\vspace{0.5cm}
\listings{\home/Integer_representation.f90}{subroutine Integer_representation_twos_complement}
{end subroutine}{Binary form representation. Integer_representation.f90}


To understand the binary internal representation of an integer overflow, 
the following lines gives the results: 
\begin{verbatim} 
call Integer_representation_twos_complement("kind=1",  130)
call Integer_representation_twos_complement("kind=2",  130)
call Integer_representation_twos_complement("kind=1", -130)
call Integer_representation_twos_complement("kind=2", -130)
\end{verbatim} 

\begin{verbatim} 
----------------------------------------------------------------
 Internal representation of x=130  with x integer  kind=1
    Integer  kind=8            x =                  130
    Same integer kind=1        x =                 -126
    Range integer kind=1         =                 -128      127
    Internal representation of x = 10000010
    Reconstruction sum b_i 2**k  =                 -126
----------------------------------------------------------------
----------------------------------------------------------------
 Internal representation of x=130  with x integer  kind=2
    Integer  kind=8            x =                  130
    Same integer kind=2        x =                  130
    Range integer kind=2         =               -32768    32767
    Internal representation of x = 0000000010000010
    Reconstruction sum b_i 2**k  =                  130
---------------------------------------------------------------
---------------------------------------------------------------
 Internal representation of x=-130  with x integer  kind=1
    Integer  kind=8            x =                 -130
    Same integer kind=1        x =                  126
    Range integer kind=1         =                 -128      127
    Internal representation of x = 01111110
    Reconstruction sum b_i 2**k  =                  126
----------------------------------------------------------------
----------------------------------------------------------------
 Internal representation of x=-130  with x integer kind=2
    Integer  kind=8            x =                 -130
    Same integer kind=2        x =                 -130
    Range integer kind=2         =               -3276     32767
    Internal representation of x = 1111111101111110
    Reconstruction sum b_i 2**k  =                 -130
----------------------------------------------------------------
\end{verbatim} 
It's shown that the value \texttt{130} overflows an integer variable of \texttt.
When this happens, no alert is shown by the compiler and non desirable results could destroy 
the performance of our programming code.    



    \newpage
    \section{Overflow of constants}
        
The issues and limits treated above also appear when, instead of integer variables, the program works 
with integer constants. 
The assignment of a constant which is out range of a \texttt integer variable 
will give rise to a compilation error.

In the following example an integer variable of size 8 bytes is assigned with a huge number: 
\begin{verbatim}
    integer (kind = 8) :: P
   
    !Example variable assignation out-of-range 8 bytes
    P = 9223372036854775808
\end{verbatim}
When trying to compile the above program, the compiler returns the following message: 
\begin{verbatim}
    Compilation Aborted (code 1)
    error #6901: The decimal constant was too large when 
        converting to an integer, and overflow occurred.
        [9223372036854775808]
\end{verbatim}
The compiler tries to store the number \texttt{9223372036854775808}, which is equal to \texttt{huge(P)}+1,
in an internal register of 8-bytes with two's complement representation. Since the constant is greater than 
the maximum allowed value of the maximum memory size available in the computer the compiler gives the above error. 


\newpage
    \section{Overflow by incorrect assignment}

If valid huge constants are assigned to lower size integer variables, only compiler warnings could alert 
of possible malfunctions.  
  
In the following example, an integer variable \texttt{Q} of 4-bytes is considered. 
The value \texttt{2147483648}, which is equal to \texttt{huge(Q)} +1, is assigned 
to \texttt{Q}. 
\vspace{0.5cm}
\listings{\home/Integer_representation.f90}{subroutine Assignment_overflow}
{end subroutine}{Assignment overflow. Integer_representation.f90}
When compiling the above program, no errors are obtained but the following warning appears: 
\begin{verbatim}
    warning #6384: The INTEGER(KIND=4) value is out-of-range.	
\end{verbatim}
Hence, the program  could be executed without taking care of warning messages giving the following result: 

\begin{verbatim}

   Intention: assign Q <- 2147483648
   Result:  Q = -2147483648 huge(Q) =   2147483647
\end{verbatim}
If warnings are not taken into account, non desirable results could be  obtained.  
In this case, variable \texttt{Q} stores the value  \texttt{-2147483648} and the 
programmer by an improper assignment tried to store the value \texttt{2147483648} which is greater 
the maximum value that \texttt{Q} can store.




    \section{Declaring kind}

A good practice is to write codes that could be used with different integer sizes: \texttt{kind=1, 2, 4, 8}. 
This can be done by not specifying the kind type of any integer variable. 
In this case, the compiler assumes default integer kind for all integer variables. 
Generally, this default can be easily changed  with a proper compilation option. 


The opposite approach is to declare specifically the kind type of any integer variable by means of 
sentences such as: 
 \begin{verbatim}
     integer (kind=1) :: i1
     integer (kind=2) :: i2
     integer (kind=4) :: i4
     integer (kind=8) :: i8
 \end{verbatim}
When declaring constants, the situation is different. A specific kind type is declared  by underscoring the 
constant with its kind type: 
 \begin{verbatim}
    integer (kind=8) :: i 
    
     i = 123_1 
     i = 123_2
     i = 123_4 
     i = 123_8
 \end{verbatim}
In the above example, the constant \texttt{123} is stored in a register of 1, 2, 4, or 8 bytes. Later, it is 
assigned to an integer variable of 8 bytes. 
When no kind type specifier is selected for integer constants (\texttt{i = 123}), 
then the compiler reserves an internal appropriate register of variable size
depending on the value of the constant. 

\begin{IN}
    A program can be independent of declared precisions so the following compiler option rules the behaviour of all integers:
    
    Default Integer KIND: $\texttt{/integer-size:{16\mid32\mid64}}$  
\end{IN}


%
%The compiler defines the memory size for each \textbf{constant} based on some rules:
%
%\begin{itemize}
%    \item If a kind parameter is specified, the integer has the kind specified. 
%    \item If a kind parameter is not specified, integer constants are interpreted as follows:
%        \subitem If the integer constant is within the default integer kind range, the kind is default integer (compiler options).
%        \subitem If the integer constant is outside the default integer kind range, the kind of the integer constant is the smallest integer 
%kind that holds the constant. 
%\end{itemize}
%
%According to \citet{IntelIntegers} the default integer size is affected by the compiler option \texttt{integer-size}, the \texttt{INTEGER} 
%compiler directive, and the \texttt{OPTIONS} statement. These configurations allow the user to change easily the execution of the code so, 
%for example, the algorithm can be checked for simple and double precision independently. 
%
%The memory size reserved for \textbf{variables} is slightly different:
%
%The first rule is also applied, the kind can be defined in the declaration. However, if it is not specified, the compiler is not permissive 
%and does not changes it automatically if finds that an out-of-range value is stored in a variable sized with default integer value.
% 
%    The compiler do not decide automatically, just conforms to what is defined by the programmer. Notice that every code file admits some 
%    compiler options where the default integer kind for variables and constants can be defined 
%    (option \texttt{integer-size}).
%    
    
\begin{IN}
    \begin{itemize} 
    \item  \textbf{Constants.}
    The compiler reserves memory size depending on the magnitude of the constant. 
    \item  \textbf{Variables.}
    The memory size of any variable is specified by its kind type. 
    If no kind type is specifically declared, the default integer kind is considered by the compiler. 
   
    \end{itemize} 
\end{IN}

%
%fixes automatically the size reserved for all constants and variables based 
%on the compiler options (which can be easily changed). 
%
%This allows you to run the program with different configurations depending on the needs of the moment. 
%
%However, codes are written with different purposes, choose the right option for each situation. 
%If, despite this, you still want to declare the kind for a specific variable just 
%include the value in the declaration by writing:
%
%\texttt{integer(kind = k) ::} where k is the bytes reserved: 1, 2, 4 or 8.
%
%%\texttt{integer(k) ::} or \texttt{integer*k ::}
%




%%OLD MATERIAL
%\newpage
%\listings{\home/Round_off.f90}{For CONSTANTS:}
%{--END OF EXAMPLE--}{Second example of Integer_representation - Round_off.f90} 
%
%The previous example shows a couple of Warnings that must be considered when managing with integers. The first one makes reference to the assignation of the value $130$ to a 1 byte integer constant, The second warning says the same for the variable Q, which is also out-of-range since it is declared as \texttt{kind = 4}. Notice that none of them interrupts the compilation of the code so if the warnings are not considered, the program will be executed with a different behaviour as expected. 
%
%\begin{verbatim}
%    warning #6047: The BYTE / LOGICAL(KIND=1) / INTEGER(KIND=1)
%    value is out-of-range.   [130]
%    
%    warning #6384: The INTEGER(KIND=4) value is out-of-range.
%\end{verbatim}


%\begin{itemize}
%    \item For a \textbf{constant} write the bytes reserved after the value preceded by an underscore (\texttt{n[\_k]}). 
%    \item For a \textbf{variable}, just include the kind in the declaration by writing \texttt{integer(k) ::}, \texttt{integer(kind = k) ::} or \texttt{integer*k ::} (where k is the bytes reserved; 1, 2, 4 or 8).
%\end{itemize}

%\newpage
%%Casos especiales
%Finally, take a look at these two especial cases treated by the compiler. 
%\begin{enumerate}
%    \item In the assignment of a lesser memory size constant to a higher size variable the variable imposes its size and the constant will be transformed to the higher value. 
%    \item If a corrupted constant (for example, \texttt{130\_1}) is assigned to a proper sized variable (for example \texttt{integer*2 :: R}), the compiler warns first about the out-of-range constant but makes the assignment properly assuming that the value expected for \texttt{R} is \texttt{130} and not the corrupted value.
%\end{enumerate}
%
%\listings{\home/Round_off.f90}{!1. R is 8}
%{--END OF EXAMPLE--}{Third example of Integer_representation - Round_off.f90}



    \section{Two's complement converter} 

A binary-to-decimal and decimal-to-binary converters are included in the codes that accompany this book.
A third function is also included, it checks if an integer value is not properly stored 
with an specific number of bits because of an overflow error. 
The converters will return the two's complement conversion for the introduced value, whether is an
integer value or a string characters of 0's and 1's. The specifications of each function are presented now.


        \subsection{\texttt{TwosCompl\_converter\_to\_decimal} function}

Returns the decimal integer value for the string of binary characters introduced according to the two's complement encoding.

Notice that the two's complement encoding is always referred to an specific number of bits which means that it make 
sense in a pre-defined memory size. Due to that, the conversion is performed according to the total length of bits 
introduced in the string, if your binary expression has leading zeros do not forget to include them. In addition, 
blank spaces are also treated as zeros so the string \texttt{' 01101'} is the same as \texttt{'001101'}.

If more than 64 bits or wrong characters are introduced the result will be automatically \texttt{0}. 
The table \ref{tab:specs1} shows the specifications of the input arguments of the converter.

    \texttt{TwosCompl\_converter\_to\_decimal( bits )}

\begin{table}[h]
    \centering
    \begin{tabular}{| c | c | c |}
        
        \hline
        Name       & Type                        & Limits  \\ \hline
        
        \begin{tabular}{@{}c@{}} \texttt{bits}  \\ (input)  \end{tabular}   & \texttt{character}    &  \begin{tabular}{@{}c@{}} Up to 64 characters with 0's, 1's  \\ or blank spaces (treated as 0's) \end{tabular}   \\ \hline
        
        Result   & \texttt{integer(kind=8)}   &  $[-9 223 372 036 854 775 808, 9 223 372 036 854 775 807]$  \\ \hline
       
        
    \end{tabular}                                                       
    \caption{Specifications of the \texttt{TwosCompl\_converter\_to\_decimal} function.}
    \label{tab:specs1}
\end{table}

Example:
\vspace{-0.5cm}
\begin{verbatim} 
write(*,*) "Two's complement conversion of binary: "
write(*,*) "010101010110111001110  =  "
write(*,*) TwosCompl_converter_to_decimal( '010101010110111001110' )
\end{verbatim} 
\vspace{-0.6cm}
\begin{verbatim} 
Two's complement conversion of binary:
010101010110111001110  =
                          699854
\end{verbatim} 

%Example 2:
%
%\begin{verbatim} 
%write(*,*) "Two's complement conversion of binary: "
%write(*,*) "110010110  =  "
%write(*,*) TwosCompl_converter_to_decimal( '110010110' )
%\end{verbatim}
%
%\begin{verbatim} 
%Two's complement conversion of binary:
%110010110  =
%                         -106
%\end{verbatim}
 

        \subsection{\texttt{TwosCompl\_converter\_to\_binary} function}

Returns the minimum length binary string that can store the decimal integer introduced according to the two's complement encoding.

Notice that the two's complement encoding is always referred to an specific number of bits which means that it make 
sense in a pre-defined memory size. This result is printed with the minimum bits that can store the number. 

Out-of-range \texttt integer values will not be compiled (see table \ref{tab:specs2}). In addition, take into account that the algorithm used for this converter is the 
one described in the section \ref{sec:TwosComp}. This algorithm will not return a proper value for the case \texttt{x = - huge(P) - 1} where \texttt{P} is a \texttt
variable. While that number can be perfectly stored in the computer, its absolute value can not:

\begin{equation*}
    \left|x\right| = \left|- huge(P) - 1\right|= huge(P) + 1
\end{equation*}

    \texttt{TwosCompl\_converter\_to\_binary( x )}

\begin{table}[h]
    \centering
    \begin{tabular}{| c | c | c |}
        
        \hline
        Name       & Type                        & Limits  \\ \hline
        
        \begin{tabular}{@{}c@{}} \texttt{x}  \\ (input)  \end{tabular}   & \begin{tabular}{@{}c@{}} \texttt{integer}  \\ (all kinds)  \end{tabular}  &  $[-9 223 372 036 854 775 807, 9 223 372 036 854 775 807]$  \\ \hline
        
        Result   & \texttt{character}  &  Up to 64 binary characters  \\ \hline
        
        
    \end{tabular}                                                       
    \caption{Specifications of the \texttt{TwosCompl\_converter\_to\_binary} function.}
    \label{tab:specs2}
\end{table}

Example 1:

\begin{verbatim} 
write(*,*) "Two's complement conversion of integer: ", -89544563
write(*,*) TwosCompl_converter_to_binary( -89544563 )
\end{verbatim} 

\begin{verbatim} 
Two's complement conversion of integer:    -89544563
1010101010011010100010001101
\end{verbatim} 

Example 2: This example shows that the number \texttt{1} must be represented with at least two bits. 
The following section explains this result. 

\begin{verbatim} 
write(*,*) "Two's complement conversion of integer: ", 1
write(*,*) TwosCompl_converter_to_binary( 1 )
\end{verbatim} 

\begin{verbatim} 
Two's complement conversion of integer:            1
01
\end{verbatim}



        \subsection{\texttt{Check\_overflowed\_int} function}

For an integer introduced it returns the actual decimal integer stored in the computer in a specific memory 
size according to the two's complement encoding. If the number is stored in more (or equal) binary digits than
the minimum length needed, the result is the same as the input.

Notice that an out-of-range \texttt integer value will not be compiled (see table \ref{tab:specs3}). 

    \texttt{Check\_overflowed\_int( int, n\_bits )}

\begin{table}[h]
    \centering
    \begin{tabular}{| c | c | c |}
        
        \hline
        Name       & Type                        & Limits  \\ \hline
        
        \begin{tabular}{@{}c@{}} \texttt{int}  \\ (input)  \end{tabular}   & \begin{tabular}{@{}c@{}} \texttt{integer}  \\ (all kinds)  \end{tabular}  &  $[-9 223 372 036 854 775 808, 9 223 372 036 854 775 807]$  \\ \hline
        
        \begin{tabular}{@{}c@{}} \texttt{n\_bits}  \\ (input)  \end{tabular}   & \begin{tabular}{@{}c@{}} \texttt{integer}  \\ (default kind)  \end{tabular}  &  $[1, 64]$  \\ \hline
        
        Result   & \texttt{integer(kind=8)}  &  $[-9 223 372 036 854 775 808, 9 223 372 036 854 775 807]$  \\ \hline
        
    \end{tabular}                                                       
    \caption{Specifications of the \texttt{Check\_overflowed\_int} function.}
    \label{tab:specs3}
\end{table}

A non-intuitive result is obtained from the experiment of representing the decimal integer \texttt{1} with just one bit. 
In a pure binary conversion (unsigned integer for example) it is clear that the number \texttt{1} is represented through 
the binary digit \texttt{1}. However, it can be checked that the two's complement conversion of the number \texttt{1} can only be 
performed with 2 or more binary digits (\texttt{01}), being the result of the following line \texttt{-1}.

\begin{verbatim}
    Check_overflowed_int( 1, 1 )
\end{verbatim}


Example:

\begin{verbatim} 
write(*,*) "Check overflow integer of 13! = ", 6227020800
write(*,*) "with 4-bytes memory size = ", 32, "bits"
write(*,*) "The actual value stored is: "
write(*,*) Check_overflowed_int( 6227020800, 32 )
\end{verbatim} 

\begin{verbatim} 
Check overflow integer of 13! =             6227020800
with 4-bytes memory size =           32 bits
The actual value stored is:
1932053504
\end{verbatim} 




%----------------------------OLD MATERIAL-----------------------
%
%A binary-to-decimal and decimal-to-binary converter is included in the codes that accompany this book.
%In order to perform a conversion specify the value to convert, whether is a binary or a decimal value, 
%and the number of memory size bits reserved to store the result. The two's complement conversion for that number of bits 
%is shown in the screen together with some useful information about the conversion.
%
%\texttt{Twos\_complement\_converter} subroutine:
%
%Displays a summarize of the conversion with:
%
%\begin{itemize}
%    \item Value to convert
%    \item Number of bits reserved to store the result
%    \item Decimal range available with that number of bits
%    \item Two's complement result of the conversion with that number of bits
%    \item Two's complement result of the conversion with 8-bytes (64 bits) memory storage 
%\end{itemize} 
%
%Notice that the result with the specified number of bits and the maximum allowed memory size (8-bytes) is the same 
%when no overflow has occurred. If an out-of-range number for the bits reserved is converted both results will differ. When this occurs, if the 
%conversion is performed from decimal to binary, the converter automatically shows the actual decimal value that is being stored due to the overflowing 
%value. The table \ref{tab:converterspecs} shows the specifications of the input arguments of the converter. 
%
%\texttt{call Twos\_complement\_converter( value, n\_bits )}
%
%\begin{table}[h]
%    \centering
%    \begin{tabular}{| c | c | c |}
%        
%        \hline
%        Name       & type                        & limits  \\ \hline
%        
%        \begin{tabular}{@{}c@{}} \texttt{value}  \\ (bin \rightarrow dec)  \end{tabular}   & \texttt{character}    &  \begin{tabular}{@{}c@{}} Up to 64 characters with 0's, 1's  \\ or blank spaces (treated as 0's) \end{tabular}   \\ \hline
%        
%        \begin{tabular}{@{}c@{}} \texttt{value}  \\ (dec \rightarrow bin)  \end{tabular}   & \texttt{integer}   &  $[-9 223 372 036 854 775 808, 9 223 372 036 854 775 807]$  \\ \hline
%        
%        \texttt{n\_bits}    & \texttt{integer}                     & $[1, 64]$                                                  \\ \hline
%        
%        \multicolumn{3}{l}{ \begin{tabular}{@{}c@{}} Notice that \texttt{n\_bits} does not need to coincide with the number of bits  \\  in the binary value to convert, it can be greater, equal or less. \end{tabular}    }   \\ \hline
%        
%    \end{tabular}                                                       
%    \caption{Input arguments of the \texttt{Twos\_complement\_converter} subroutine.}
%    \label{tab:converterspecs}
%\end{table}
%
%
%
%\newpage
%Examples of binary\rightarrow decimal use:
%
%\begin{verbatim} 
%call Twos_complement_converter( '0110100101101001', 30 )
%\end{verbatim} 
%
%\begin{verbatim} 
%--------------------------------------------------
%Conversion from two's complement binary to decimal
%                Value to convert =  0110100101101001
%                     Memory size =  30 bits
%Integer range = [          -536870912           536870911]
%     Two's complement conversion =                 26985
%Two's compl. 8-bytes conversion. =                 26985
%--------------------------------------------------
%\end{verbatim} 
%
%In this example the binary string is converted with enough memory size to hold the integer value, notice that 30 bits 
%is not a supported integer in most programming languages since this data type is always stored in 1, 2, 4 or 8 bytes of memory. 
%
%\begin{verbatim} 
%call Twos_complement_converter(                 &
%    '0000000101110011001010001100110000000000', &
%                                           32 )
%\end{verbatim} 
%
%\begin{verbatim} 
%--------------------------------------------------
%Conversion from two's complement binary to decimal
%Value to convert =  0000000101110011001010001100110000000000
%     Memory size =  32 bits
%Integer range = [         -2147483648          2147483647]
%     Two's complement conversion =            1932053504
%Two's compl. 8-bytes conversion. =            6227020800
%--------------------------------------------------
%\end{verbatim}
%
%In this example 4 bytes is not enough memory for this value, an overflow has occurred. The programmer should consider to use a 
%\texttt variable to hold it if this value is used in the program. 
%
%
%
%
%
%\newpage
%Examples of decimal\rightarrow binary use:
%
%\begin{verbatim} 
%call Twos_complement_converter( 735, 16 )
%\end{verbatim} 
%
%\begin{verbatim} 
%--------------------------------------------------
%Conversion from decimal to two's complement binary
%Value to convert =                   735
%Memory size =  16 bits
%Integer range = [              -32768               32767]
%Two's complement conversion =  0000001011011111
%Two's compl. 8-bytes conversion. =  0000000000000000000000
%                000000000000000000000000000000001011011111
%--------------------------------------------------
%\end{verbatim} 
%
%
%
%\begin{verbatim} 
%call Twos_complement_converter( 35000, 16 )
%\end{verbatim} 
%
%\begin{verbatim} 
%--------------------------------------------------
%Conversion from decimal to two's complement binary
%Value to convert =                 35000
%Memory size =  16 bits
%Integer range = [              -32768               32767]
%Two's complement conversion =  1000100010111000
%Two's compl. 8-bytes conversion. =  0000000000000000000000
%                000000000000000000000000001000100010111000
%--------------------------------------------------
%Overflow, not enough bits to store this number.
%The actual value stored in the computer is:
%--------------------------------------------------
%Conversion from two's complement binary to decimal
%                Value to convert =  1000100010111000
%                     Memory size =  16 bits
%Integer range = [              -32768               32767]
%     Two's complement conversion =                -30536
%Two's compl. 8-bytes conversion. =                 35000
%--------------------------------------------------
%\end{verbatim}
%
%In this last example notice how the number 35000 is an out-of-range value for a 2-bytes integer. 
%The converter warns that an overflow has occurred and the actual value stored in the computer is 
%$-30536$, which is the two's complement conversion of the 16 least significant bits involved in the 
%expected value. 







       %--------------------------------------------------------------------------------------------------------------------------------------------------------------------
\renewcommand\home{./Fortran_project/sources/IEEE}

\chapter{Reals representation and operations} \label{chap:reals}
     
\section*{Overview}

Real numbers can be represented by 
an integer part and a decimal parts separated by a character which is usually a point. 
The integer and the decimal part have, in general, infinite length.  
However, since the internal representation of a real number 
in a computer  is stored  on a finite amount of memory, 
only some real numbers can be represented exactly. 
In other words, working with a finite precision machine,  real numbers are approximate.

    
%Let's talk now about reals, the base of any numerical calculation that can be developed in a computer. 
%If the infinite integers that exist can not be represented in the computer and a computer 
%scientist is limited to the values in a range, 
%in the case of reals it happens the same. 
%However, between any two real numbers, 
%there are also infinite reals so a finite number of bits is not able to represent that set either. 


This idea is extended to real operations and allows to understand that most calculations 
are approximate and the resulting errors are called "round-off" errors. 
The origin and the consequences of these errors will be discussed in this chapter. 
The following topics are covered in this chapter: 
%
%to perform will result in values 
%that are not represented in the standard implemented in the machine. 
%Thus, those value are rounded to the nearest values that has representation in the computer. 
%Either in assigning a constant to a variable or an operation, 
%this is the origin of the rounding error, which is treated in this chapter. 




  
%  \vspace{0.5cm}
%   \renewcommand{\home}{./Fortran_project/sources} 
%    \listings{\home/main_advanced.f90}{select}{Games}{main_advanced.f90}
  
  




\begin{enumerate} 
    \setlength\itemsep{-0.1cm}
    \item Example of round-off errors. 
    \item Fixed-point and Floating-point representation.
    \item Representation in IEEE 754.
    \item Distance between floating point real numbers.
    \item Reconstruction from its internal binary representation.
    \item Writing floating point expressions.
    \item Condition number and stability.
    \item Catastrophic cancellation.
    \item Summation example.
    \item IEEE exception examples.
    
\end{enumerate} 


  \newpage    
\section{Example of round-off errors} 
    
To introduce unexpected behavior when working with real variables and round-off errors, 
two examples are presented.   
In the first example, a real variable \texttt{S} stored in 4 bytes of memory is used to sum \texttt{N=100000}
times the value \texttt{dX = 0.3}. The expected result is \texttt{S = 0.3 \times 100000 = 30000 }. 
However, the execution gives \texttt{S = 3027.90}. Later, in the same subroutine, la magnitude   
\texttt{dX = 0.3} is substracted  \texttt{N} times. Again, the expected result is  \texttt{S =0 }
but the computer result is \texttt{S = 26.1582}

\vspace{0.5cm}
 \renewcommand{\home}{./Fortran/sources/IEEE} 
\listings{\home/Round_off.f90}
{errors_in_operations}{end subroutine}
{Round_off.f90} 

\newpage
In the second example, an infinite loop is written to check the smallest value \texttt{eps} 
that being added to the unity gives a result different to the unity.  
%This snippet allows to determine  the 
%the smallest number E of the same kind as x such that 1 + E > 1. 
Fortran implements this value by means of the intrinsic function \texttt{epsilon(x)}.

\vspace{0.5cm}  
  \renewcommand{\home}{./Fortran/sources/IEEE}  
\listings{\home/Round_off.f90}
{loss_of_precision}{end subroutine}
{Round_off.f90} 


The above examples illustrate two main problems when working with real variables and real constants: 
\begin{enumerate} 
\setlength\itemsep{0.1cm}
\item Real constant numbers do not have an exact representation in the computer. 
\item Operations with reals give rise to round-off errors that could be important. 
\end{enumerate}
The first problem is illustrated with the following example: 
\begin{verbatim}
    write(*,'(a,f20.15)') "Constant 1.1 with single precision = ", 1.1    
\end{verbatim}
which gives the result:  
\begin{verbatim}
    Constant 1.1 with single precision =  1.100000023841858   
\end{verbatim}
In this case, the default real kind option of the compiler is single precision. 
There are two options to improve the  precision for the constant \texttt{1.1}
\begin{enumerate} 
\setlength\itemsep{-0.1cm}
\item Specify the real kind of the constant by writing \texttt{1.1d0} 
\item Modify the deflaut real kind of the compiler. 
\end{enumerate}
When executing the following code  with default  double precision for real variables and constants:
\begin{verbatim}
     write(*,'(a,f20.15)') "Constant 1.1 with double precision = ", 1.1    
\end{verbatim}
the result gives: 
\begin{verbatim}
    Constant 1.1 with double precision =  1.100000000000000   
\end{verbatim}

The second problem related to round--off errors of real operations is even more complicated
but the same criteria applies to increase the accuracy of operations.  To assure seven 
significant digits in a real operation, single precison is enough. To assure fifteen significant digits,
double precision is needed. In the same manner, this double precision can be attained by specifying the 
real kind of variables or by configuring the default real kind by means of a compiler option.  
The explanation of seven or fifteen significant digits associated to  single or double precision 
is epxlained in the next sections. 

%
%
% the first thing a programmer notice when working with reals. 
%Consider that, independently of the precision 
%used for the number, 
%the \texttt{write} statement shows the number with 20 digits (15 of them reserved for the decimal part). 
%For this example, 
%the compiler is configured to use simple precision for the constants (unless double precision is imposed):
%
%\begin{verbatim}
%write(*,'(a,f20.15)') "1. = ", 1.    
%write(*,'(a,f20.15)') "1d0 = ", 1d0
%
%write(*,'(a,f20.15)') "1.1 = ", 1.1    
%write(*,'(a,f20.15)') "1.1d0 = ", 1.1d0
%
%write(*,'(a,f20.15)') "300.2 = ", 300.2
%write(*,'(a,f20.15)') "300.2d0 = ", 300.2d0
%
%write(*,'(a,f20.15)') "1.3 = ", 1.3
%write(*,'(a,f20.15)') "1.3d0 = ", 1.3d0
%\end{verbatim}
%
%The result is the following:
%
%\begin{verbatim}
%1. =    1.000000000000000
%1d0 =    1.000000000000000
%
%1.1 =    1.100000023841858
%1.1d0 =    1.100000000000000
%
%300.2 =  300.200012207031250
%300.2d0 =  300.199999999999989
%
%1.3 =    1.299999952316284
%1.3d0 =    1.300000000000000
%\end{verbatim}
%
%While the real number 1 is exactly represented in simple and double precision (at least with 15 decimal digits), the reals 1.1, 300.2 and 
%1.3 
%are not exactly represented. This issue is ``solved'' for 1.1 in double precision for example but it is not fixed for the real 300.2. The 
%user supposes that is working with a specific real, however, the computer is performing calculations with a slightly different number most 
%of 
%the times. 

%Take a look at another example:
%
%\begin{verbatim}
%    c = 0
%    do i = 1, 100
%    c = c + 0.1
%    write(*,'(a,f20.15)') "c =", c
%    enddo
%    
%    write(*,'(a,f20.15)') "0.1 = ", 0.1
%    write(*,'(a,f20.15)') "prod = ", 0.1 * 100
%\end{verbatim}
%
%which results in:
%
%\begin{verbatim}
%               .
%               .
%               .
%    c =   9.800001144409180
%    c =   9.900001525878906
%    c =  10.000001907348633
%    
%    0.1 =    0.100000001490116
%    prod =   10.000000000000000
%\end{verbatim}
%
%Notice how the successive add operations accumulate an error due to the 





\section{Fixed-point and Floating-point representation}

There are two main ways to represent real numbers in computers: 
\begin{itemize}
\item Fixed-point representation. 
The real number is represented by its integer part and its decimal part (e.g. \texttt{55.88}). 
\item Floating-point representation. 
 The real number is represented by its mantissa and its exponent  (e.g. \texttt{0.5588e02}).
\end{itemize} 

Fixed-point representation fixes the position of the binary point that separates 
the integer part from the decimal part. 
Then, the bits reserved for both parts are prefixed. 
This method has the advantage that processing the numbers is simpler
but has strong limitations in the available range  to work with.

Nowadays,  computers use  the floating point representation through the standard IEEE 754.
The numbers are stored similarly to the way standard scientific notation is written. 
Hence, a number of bits store the mantissa of the number and another part of bits store the exponent 
of that number. 
Notice that the exponent is telling how many positions the 
Since the exponent allows to position the decimal point 
to the left or to the right in the mantissa,  huge numbers and tiny numbers can be represented.
As an inconvenient, for the same number of bytes, 
this representation is slower to process and has lower resolution than fixed-point representation. 


As an example, 
consider a fixed point 1-byte precision number with 5 bits the integer part and 3 bits
for the decimal part as it is shown in Figure \ref{fig:FixedFloat}. 

\begin{figure}[h]
    \centering
    \includegraphics[width= 0.8\textwidth]{./doc/Figures/FixedFloat2.png}
    \caption{Representation of a couple of numbers with fixed point and floating point formats using only 1 byte (8 bits).}
    \label{fig:FixedFloat}
\end{figure}
Giving a binary fixed point representation with 8 bits ($b_7, \ldots, b_0$), the decimal number is obtained by: 
$$
   x = b_7 \ 2^4 + b_6 \ 2^3 + b_5 \ 2^2  + b_4 \ 2^1 + b_3 \ 2^0 + b_2 \ 2^{-1}  + b_1 \ 2^{-2} + b_0 \ 2^{-3}
$$
Hence, the number $20.25$ is $\texttt{10100.010}$ in binary fixed point 
representation. 
Notice that the smallest number that can be represented would be 
$\texttt{00000.001}$ which is only $0.125$ 
and also notice that the biggest number available with this method is $\texttt{11111.111}$
which is $31.875$. 
This representation allows numbers from $0.125 = 2^{-3}$ to $31.875$  with jumps of $0.125$. 
If 2, 4, 8 or 16 bytes are used to represent this number the range increases 
and the distance among numbers decreases but the range is still very limited. 


In order to compare with floating-point representation,
let's consider a mini-float of 8 bits with the same memory size that 
the above fixed point representation. 
Consider a mini-float with
4 bits for the exponent(1 for its sign and 3 for its value) and 4  bits for the mantissa.
Giving a binary floating point representation with 8 bits ($b_7, \ldots, b_0$), 
the decimal number is obtained by: 
$$
   x = m \ 2^e, \quad m =  b_7 \ 2^{-1}  + b_6 \ 2^{-2} + b_5 \ 2^{-3} + b_4 \ 2^{-3}, \quad 
    e =  \pm (  b_2 \ 2^{2} + b_1 \ 2^{1} + b_0 \ 2^{0} ) 
$$
The maximum value with this representation is: 
$$
   \texttt{1111 1111} = 15 \times 2^{7}
$$   
and the smallest number is: 
$$
   \texttt{0001 0111} =  2^{-7}.
$$ 
The distance between the maximum number and its closest  number is:  $  2^{7}$
and the distance between the minimum number and its closest number is:  
$ 2^{-7}.$

%These are not used in numerical calculations where simple precision (4 bytes) 
%is the smallest precision used. 
%However, it needs from the same amount of memory space 
%as the previous example so it can be a good comparison to understand the concept. 
%For special purposes it is used the $(1, 4, 3, 7)$ minifloat, 


% and an exponent bias of +7 
%(these concepts are broaden in this section). 
%This representation allows a range between $r\in (-480 = 1 1111 111_2, 480 = \texttt{0 1111 111}_2)$ 
%and 

% $0.0078125_{10} =_2$. 
%We are not discussing now if the number $0 0000 000_2$ is reserved for the $0$ 
%or the number $\texttt{0 1111 111}$ or $\texttt{1 1111 111}$ are reserved for $\pm\infty$ 
%since this is just an example of the capabilities of floating-point representation. 


%Do not worry if the notation is not clear right now, it is explained later for the 4-bytes precision, 
%which is similar. 



To conclude, for the same number of bits, fixed-point numbers are equal-spaced  along 
the whole range but with a smaller range ($31.875$ vs $ 15 \times 2^{7}$).
The distance among real numbers in the fixed-point representation is constant and equal to $0.125$. 
However, in the floating-point representation, the distance among real numbers 
changes from $ 2^7$ to $ 2^{-7}$ depending on the exponent of the real number. 
It is also important to notice that the total amount of real numbers in fixed point or floating point 
representation is the same but their distribution and range are different.



%for the same exponent, all the mantissas grow with the same resolution and adding 
%one to the exponent reduces highly the resolution. 

%Quickly in the floating-point 
%of 1-byte the resolution is poor, 
%for an exponent of $1110_2 = 14_10\rightarrow  \textrm{exponent} = 7$ the resolution 
%is 16 units while the resolution for an exponent of $0001_2 = 1_10\rightarrow  \textrm{exponent} = -6$ 
%is $0.001953...$. Why this happens is simple, we must not forget that the number of bits is
% the same for this examples, in both cases we can not store more values than $2^8 = 256$, 
% if we want more range, is at the cost of less resolution. 

\begin{IN}
%     Fixed-point format actually could represent larger numbers or smaller numbers,
%     but it has a static range, which means that the choose is done when designing the code, 
%     later you can only use that range, 
%     either big or small numbers. 
%     Luckily, floating-point format has dynamic range, 
%     which means that once designed the format it can handle 
%     at the same time large and tiny numbers just varying the exponent. 
\begin{enumerate}
\item Fixed-point format has constant distance among all representable real numbers and a small range. 
\item Floating-point format has variable distance among all representable real numbers with a huge range.  
\end{enumerate}
\end{IN}

%\begin{IN}
%    Do not forget that, exactly than in decimal system, the positions on the left of the decimal point are numbers bigger than 1 and the positions on the right are smaller than 1. While in decimal system each position multiplies by ten the value on his right, in binary system the value is doubled going to the left and divided by two going to the right. 
%\end{IN}


In the following section, 
the standard IEEE 754 which is used in all computers is 
explained in detail to understand how real numbers are stored in 
floating point representation.




\section{Floating-point representation in IEEE 754}





%--------------------------------------------------------------------------------------------------------------------------------------
\newpage 
\FloatBarrier
    \section{Distance between floating point real numbers} \label{sec:roundoff}

It has been introduced in the previous section that there are two reasons why the number you may want to represent in the computer can not be 
exactly represented and has to be rounded (to an IEEE floating-point value). First and the less common, the number is out of the 
representable range in the specific precision (either for an overflow or underflow) and second, while the decimal number has finite number of 
digits, the exact binary representation has infinite digits or just more significant digits than the allowed by the precision (see 
\cite{articleIEEE}). 

In a previous section the concepts of range and resolution of the numbers that are represented are explained. Also, how both concepts are 
opposed due to the finite precision. A visually way to understand this is thinking of the exponent as a window between two values, and the 
mantissa as the offset of an specific number from the initial value of the window \cite{VisExpl}. In the decimal system the window tells 
which two consecutive power-of-ten are we treating: [0.1,1], [1,10], [10,100], etc. and in binary system the windows jump with powers-of-two 
so [0.5,1], [1,2], [2,4], etc. This window then is divided in equidistant divisions according to the number of bits used for the mantissa. 
However, notice that from one window to the next one the number of divisions is the same and the length of the window much bigger (each 
window doubles the previous), then the resolution in that window is lower and the programmer is loosing capability to represent numbers since 
the absolute value of the number is bigger. Said in other words, the density of IEEE 754 floating-point values near the zero is bigger than 
the density of values with growing values. 

Let's see this with an example, with 23 digits (the omitted leading 1 is always a 1) we have 23bits precision to divide every window, this 
means $2^{23} = 8688608$ possible values. In the window $\left[1,2\right]$ the resolution is $\frac{\left(2-1\right)}{2^{23}} \sim 
0.000000119$ while in the window $\left[32768=2^{15},65536=2^{16}\right]$ the resolution is $\frac{\left(65536-32768\right)}{2^{23}} \sim 
0.003906$. Notice the essential consequences of this, the simple step of writing a number in your program to be used as a constant or stored 
in a variable is going to have an error associated, and this error is going to be higher when working with big numbers than small numbers. 
For simple precision, like in the previous example, when the window corresponds to $\left[16777216=2^{24},8388608=2^{23}\right]$ the absolute 
error committed is $1$. 

The density of representable numbers can be represented in the number line, however, for the representation let's consider a lower precision. 
For example, as it is broaden in \cite{articleIEEE}, consider the case where the base of representation is $2$, there are only $3$ 
significant digits in the mantissa and only $4$ available exponents ($-1, 0, 1, 2$) without sign bit (see Figure \ref{fig:DensityNumbers}). A 
general number in this system is written: $d.dd \quad x \quad 2^{e}\quad \textrm{with}\quad e \in \left[-1, 2\right] $ Take a look at the 
possible decimal values that this system could represent (see table \ref{tab:PossibleValues}).

\begin{table}
    \centering
    \begin{tabular}{| c | c | c | c | c | }
        \hline
        Mantissa/Exponent   & $-1$ & $0$  & $1$  &   $2$  \\ \hline
        1.00                & $0.5$ & $1$  & $2$  &   $4$  \\ \hline
        1.01                & $0.625$ & $1.25$  & $2.5$  &   $5$  \\ \hline
        1.10                & $0.75$ & $1.5$  & $3$  &   $6$  \\ \hline
        1.11                & $0.875$ & $1.75$  & $3.5$  &   $7$  \\ \hline
    \end{tabular}
    \caption{Possible values (shown the decimal value) covered by the example system treated (read text for explanation).}
    \label{tab:PossibleValues}
\end{table}

\begin{figure}[h]
    \centering
    \includegraphics[width= \textwidth]{./doc/Figures/DensityNumbers.png}
    \caption{Representable numbers in the binary numbers system defined by $p = 3$ and $e\in\left[-1, 2  \right]$ (read text for more 
    details).}
    \label{fig:DensityNumbers}
\end{figure}


In those cases where the exact number to be represented is approximated by the nearest IEEE 754 number, the absolute error committed is 
bounded. This happens in the declaration of a variable or the use of a constant value. On the opposite, in the case of operating with 
variables and constants the result of the calculation could be different from the nearest IEEE value to the exact result. 

As an example, consider the binary system explained above, imagine that the program needs to use the value $3.1875_{10} = 11.0011_2 = 
1.10011x2^1_2$. This number will be represented as $3_{10} = 1.10x2^1_2$ in the system proposed so the error committed is $0.1875_{10} = 
0.0011_2$. Notice that this is $0.011_2$ units in the last place (\textit{ulps}). The absolute error in this system, when the exact number is 
approximated by the nearest floating-point value, is bounded by $0.1_2 = 0.5_{10}$ ulps and the exact absolute error committed for a number 
$z$ will be $\abs{d.dd-\left(\frac{z}{2^e}\right)} 2^{3-1}$ ulps. It seems that this absolute error is small for every number to be 
represented, but take into account that the concept ``unit in the last place'' depends on the exponent, said in other words, depends on the 
window of the number to be represented, so $0.5$ ulps in the number $z = 0.5625$ means an error of $0.0625_{10}$ units, but if the number $z 
= 5.5$ is represented by the number $6_{10}$, then the error committed is $0.5_{10}$ units, which is quite bigger. 

Another way to calculate and measure the error when the number is approximated by the nearest value is the relative error. This is the 
difference between both the exact number and the approximation and divided by the exact value. Since this value is divided by the real 
number, the relative error is going to have the same bounds for all the representable numbers, while the absolute error grows for growing 
exponents, the denominator also grows and the relative errors maintains ``constant''. Actually, the relative error does not vary from one 
window to the next one, it repeats the same behaviour, but it does change inside the window, just take a look at his expression. What it is 
interesting is to calculate the relative error that is related to the maximum absolute error. In this case, for every exponent (window), the 
real number varies in the range $z \in \left[\beta^e, \beta \beta^e\right)$ while the maximum absolute error is constant 
$\frac{1}{2}\textrm{ulp}$ so the relative error ranges in: $\left[\frac{\beta}{2}\beta^{-p}, \frac{1}{2}\beta^{-p} \right]$

\begin{IN}
    Use the absolute error, whether expressed absolutely or with ulps, to understand the rounding error and use the relative error to analyse 
    the results of calculations in the computer.
\end{IN}  

For a general number system where $\beta$ is the base of the system, $p$ is the number of digits reserved for the mantissa and $e$ is the 
exponent, the table \ref{tab:errors} shows the errors and their bounds. Notice that the definition of the ulp is composed by a part that 
depends number system ($\beta$ and $p$) and another part that depends on the specific exponent that is represented (the window), $e$. 

\begin{table}
    \centering
    \begin{tabular}{| c | l | }
        \hline
        \textbf{Definition}   & \textbf{Expression }  \\ \hline
        ulp                & $\beta\beta^{-p}\beta^e = 0.0000...\beta' \textrm{x} \beta^e    \quad     \textrm{with} \quad \beta' = 
        \frac{\beta}{2} \textrm{digits}$  \\ \hline
        Absolute error of $z$             & $\abs{d.dddd...\textrm{x} \beta^e - z} = \abs{d.dddd...-\frac{z}{\beta^e}}\beta^e$ \\ \hline
        Maximum absolute error            & $\left[\left(\frac{\beta}{2}\right)\beta^{-p}\right]\beta^e = \frac{1}{2} ulp$  \\ \hline
        Relative error                & $\abs{\frac{z - d.dddd...\textrm{x} \beta^e}{z}}$  \\ \hline
        \begin{tabular}{@{}c@{}} Relative error of the   \\    maximum absolute error    \end{tabular}           & 
        $\left[\frac{1}{2}\beta^{-p}, \frac{\beta}{2}\beta^{-p} \right]$   \\ \hline
    \end{tabular}
    \caption{Different errors committed when the real number $z$ is approximated by the nearest floating-point value.}
    \label{tab:errors}
\end{table}

A good way to understand the expressions and behaviour of these errors is to represented them in the same number line shown above, using that 
representation system. Because of the high density of the representable number in the IEEE-754 system for simple precision, in that case only 
the boundaries of the errors are represented, while in the simpler case of only 3 significant digits, is possible to show the exact error for 
every real number (see Figure \ref{fig:AbsErrorGraph}).

For the relative error, take into account that the each number has its own value, exactly like the absolute error. However, we are more 
interested in the general behaviour of this error. Since it is calculated dividing the absolute error for the real number ($z$) notice that 
for each window the relative error is smaller in the higher values (in the right part of the window) while it is higher in the left part. 
Furthermore, the relative error is exactly the same for all the windows (it does not vary with the exponent of the number represented). Said 
that, the relative error is always bounded by $\frac{\beta}{2}\beta^{-p}$ in the left part of each window and by $ \frac{1}{2}\beta^{-p} $ in 
the right part (see Figure \ref{fig:RelErrorGraph}). 

\begin{figure}[h]
    \centering
    \includegraphics[width= \textwidth]{./doc/Figures/AbsErrorGraph.png}
    \caption{Graphic of the absolute error committed when the real numbers are approximated by the nearest floating-point value with the 
    maximum value represented for each window (exponent).}
    \label{fig:AbsErrorGraph}
\end{figure}

\begin{figure}[h]
    \centering
    \includegraphics[width= \textwidth]{./doc/Figures/RelErrorGraph.png}
    \caption{Graphic of the relative error committed when the real numbers are approximated by the nearest floating-point value with the 
    maximum value represented for all windows (exponent).}
    \label{fig:RelErrorGraph}
\end{figure}

\begin{IN}
    The value $\frac{\beta}{2}\beta^{-p} = \epsilon$ is called machine epsilon, always take into account this magnitude because it bounds the 
    relative error of any number when is rounded to the closest floating-point value. In the case of binary floating-point it can be 
    expressed as $\epsilon = 2^{-p}$ taking the value of $2^{-24} \sim 5.96e-8$ in simple precision (23 bits plus 1 implicit bit), $2^{-53} 
    \sim 1.11e-16$ in double precision and $2^{-113} \sim 9.63e-35$ in quadruple precision. In other conventions, the machine epsilon is 
    considered the value $\beta^{1-p} = \epsilon$ which is the ulp for the value $1.0$ and in this case it is defined as: \textit{machine 
    epsilon is defined as the difference between 1 and the next larger floating point number}.
\end{IN}

%Grafica de las cotas para el IEEE-754 hacer nosotros
The example above helps to understand the behaviour of both errors, for a real situation where simple or double precision of IEEE standard is 
used, the density of numbers is much higher but the curve is the same. For real applications a programmer is more interested in the upper 
bound of the absolute and relative errors already expressed in decimal system since it is the numbers system used for a numerical simulation. 
This values are represented in the Figures \ref{fig:AbsErrorIEEE} and \ref{fig:RelErrorIEEE}.

%Ejemplo con repr exacta y sin repr exacta para distintas precisiones
Now, with all the explanations above, the following examples are easy to understand based on converting the reals to floating-point values 
from the IEEE standard. 
%PONER EJEMPLOS

%ESTE PARRAFO HABRA QUE EXPLICARLO MEJOR
It can be concluded from this chapter that the density of numbers that can be represented near 0 is enormous, while this density decreases as 
we move away from 0. This property of the floating point representation can be used by the numerical programmer through the normalization of 
the operations to perform. For example, consider an ODE solution where the whole equation is dimensionless with all the terms divided by the 
maximum values that the variables are going to reach. In this case, the solution will vary between 0 and 1 where the IEEE 754 floating point 
set is more dense and the absolute error is lower. Anyway, do not forget that the relative error performed in a operation is going to be 
constant. 




\section{Decimal real number from its internal IEEE binary representation}
Use scientific notation with ES format (normalized mantissa).

$$
   x = (-1) ^s \sum_{i=0}^{M-1}  b_i 2^{-i} 2^e 
$$
with $b_0 = 1$. 

$$
   \epsilon = m_1 2 ê - m_2 2^e  = 2^{e-M} 
$$


\section{IEEE binary representation from decimal real numbers}


\section{IEEE exceptions}

\section{Practical guidance to use real numbers} 


The best way to understand the differences between both representations is with an example. 

\begin{IN}
    Before the example, revise these concepts in the context of numerical calculations:
    \begin{itemize}
        \item Precision/Resolution: The smallest change that can be represented in floating point representation, which means, how exactly we 
        can specify a number we want o represent. Sometimes it is used the word ``precision'' for the number of bits used (remember that a 
        change in the least significant bit is the smallest available change) and ``resolution'' for that specific quantity changed. For 
        every value in the exponent, the resolution is fixed by the value of the least significant bit of the mantissa in floating-point 
        arithmetic, or the number bits in the mantissa which is similar. 

        \item Accuracy: How close a value is to what it is meant to be or the closeness of floating point representation to the actual value. 
        This can be used in the result of an operation or the assignment of a constant to a variable, the result should be a value while the 
        computer gives a nearer but not equal number. The accuracy is governed also by the number of bits used in the mantissa. The 
        obligation to round numbers (rounding error) is related to accuracy. Dedicating more bits for the mantissa increases the resolution 
        and the precision.
        
        \item Range: Highest and tiniest number representable with the number of bits available. This concept is governed by the number of 
        bits of the exponent so for a fixed number of bits (precision), dedicating more bits to the exponent and less bits to the mantissa 
        extends the range but decreases the accuracy and resolution. 
    \end{itemize}
\end{IN}







\begin{IN}
    There are some basic concepts that you can revise to confront this section with all the tools. First of all, take a look at the positional numeral systems and how to convert a decimal number to a pure binary number either integer or fractional and either positive or negative. Then, take a look at the standard scientific notation in decimal system used to express large and tiny numbers, at this point revise the scientific notation in binary system which is similar. Once this is done, some conclusions can be obtained:
    \begin{enumerate}
        \item The computer can not use the symbol (-) for negative values and it neither uses an explicit decimal point to represent the fractional part. It needs a method to express reals (and specially negative reals) just using 0's and 1's.
        \item Except for the number $0_{10} = 0_{2}$, all the numbers to be expressed in binary system are going to have the digit 1 as first significant digit. Exactly like in a decimal number the first significant digit must be a digit from 1 to 9. 
        \item Exactly like the case of a decimal number expressed in standard scientific notation ($r=c\cdot b^e$ with $b=10$), where the exponents is chosen to accomplish $\abs{c}\in\left[ 1,10\right)$, for a binary number expressed in standard scientific notation ($r=c\cdot b^e$ with $b=2$), the value of $\abs{c}\in\left[ 1,2\right)$ (except for the 0, which is an special value). That condition is expressed in decimal form, while it is referred to a binary number, the meaning is not other than the mantissa of a binary number expressed in binary standard scientific notation is going to be contained  between 1 and 2 while expressed in decimal form. In binary form, this means that the mantissa of the number always has the digit 1 followed by the fractional point and a series of binary digits. Well, notice that standard IEEE 754 for floating point numbers is really similar to scientific notation in base 2. 
        
        Just as an example, a binary number in standard scientific notation could be $\texttt{1.01101}\cdot2^{3}$. In this case, the mantissa is written in binary form while the base and exponent are written in decimal form, it does not change the meaning. We could write all the number in binary form like: $\texttt{1.01101}_2 \cdot 10_2^{11_2}$
    \end{enumerate}
\end{IN}     

Now the conversion between decimal and floating-point number in the standard IEEE can be treated.

The current standard implemented in most machines to work with floating-point Arithmetic is the IEEE 754 Standard (see table \ref{tab:properties}). Let's take a look at the basics of a representation in this standard. Similarly to the scientific notation, this representation assumes that the binary point is located immediately at the right of the first binary digit, the sign bit as will be seen later. The exponent then will float the binary point to the right or to the left depending on the value. 

\newpage
%\begin{figure}[h]
\begin{sidewaysfigure}
        %\begin{flushleft}
            \centering
            \includegraphics[width= \textwidth]{./doc/Figures/ParametersIEEE.png}
            \caption{Main formats in the standard IEEE 754: single, double and quadruple precision with an example in single precision.}
            \label{fig:ParametersIEEE}
        %\end{flushleft}
        \begin{table}[H]
            %\begin{flushright}
                %\begin{turn}{90}
                    \begin{tabular}{| r | c | c | c | c | c | c | c | c |}
                        
                        \hline
                        Name & Sign & Exp. & Mantissa & Exp. Bias & \begin{tabular}{@{}c@{}}Bits\\  precision\end{tabular}  & \begin{tabular}{@{}c@{}}Normalized \\ range \end{tabular} & \begin{tabular}{@{}c@{}}Approximate\\decimal\end{tabular} & Precision \\ \hline
                        
                        \begin{tabular}{@{}c@{}}Single precision \\ (binary32) \end{tabular}      & 1 & 8  & 23    & +127   & 24 & \begin{tabular}{@{}c@{}}$\pm2^{-126}$ to \\$\pm2^{127+1}$  \end{tabular}    & \begin{tabular}{@{}c@{}}$\pm1.18\cdot10^{ −38}$ to \\ $\pm3.4\cdot10^{38}$ \end{tabular}     & \sim 7.2 digits  \\ \hline
                        
                        \begin{tabular}{@{}c@{}}Double precision \\ (binary64) \end{tabular}    & 1 & 11 & 52    & +1023  & 53 & \begin{tabular}{@{}c@{}}  $\pm2^{-1022}$ to\\  $\pm2^{1023+1}$\end{tabular}  & \begin{tabular}{@{}c@{}} $\pm2.23\cdot10^{ −308}$ to \\ $\pm1.80\cdot10^{308}$ \end{tabular} & \sim 15.9 digits        \\  \hline
                        
                        \begin{tabular}{@{}c@{}} Quadruple precision\\(binary128) \end{tabular}   & 1 & 15 & 112   & +16383 & 113 & \begin{tabular}{@{}c@{}}  $\pm2^{-16382}$ to\\  $\pm2^{16383+1}$\end{tabular}   & \begin{tabular}{@{}c@{}} $\pm3.3621\cdot 10^{-4932}$ to \\ $\pm1.1897\cdot10^{4932}$ \end{tabular}  & \sim 19.2 digits          \\ \hline
                        
                    \end{tabular}                                                       
                %\end{turn}
                \caption{Main properties of the different precisions covered by the IEEE 754 standard.}
                \label{tab:properties}
            %\end{flushright}
        \end{table}
\end{sidewaysfigure}
%\end{figure}

This information can also be accessed by code, take a look at the useful intrinsic functions that can be used in Fortran.

\begin{verbatim}
    real(kind=4) :: x
    real(kind=8) :: y
    
    write(*,*) 'Declaration of x with - real(kind = 4):: x'
    write(*,*) 'Maximum value', huge(x)
    write(*,*) 'Minimum value', tiny(x)
    write(*,*) 'Round_off', epsilon(x)
    write(*,*) 'Significant digits', precision(x)
    
    write(*,*) 'Declaration of y with - real(kind = 8) :: y'
    write(*,*) 'Maximum value', huge(y)
    write(*,*) 'Minimum value', tiny(y)
    write(*,*) 'Round_off', epsilon(y)
    write(*,*) 'Significant digits', precision(y)
\end{verbatim}

which results in:

\begin{verbatim}
    Declaration of x with - real(kind = 4):: x
    Maximum value  3.4028235E+38
    Minimum value  1.1754944E-38
    Round_off  1.1920929E-07
    Significant digits           6
    
    Declaration of y with - real(kind = 8) :: y
    Maximum value  1.797693134862316E+308
    Minimum value  2.225073858507201E-308
    Round_off  2.220446049250313E-016
    Significant digits          15
\end{verbatim}

%\newpage
%\begin{table}[H]
%    \begin{flushright}
%        \begin{turn}{90}
%            \begin{tabular}{| r | c | c | c | c | c | c | c | c |}
%                
%                \hline
%                Name & Sign & Exp. & Mantissa & Exp. Bias & Bits precision & \begin{tabular}{@{}c@{}}Normalized \\ range \end{tabular}   & Approximate decimal  & Precision \\ \hline
%                
%                \begin{tabular}{@{}c@{}}Single precision \\ (binary32) \end{tabular}      & 1 & 8  & 23    & +127   & 24 & \begin{tabular}{@{}c@{}}$\pm2^{−126}$ to \\$\pm2^{127+1}$  \end{tabular}    & \begin{tabular}{@{}c@{}}$\pm1.18\cdot10^{ −38}$ to \\ $\pm3.4\cdot10^{38}$ \end{tabular}     & \sim 7.2 digits  \\ \hline
%                
%                \begin{tabular}{@{}c@{}}Double precision \\ (binary64) \end{tabular}    & 1 & 11 & 52    & +1023  & 53 & \begin{tabular}{@{}c@{}}  $\pm2^{−1022}$ to\\  $\pm2^{1023+1}$\end{tabular}  & \begin{tabular}{@{}c@{}} $\pm2.23\cdot10^{ −308}$ to \\ $\pm1.80\cdot10^{308}$ \end{tabular} & \sim 15.9 digits        \\  \hline
%                
%                \begin{tabular}{@{}c@{}} Quadruple precision\\(binary128) \end{tabular}   & 1 & 15 & 112   & +16383 & 113 & \begin{tabular}{@{}c@{}}  $\pm2^{-16382}$ to\\  $\pm2^{16383+1}$\end{tabular}   & \begin{tabular}{@{}c@{}} $\pm3.3621\cdot 10^{-4932}$ to \\ $\pm1.1897\cdot10^{4932}$ \end{tabular}  & \sim 19.2 digits          \\ \hline
%                
%            \end{tabular}                                                       
%        \end{turn}
%        \caption{Main properties of the different precisions covered by the IEEE 754 standard.}
%        \label{tab:properties}
%    \end{flushright}
%\end{table}

%\newpage
%\begin{table}[H]
%    \begin{flushleft}
%        \begin{turn}{90}
%            \begin{tabular}{| r | c | c | c | c | c |}
%                
%                \hline
%                Name & Sign bits & Exp.bits & Mantissa bits & Exp. Bias & Bits precision \\ \hline
%                
%                \begin{tabular}{@{}c@{}}Single precision \\ (binary32) \end{tabular}      & 1 & 8  & 23    & +127   & 24  \\ \hline
%                
%                \begin{tabular}{@{}c@{}}Double precision \\ (binary64) \end{tabular}    & 1 & 11 & 52    & +1023  &   53  \\  \hline
%                
%                \begin{tabular}{@{}c@{}} Quadruple precision\\(binary128) \end{tabular}   & 1 & 15 & 112   & +16383 & 113 \\ \hline
%                
%            \end{tabular}                                                       
%        \end{turn}
%        \caption{Fruta disponible}
%        \label{tab:properties}
%    \end{flushleft}
%\end{table}
%\begin{table}[H]
%    \begin{flushright}
%        \begin{turn}{90}
%            \begin{tabular}{| r | c | c | c |}
%                
%                \hline
%                Name & \begin{tabular}{@{}c@{}}Normalized \\ range \end{tabular}   & Approximate decimal  & Precision \\ \hline
%                
%                \begin{tabular}{@{}c@{}}Single precision \\ (binary32) \end{tabular}      &  \begin{tabular}{@{}c@{}}$\pm2^{−126}$ to \\$\pm2^{127+1}$  \end{tabular}    & \begin{tabular}{@{}c@{}}$\pm1.18\cdot10^{ −38}$ to \\ $\pm3.4\cdot10^{38}$ \end{tabular}     & \sim 7.2 digits  \\ \hline
%                
%                \begin{tabular}{@{}c@{}}Double precision \\ (binary64) \end{tabular}    &  \begin{tabular}{@{}c@{}}  $\pm2^{−1022}$ to\\  $\pm2^{1023+1}$\end{tabular}  & \begin{tabular}{@{}c@{}} $\pm2.23\cdot10^{ −308}$ to \\ $\pm1.80\cdot10^{308}$ \end{tabular} & \sim 15.9 digits        \\  \hline
%                
%                \begin{tabular}{@{}c@{}} Quadruple precision\\(binary128) \end{tabular}   & \begin{tabular}{@{}c@{}}  $\pm2^{-16382}$ to\\  $\pm2^{16383+1}$\end{tabular}   & \begin{tabular}{@{}c@{}} $\pm3.3621\cdot 10^{-4932}$ to \\ $\pm1.1897\cdot10^{4932}$ \end{tabular}  & \sim 19.2 digits          \\ \hline
%                
%            \end{tabular}                                                       
%        \end{turn}
%        \caption{Fruta disponible}
%        \label{tab:properties}
%    \end{flushright}
%\end{table}

\FloatBarrier
The explanation and example is done with single precision, the 4-bytes (32bits) format. Check in the table \ref{tab:properties} that 23bits are reserved for mantissa, 8 bits for the exponent and 1 bit for the sign. Actually, since the mantissa always omit the value 1 before the binary point in the representation (because it is always a 1 in normalized notation) the actual precision of the mantissa is 24 bits and not 23. Although it is not represented, when the computer needs to process any number it adds that value 1 to operate properly with reals. In the memory of the machine the standard says that the first bit is the sign bit, followed by the bits of the exponent and followed by the 23 mantissa bits. This order is the same for any precision but with different quantity of bits. 

Let's convert the number $-110.3125$ to a 32bits IEEE 754 floating point explaining step by step.

\begin{enumerate}
    \item First of all give a value to the sign bit, which is decided by the sign of the mantissa. If your number is negative use a 1, if it is positive, a 0. In this case the example is negative so our first bit in the memory is a 1.
    
    \item Convert the number to pure binary, notice that pure binary is similar to the fixed-point representation but in pure binary there are not limits on the number of bits. Since we have defined the sign in the previous step, consider now the number as positive. The whole part is $110_{10}=\texttt{1101110}_2$ and the decimal part is $0.3125_{10}=\texttt{0.0101}_2$. Hence, the complete number in pure binary is $110.3125_{10}=\texttt{1101110.0101}_2$. 
    
    \item Put the binary point in the first position, according to the scientific notation and then find the unbiased exponent:
    
     $\texttt{1101110.0101}_2=\texttt{1.1011100101}_2\cdot2^6$.
    
    \item Omit the first significant digit, which is always a 1, there i no need to waste a representable digit covering this information: $\texttt{1.1011100101}_2\cdot2^6 \rightarrow \texttt{.1011100101}_2\cdot2^6$.
    
    \item Calculate the biased exponent. For 4-bytes precision (8 bits reserved for exponent) the bias is $+127$. We are not covering here the advantages or disadvantages of the biased exponent against two's complement or one's complement. All the option allow to store positive and negative exponents in a more or less proximate way to the pure binary representation, in the case of the bias all the exponents have an offset from the smallest value. In this precision, the smallest available value is $-126$ and the highest is $+127$. Hence, our exponent of $6$ is covered by the value $6+127 = 133_{10}= \texttt{10000101}_2$. 
    
    \item Fill in the rest of mantissa digits with zeros at the right to complete the whole number representation. The result is: 
    
    $\texttt{1}\quad \texttt{10000101}\quad \texttt{10111001010000000000000}_{2, IEEE 754}$
    
    In some cases, instead of filling with zeros maybe is necessary to truncate or round the excess of digits obtained in the conversion to pure binary, either because the result has more than 23 decimal digits or because, for example, we have obtained a periodic binary number. The normal way to do it is that one called ``round to the nearest, ties to even'': if the 24th bit is a zero we chop the digits, if it is a 1 we add one to the 23th bit. The special case of a 1 in the 24th bit followed by zeros involves (similarly to the way of rounding in decimal system) that, if the 23th is a 1, we add one to the mantissa, if the 23th is a zero, we chop the digits. 
\end{enumerate}

In order to convert from the IEEE standard to decimal system we have to cover the same steps in the opposite order. In that case, do not forget to add 1 to the mantissa and always take into account that some rounding could has happened in the process of converting the value to binary IEEE 754 representation. Hence, is possible that after converting the number to binary and again to decimal system, the result is not exactly the same.  

Finally, the standard reserves some exponent values to special situations (see Table \ref{tab:SpecialValues}). It is important to always consider that there are two numbers equal to $0$, $\pm 0$. Infinity and NaN's are essential to denote whether the result of a computation is too large to be represented in IEEE-754 (then infinity is used, for example, when the maximum exponent is exceeded in an operation, also called overflow) or a variable didn't obtained a known value or it is illegal (then NaN is used). The operations between those special values are totally defined in the standard (see Table \ref{tab:SpecialOperations}). The denormalised numbers are used when underflow occurs, which means, a value is obtained in the gap that exits between the smallest normalised number representable by the standard and the same negative value. That gap is many orders of magnitude larger than the machine epsilon (the distance between two representable values outside the gap, this is treated in the section \ref{sec:roundoff}). With the denormalised numbers (or subnormal numbers) a gradual underflow is achieved. Said in other words, the numbers too small to be represented (and then forced to be replaced by zero) are gradually decreased. In this case the number does not have an assumed leading one before the binary point, it is a zero. The range of the mantissa is then $\abs{c}\in\left[ 0,1\right)$.

\begin{table}
    \centering
    \begin{tabular}{| c | c | l |}
        \hline
        Exponent & Mantissa & Value represented \\ \hline
        All 0's  & All 0's & $\pm 0 $ depending on the sign bit, they are equal  \\ \hline
        All 1's  & All 0's & $\pm \infty$ depending on the sign bit \\ \hline
        All 1's & NOT all 0's & Not a Number (NaN)  \\ \hline
        All 0's  & NOT all 0's & Denormalised numbers  \\ \hline
    \end{tabular}
    \caption{Special values covered by the IEEE 754 standard.}
    \label{tab:SpecialValues}
\end{table}

\begin{table}
    \centering
    \begin{tabular}{| l | c |}
        \hline
        Operation & Result \\ \hline
        $ n / \left(\pm \infty\right) $            &     $	0$              \\ \hline
        $\pm \infty * \left( \pm \infty\right) $    &     $	\pm \infty$      \\ \hline
        $\pm$ nonZero$ / \pm 0 $            &    $	\pm \infty$       \\ \hline
        $\pm $finite$\quad * \pm \infty $      &    $	\pm \infty$       \\ \hline
        $\infty+\infty$  or  $\infty- \left(-\infty\right) $	     &       $+\infty$\\ \hline    
        $-\infty - \infty$   or  $-\infty + \left( -\infty \right) $   &        $	-\infty$\\ \hline
        $\pm 0 / \pm 0 $                  &        $	NaN$         \\ \hline
        $\pm \infty / \pm \infty $    &        $	NaN$     \\ \hline
        $\pm \infty * 0 $            &        $	NaN$     \\ \hline
        $ NaN == NaN $               &    $	False $      \\ \hline
    \end{tabular}
    \caption{Special operations covered by the IEEE 754 standard.}
    \label{tab:SpecialOperations}
\end{table}

Notice that the exponent with all 0's would correspond to $0_{10}$ for biased exponent while the exponent all 1's would be equal to $255_{10}$ for the biased exponent in simple precision.

An essential concept to manage when programming is the relation between the precision used in a number and his decimal significant digits, which means, the number of reliable digits. This concept is broaden in the next section, however, notice now that when a decimal value is written in a program to be assigned to a variable, used as a constant or as a named constant, the binary value stored is the nearest IEEE 754 floating point value to the expected value. We always have to work with a rounding error, hence, the main question is how many digits are are reliable in the assignment of a variable? Well, it depends on the precision used, in the case of simple precision normally the first 7 digits are reliable but remember that the resolution is not the same everywhere. In some places the binary representation is dense and extra digits can be considered, in other places less digits. 

As a quick approximation, consider that $log_2(10)\approx 3.32 $ ($2^n = 10$) binary digits are needed in order to represent one single decimal digit and with simple precision we count on 24 binary digits for mantissa (23 + omitted leading 1) so: $24/3.32 = 7.2$ decimal digits are represented. Notice that this does not mean that the first 7 digits of the represented number are equal to the 7 first digits of the expected value, it means that the relative error between both numbers is in the order of magnitude of $1e-7$. 

Just consider the number $1.3_{10}$, in the standard IEEE 754 in simple precision this value is stored as $$\texttt{ 0 01111111 01001100110011001100110}_2 = 1.2999999523162841796875_{10} $$

which means that the error caused by the conversion is $-4.76837158203125E-8$ while only the first digit is the same. Furthermore, this is only the error caused by the conversion to the binary value, more errors can appear when the program operates with that value. 

\begin{IN}
    Notice that not all the integers contained in the representable range of real numbers have exact representation in this set. In the case of simple precision for example, at least all the integers with 6 or less significant decimal digits can be converted to a IEEE 754 value and not lose precision in the conversion. Some integers with 9 digits can also be converted but more than 9 digits is inevitably related to loss of precision. As an example, the number $899565_{10}$ is exactly transformed, but the number $45962178_{10}$ is converted to $ 45962176_{10}$ with an error of $-2$ units. 
\end{IN}

With the following subroutines and functions you can follow the deconstruction of a real number through his internal bits representation. This representation is charged in a string of bits and once extracted the different parts of the string (sign bit, exponent and mantissa), the value is reconstructed according to the standard. Notice that this code covers the normalized numbers and not the special values.

Either single, double or quadruple precision, the number is converted to a quadruple precision number. Notice that this conversion carries the error made in the first assignation (simple, double or quadruple) and the assignment to quadruple just simplifies the rest of the code since it is developed only for one kind. If a simple precision value is introduced in the program, where only 32bits are used, at the moment of assigning to the 128bits, lots of values will be filled will be filled with zeros. Later, the program extracts sign, exponent and mantissa. While sign and exponent are charged to an integer variable, the mantissa is stored in a quadruple precision real value (already adding the omitted 1). Taking into account the bias for the exponent and the basic conversion from binary to decimal the number can be reconstructed. While representing in the screen the bits, do not forget that leading zeros are not displayed, however, they are there and blanks are left instead.

This program is not only useful to understand the conversion but also to make tests with the same number in different precisions. Call the program with the same number declared as simple, double and quadruple precision and take a care look at the big differences in the reconstructed number, specially the significant digits for all the precisions. 


\newpage
 \renewcommand{\home}{./Fortran/sources/IEEE} 
\listings{\home/IEEE_representation.f90}
{subroutine mantissa_exponent_base_2}{end subroutine}
{Subroutine to reconstruct the number in IEEE_representation.f90} 

\newpage
 \renewcommand{\home}{./Fortran/sources/IEEE} 
\listings{\home/IEEE_representation.f90}
{function normalized_mantissa}{end function}
{normalized_mantissa function in IEEE_representation.f90} 

\newpage
 \renewcommand{\home}{./Fortran/sources/IEEE} 
\listings{\home/IEEE_representation.f90}
{biased_exponent}{end function}
{biased_exponent function in IEEE_representation.f90} 





%\section{Writing floating point expressions}

Exactly like in the case of integers, in a code, the programmer works with real variables and real constants, and it is interesting to manage properly the precision reserved for both. Specially in strong typed languages like Fortran, the programmer must understand how assignations are performed and the typical errors that could appear when types of variables are not properly defined. If the type of a variable is defined in the declaration of the variable it is typically used \texttt{real(kind = n) :: } or \texttt{real*n :: } where \texttt{n} is 4, 8 or 16. For each situation the strategy can vary, however, it can be interesting to write the program independently of the precision, this means that the declaration of variables is made with no kind specification, just \texttt{real :: x}. If the ``kind'' parameter is not specified, the kind of the variable is the default value, which can be modified in the compiler options. The main advantage of this strategy is that the code can be executed with all the precisions by changing an option of the compiler. Thus, the program is not dependent on the precision imposed when it was written. 

In the case of constants the precision can be also defined when used, just add \texttt{\_k} besides the value of the constant with \texttt{k} is 4, 8 or 16 and that kind will be imposed to the constant. However, once again, this is not the common way to do it, consider not declaring the kind for the constants neither so it automatically adopts the default real kind value and, in case different precision is needed for the whole program, that option is changed in the compiler options. Unless you have changed it, the default real kind is simple precision (\texttt{kind = 4}). 

When the constant has an exponent part, for example, \texttt{1.35E+9} which means \texttt{1.35 * 10**9} the constant is automatically a simple precision value because of the ``E'', unless the kind parameter besides specifies anything different (\texttt{1.35E+9\_8} would be double precision). If a ``D'' or a ``Q'' is used, then the constant is double or quadruple precision instead, in this case no optional parameter \texttt{\_k} can be used since the precision is already chosen. All these different ways to declare the same thing should be used in those cases the programmer needs it, consider not declaring the precision for each real value in the program and make it independent of the precision. When the constant does not have an exponent and does not have decimal part (for example the real value \texttt{8.}) the number written must have a decimal point to tell the compiler that is a real value, whether the kind parameter is imposed or not. Otherwise, the compiler will consider it an integer value. Later in the text is treated the possible errors that appears commonly in a program when operations between different types of variables and constants are performed.

%Ejemplo de declaracion viciada y ejemplo de declaracion no viciada
%Ejemplo de real (N) que se pueda usar en la realidad. 

\begin{IN}
    As a conclusion, get used to think first the needs of your program regarding the precision of the real variables and constants so you can define it properly in their declarations. In case you prefer to write a code that does not depend on the precision do not specify values for precisions and choose the proper value in the compiler options for each compilation and execution. Anyway, do not forget to always specify that a constant is a real value, so it is not confused with an integer. Use one of the following ways:
    \begin{enumerate}
        \item Use a decimal point after the integer part of the number so it is clear that the number is treated as a real. For example \texttt{3. * 78.} is an operation between reals. 
        \item If the real constant has exponent use any of the symbols ``E'', ``D'' or ``Q'' depending on the precision. Use ``E'' if you do not want to specify precision and let the compiler use the default value. In this case the decimal point is optional. For example \texttt{45E-3 * 7.E2} is also an operation between reals with the default precision imposed by the compiler. In this case, after the symbol it must be an integer number, but you can use the value zero, for example \texttt{7e0 * 34.}
        \item If you need to work with an integer variable but operate it as a real value, then use the intrinsic function \texttt{real (N)}. In this situation the variable \texttt{N} which may be declared as an integer value could be used in operations with reals properly. 
    \end{enumerate}
\end{IN}

Unicamente tener cuidado con la asignacion de una constante o variable de doble a simple que no estas aprov3echando toda la precision y de simple a doble, que se hace la asignacion con el error que lleve el numero en simple precision. 


%Poner ejemplos del epsilon en fortran
%Ver lo codigos que ya tengo hechos. Detalle de como mostrar resultados
%Recomendacion en la notacion


%Acabar el articulo de What Every... y capítulo de Trefethen




    %--------------------------------------------------------------------------------------------------------------------------------------
   % \section{Operations}

Take a look at the following example of simple arithmetic operations between different data types (whether different type or kind in the same type):

\begin{verbatim} 
    write(*,'(a20, f17.15)') '1.1/2.        ', 1.1/2. 
    write(*,'(a20, f17.15)') '1.1e0/2e0     ', 1.1e0/2e0
    write(*,'(a20, f17.15)') '1.1d0/2d0     ', 1.1d0/2d0 
        
    write(*,'(a20, f17.15)') '1.1/2         ', 1.1/2
    
    write(*,'(a20, f17.15)') '1.1/2d0       ', 1.1/2d0
\end{verbatim}

The result when the compiler has default real kind defined as simple precision is the following, try to understand why those results:

\begin{verbatim}
    1.1/2.        0.550000011920929 
    1.1e0/2e0     0.550000011920929 
    1.1d0/2d0     0.550000000000000 

    1.1/2         0.550000011920929 

    1.1/2d0       0.550000011920929 
\end{verbatim}

Let's analyse each example and obtain some conclusions from them. 

The first three examples performs the same operation, it divides the number \texttt{1.1} (which does not have exact representation in the standard IEEE 754) by two, which is exact. They perform the operation with no precision imposed the first two of them and in double precision the third one. However, if we change the default real kind in the compiler options and impose to treat constants and variables as double precision by default, the result is:

\begin{verbatim}
    1.1/2.        0.550000000000000
    1.1e0/2e0     0.550000000000000
    1.1d0/2d0     0.550000000000000
\end{verbatim}

The conclusion to obtain from this is: write codes that do not depend on a precision, just use the first case (\texttt{1.1/2.}) and change the compiler options whether you need simple precision or double precision result. It is not necessary to write in each operation \texttt{d0} to make sure that the operation is performed in double precision, just configure the compiler to treat all constants (and variables) as double precision. In the next example it is demonstrated that writing (\textit{2.}) is not necessary neither. However, notice that in order to force the constant to be real (\texttt{e0}) is not needed. 

Now take a look at the fourth example, it uses the integer 2 instead of converting it to a real number. While operations between two integer operands or between two real operands are developed as expected, a mixed-mode expression (where different data types are involved) must be treated with care. Arithmetic involving different types of operands or different kinds of the same type (i.e. \texttt{real (kind 4)} and \texttt{real (kind 8)}) will be carried out by converting the lowest-ranking operand to the highest-ranking operand so the result has this type and kind. The table \ref{tab:ranking} shows the ranking of each type. 

\begin{table}[h]
    \begin{tabular}{| c | c |}
        
        \hline
        Data Type & Ranking \\ \hline
        LOGICAL(1) and BYTE & Lowest \\ \hline
        LOGICAL(2)     &  . \\ \hline
        LOGICAL(4)     &  . \\ \hline
        LOGICAL(8)      & . \\ \hline
        INTEGER(1)     &  . \\ \hline
        INTEGER(2)    &   . \\ \hline
        INTEGER(4)    &   . \\ \hline
        INTEGER(8)    &   . \\ \hline
        REAL(4)     &  . \\ \hline
        REAL(8)     &    .   \\ \hline
        REAL(16)    &   . \\ \hline
        COMPLEX(4)  &     . \\ \hline
        COMPLEX(8)  &     . \\ \hline
        COMPLEX(16) &    Highest\\ \hline
        
        
    \end{tabular}                                                       
    \caption{Ranking assigned to each data type, arithmetic will be performed with the highest ranking.}
    \label{tab:ranking}
\end{table}

This means that the integer 2 is automatically converted to a real value (which has higher-ranking associated) and the operation is performed. If double precision is needed is simple, just change the compiler option and execute the same program:

\begin{verbatim}
    1.1/2         0.550000000000000
\end{verbatim}

The main conclusion is that there is no need of specifying always that constants are real values if the operation is performed with one operand being already real. But do not forget that at least one operand must define the type of operation to perform, if you do not write at least one real value, you are operating in the integers field and the divisions in the integer field totally ignore the decimal part of the result, so it is truncated (the rest of the operations in the integer field are performed as expected):

\begin{verbatim}
    write(*,'(a20, f17.15)') '1/3           ', 1/3
\end{verbatim}

which results in the :

\begin{verbatim}
    1/3           0.000000000000000
\end{verbatim}

The situation can be tricky when more operands are involved: 

\begin{verbatim}
    write(*,'(a20, f17.15)') '5/2 * 3. =      ', 5/2 * 3.
    write(*,'(a20, f17.15)') '3. * 5/2 =      ', 3. * 5/2
\end{verbatim}

\begin{verbatim}
5/2 * 3. =      6.000000000000000
3. * 5/2 =      7.500000000000000
\end{verbatim}

Both are the same operation, however the first example is not properly performed since the precedence of the operation is from left to right in products and divisions. Hence, 5/2 is operated in first place, and the result is 2 in the integer field, which multiplied by 3. is 6. At least, in the division, it would be nice to force one value to be real with no need of changing the order of the operation. 

Take a look at the following examples in real situations:
PONER EJEMPLOS REALES
%Ejemplo de dividir por 2. en grids
%Otros ejemplos
%Poner el caso de variable constante N que hay que pasar a real usando real(N) FUNDAMENTAL


Finally, look at the fifth example, it can be a little tricky to understand at first. Notice that the compiler is configured for default real kind in simple precision so the value 1.1 is simple precision. According to the table above we could think that the value 1.1 is transformed to double precision in order to be operated with the value \texttt{2d0}. However, the result is clearly carrying with the round-off of the value 1.1 in simple precision. The reason is that the value 1.1 is stored in double precision but no transformed to double precision. 

%Aqui poner los bits de este ejemplo

If the same code is executed with default double precision then the value 1.1 is already double precision so there is not problem. Once again, according to our first conclusion, writing \texttt{2d0} in both cases is not necessary at all and just blurs the program.

\begin{verbatim}
    1.1/2d0       0.550000000000000
\end{verbatim}

%Profundizar en UNA sola operacion entre dos numeros, que precision asegura. epsilon y epsilon de la maquina




To be explained:

\begin{verbatim} 
x**2  = x * x 

y = 2 * x 

y = 2d0 * x 

y = 2. * x 

x**2d0 = exp( 2 * ln x ) 


x = 1 / 2
x = 1 / real(2) 
x = 1 / 2. 

x = 1d0 * i / N  ! NO GUSTA 

x = i / real(N) 

! same numbers 
x = 1 
x = 1. 
x = 1D0 
x = 1e0 

y = x**2
y = x**2.

\end{verbatim} 






   
        
  
  \part{Advanced programming techniques: }\label{PartIII}
       
 
 
 
  
\chapter{Overview} 
One of the main characteristics to reuse code is generic programming.  
Generic programming is based on abstract variable types that are then instantiated when they are used for specific variable type.

Since Python is non typed language, 
generic programming in this language is straightforward. 
However, in Fortran the use of abstract  \lstinline{class(*)} 
allows to use different data types at run time. 
  
    
  \vspace{0.5cm}
   \renewcommand{\home}{./Fortran/sources/Advanced_programming} 
   
    \listings{\home/advanced_programming_techniques.f90}{while}{Games}
             {advanced_programming_techniques.f90}
  
  \newpage 
  
   \vspace{0.5cm}
     \renewcommand{\home}{./Python/sources/Advanced_programming} 
     
      \listings{\home/advanced_programming_techniques.py}{while}{Wrappers}
               {advanced_programming_techniques.py}
        
  %__________________________________________________________________________________________________
 \chapter{Scope} 
 
 \section{Introduction}
 One of the most important matters that we need to understand when we begin to write our own programming codes is the scope. In other words,  variables which are public and those which are private. This is called the scope of the visibility of some variable in some part of our code. 
 
 In any programming language, the scope of variables, objects, functions or procedures is the set of statements in which the variable can be used or modified. The region of a program in which this variable or identifier is visible is called the scope. 
 
 Hence, public and public variables or procedures are specified explicitly. If not, local and global variables are visible. Local variables are those which specified inside the function or subroutine that we are dealing with and global variables are those that can be accessed by common variables of my own module  or by inclusions of other modules.   
 

 In the following code,  variables $x, y,z $ are visible inside the subroutine \texttt{Test}.  Variable $ z $ is a local variable, $ y $ is a global variable of module \texttt{modB} and $ x $ is a global variable of module \texttt{modA}.  All global variables of \texttt{modB} are seen in \texttt{Test}  by means of the  sentence \texttt{use modB}. Besides, since \texttt{modB} uses \texttt{modA}, all global variables of \texttt{modA} are seen in \texttt{modB}.   
 \vspace{0.5cm} 
 
 
 
 \newpage  
 \section{Fortran}
 
 \renewcommand{\home}{./Fortran/sources/scope} 
 \listings{\home/modB.f90}{module modB}{end module}{modB.f90}
 
 \listings{\home/modA.f90}{module modA}{end module}{modA.f90}
 
 
 \newpage
 \section{Python}
 
 
         
        \chapter{First class functions and lexical scoping} 
  Named parameters and default parameters 
  
  \section{Introduction} 
  
  \newpage 
  \section{Fortran} 
   \renewcommand{\home}{./Fortran/sources/Advanced_programming/functional programming} 
  \listings{\home/First_class_functions.f90}{abstract interface}
  {end interface}{First_class_functions.f90}
  
  
  \listings{\home/First_class_functions.f90}{subroutine Function_examples}
  {end subroutine}{First_class_functions.f90}
  
  \newpage 
  \listings{\home/First_class_functions.f90}{real function Integral}
  {end function}{First_class_functions.f90}
  
  
  \listings{\home/First_class_functions.f90}{real function Moment}
  {end function}{First_class_functions.f90}
  
  \newpage
  \section{Python}
   \renewcommand{\home}{./Python/sources/Advanced_programming/functional programming} 
 
  \listings{\home/First_class_functions.py}{from}
  {x**2}{First_class_functions.py}
  
       
\chapter{Overloading functions and operators}

From the mathematical point of view operators, such as\texttt{+}, can be 
applied to elements of sets with different types. Additionally, functions representing 
common concepts, such as integrals, can be applied to dimensional spaces of different rank. 
It is desirable to maintain the same opeands or the same function concepts independently 
of data types and dimensional rank.  

If the programming language is a compiling language, such as Fortran, 
the compiler identifies, at compilation time, the type of the operands or the arguents 
involved when calling a function and it selects automatically the proper operation. 
If the language is interpreted and non-typed language, such as Python, overloading is not supported. 
However, there are many alternatives to emulate overloading in Python. 

For example, the operator \texttt{+} 
is defined differently for real numbers of complex numbers. However, the same operator 
is used for those different operations. This is done by overloading the operator. 
If $x$ and $y$ are reals, the expression $ x+y $ involves the operator \texttt{+}
which is the classical sum. 
However, if  $u$ and $v$ are complexes, 
the expression $ u+v $ involves the same operator \texttt{+}. It this case, 
the result yields a complex number in which 
the real part is the sum of real parts of $ u $ and $ v $
and the imaginary part is the sum of imaginary parts of $ u $ and $ v $.
Many operands and functions are previously overloaded by the native language 
as it is this example. 




Additionally to operands, functions can also be overloaded. 
In this chapter, the function \texttt{Integral} is created to overload line integrals 
and surface integrals. 
Let's consider the  line integral in a compact segment $[a, b]$: 
\begin{equation}   
    I =  \int _a ^b f(x) \ dx, \qquad f: \mathbb{R} \rightarrow \mathbb{R},   
\end{equation} 
and the surface integral in a compact square $ \Omega = [a, b] \times [c,d]$:   
\begin{equation}   
    I =  \int _{\partial \Omega} f(x, y) \ dx \ dy, 
    \qquad f: \mathbb{R}\times \mathbb{R} \rightarrow \mathbb{R}.   
\end{equation} 
From the software point of view, using the same word \texttt{Integral}
to invoke surface or line integrals is desirable. 
Besides, a surface integral can be expressed by the following way: 
\begin{equation}   
    I =  \int _{a} ^b \left(   \int _{c} ^d f(x,y) \ dy      \right) \ dx,
\end{equation} 
in which the integrand is defined with the parametric line integral:
\begin{equation}   
   I_1(x) = \int _{c} ^d f(x,y) \ dy.  
   \label{I1}   
\end{equation} 
The final expression for the surface integral is:  
\begin{equation}   
    I =   \int _{a} ^b I_1(x) \ dx. 
    \label{I2D}  
\end{equation} 
Hence, a surface integral can be defined by functional composition of line integrals. 
In the expression,  $ I_1(x) $, the variable $ x $ acts as a parameter
and $ I_1(x) $ is calculated by means a line integral. 
 
In the following pages, these two concepts overloading and functional composition
are taken into account to write two implementations in Fortran and Python. 

 
\newpage  
\subsection*{Fortran}

\renewcommand{\home}{./Fortran/sources/Advanced_programming/overloading} 
\lstfor

First, the overloading implementation is explained. Once two different functions are 
created for line and surface integrals, the overloading implementation in done
with the following code: 
\vspace{0.5cm}  
\listings{\home/overloading.f90}{interface Integral}{end interface}{overloading.f90}
At compilation time, function invocations are analyzed ans its proper function
is identified. In the following code, two integrals are invoked with 
different number and type of arguments.
\vspace{0.5cm} 
\listings{\home/overloading.f90}{subroutine test_Integral}{end subroutine}{overloading.f90}
The first \texttt{Integral} invocation has three arguments, the two first are reals and the 
third \texttt{f1} is a real function.
The second \texttt{Integral} invocation has five arguments, the four first are reals and 
the fith \texttt{f2} is a real function 
$(\mathbb{R}\times \mathbb{R} \rightarrow \mathbb{R})$.
What rest is to write, \texttt{Integral1D} and \texttt{Integral2D} accoding to their 
specifications.  
\newpage  

Let's begin with  \verb|Integral1D| implementation. 
Even though the numerical method to approximate the line integral is not relevant, 
a Riemann approximation is shown in the following code to have a complete testing
example that could be used as a template. 
\vspace{0.5cm} 
\listings{\home/overloading.f90}{function Integral1D}{end function}{overloading.f90} 
Note that the third argument is defined as a \verb|f_R_R| procedure or function. 
Hence, the definitions of \verb|f_R_R| for line integrals and \verb|f_R2_R|
for surface integrals are given in the following code: 
\vspace{0.5cm} 
\listings{\home/overloading.f90}{abstract interface}{end interface}{overloading.f90} 
 
\newpage  
Let's finish with \verb|Integral2D| implementation. 
Taking into account the definition of a surface integral expressed 
by  equations (\ref{I2D}) and (\ref{I1}), the following implementation is clear: 

\vspace{0.5cm}  
\listings{\home/overloading.f90}{function Integral2D}{end function}{overloading.f90} 
\verb|Integral2D|  is built by a line integral (\verb|Integral1D|) 
in which the integrand is parametric line integral defined by: 
\vspace{0.5cm}   
\listings{\home/overloading.f90}{function Parametric_I1D}{end function}{overloading.f90} 
    
 
 
\newpage
\subsection*{Python}
\renewcommand{\home}{./Python/sources/Advanced_programming/overloading}
\lstpython 
The same concepts are applied to implement an overloaded \verb|Integral| in Python. 
As it is mentioned repetitively, Python allows writing shorter codes 
than Fortran but paying the price of no specifying function arguments or variables. 


Even though Python does not allow overloading, an easy proposa  is shown to 
emulate overloading in the following code: 
\vspace{0.5cm} 
\listingsp{\home/overloading.py}{def Integral }{None}{overloading.py} 
At execution time, the \verb|Integral| function 
checks how may arguments are present. If the invocation is done with three
arguments, line integral is assumed. 
 If the invocation is done with five
arguments, surface integral is assumed. Otherwise, \verb|Integral| function
will show an error.  

In the following code, an example of the overloaded  \verb|Integral| is shown: 
\vspace{0.5cm}
\listingsp{\home/overloading.py}{def test_Integral}{y}{overloading.py}
Note that the first invocation has three arguments and the second invocation has five 
arguments. Note also that anonymous functions or lambda functions have been 
expressed in the last argument to make the example more compact.  


\newpage
Finallly, \verb|Integral1D| and \verb|Integral2D| ae implemented according to their
definitions.
\verb|Integral1D| is approximated by Riemann sums 
\vspace{0.5cm}
\listingsp{\home/overloading.py}{def Integral1D}{return}{overloading.py}

And \verb|Integral2D| is expressed by means of \verb|Integral1D| function 
according to its definition given by (\ref{I2D}) and (\ref{I1}). 
\vspace{0.5cm}
\listingsp{\home/overloading.py}{def Integral2D}{a, b, I1}{overloading.py}


 
       

\chapter{Vector operations}

\section{Introduction}

\newpage
\section{Fortran}

 
 \renewcommand{\home}{./Fortran/sources/Advanced_programming/functional programming}  
 \listings{\home/Fourier.f90}
 {subroutine Fourier_examples}
 {end subroutine}{Fourier.f90}



\listings{\home/Fourier.f90}
{elemental function Fourier_series}
{end function}{Fourier.f90}




\newpage
\section{Python}



\renewcommand{\home}{./Python/sources/Advanced_programming/functional programming} 
 \listings{\home/Fourier.py}
 {def Fourier_example}
 {plt.show}{Fourier.py}



\listings{\home/Fourier.py}
{def Fourier_series}
{return}{Fourier.f90}

       \chapter{Object Oriented Programming} 

%Palabrejos que luego explicamos que son las 4 propiedades 



\section{Polymorphism} 







        \newpage 
        \subsection*{Fortran} 
        
        \renewcommand{\home}{./Fortran/sources/Advanced_programming/polymorphism} 
        \lstfor
        \listings{\home/polymorphism.f90}{subroutine polymorphism_example}{end subroutine}{polymorphism.f90}
        
        
        \newpage
        \subsection*{Python}
        \renewcommand{\home}{./Python/sources/Advanced_programming/polymorphism} 
        \lstpython
        \listingsp{\home/polymorphism.py}{class}{Total perimeter}{polymorphism.py}


%\newpage 
%\section{ODEs integration} 
% 
%\subsection*{Fortran}
%\renewcommand{\home}{./Fortran/sources/Advanced_programming/polymorphism} 
%\lstfor
%\listings{\home/polymorphic_ODES.f90}{def Euler}
%{polymorphic}{polymorphic_ODES.f90} 
% 
% 
%\subsection*{Python}
%\renewcommand{\home}{./Python/sources/Advanced_programming/polymorphism} 
%\lstpython
%\listingsp{\home/polymorphic_ODES.py}{def Euler}
%{polymorphic}{polymorphic_ODES.py}



       
 

\chapter{Map, filter and reduce} 

\section{Introduction}

\section{Fortran} 

\renewcommand{\home}{./Fortran/sources/Advanced_programming/functional programming} 

\listings{\home/map_filter_reduce.f90}{subroutine test_map_filter_reduce}
{end subroutine}{map_filter_reduce.f90}



\listings{\home/map_filter_reduce.f90}{elemental integer function str_to_number}
{end function}{map_filter_reduce.f90}


\newpage 
\section{Python}

\renewcommand{\home}{./Python/sources/Advanced_programming/functional programming} 



\listings{\home/map_filter_reduce.py}{def test_map_filter_reduce}
{max value}{map_filter_reduce.py}



\listings{\home/map_filter_reduce.py}{def str_to_number}
{return}{map_filter_reduce.py}



       
 
\chapter{Pointers} 

\section{Introduction}

\newpage 
\section{Fortran}
\renewcommand{\home}{./Fortran/sources/Advanced_programming/odes} 
\listings{\home/n_body_problem.f90}{function F_NBody}{end function}{n_body_problem.f90}

\newpage 
\listings{\home/n_body_problem.f90}{subroutine F_N_body_problem}
{end subroutine}{n_body_problem.f90}


\section{N body problem} 

\newpage
\section{Python}
\renewcommand{\home}{./Python/sources/Advanced_programming/odes} 
\listings{\home/n_body_problem.py}{def F_NBody_problem}{return F}{n_body_problem.py}


\newpage
\listings{\home/n_body_problem.py}{def Initial_positions_and_velocities}{-0.4}{n_body_problem.py}
\listings{\home/n_body_problem.py}{def Integrate_NBP}{plt.show}{n_body_problem.py}





       
 \chapter{Cauchy problem}
 
 \section{Introduction}
 
 
 \section{Cauchy problem solver}
 
 
 \section{Temporal schemes} 
 
 
 \section{Stability region} 
 
 
 \section{N body problem} 

       
 


\chapter{Examples using advanced techniques}
 
 
\newpage

\section{VonKoch fractal} 

\renewcommand{\home}{./Fortran/sources/Advanced_programming/miscellaneous/fractals} 
\listings{\home/VonKoch.f90}{subroutine VonKoch}
{end subroutine}{VonKoch.f90}

\newpage 
\section{Mandelbrot set} 
\listings{\home/Mandelbrot.f90}{function Mandelbrot_set}
{end function}{Mandelbrot.f90}

%
%\section{Python} 
%\subsection{VonKoch} 
%
%\subsection{Mandelbrot} 





%\chapter{Games} 

%\section{Introduction}

%\section{Fortran} 
\newpage 
\section{Sudoku game} 

\renewcommand{\home}{./Fortran/sources/Advanced_programming/miscellaneous/games} 
\listings{\home/Sudoku.f90}{recursive subroutine}
{end subroutine}{Sudoku.f90}

%\newpage
%\section{Python} 
%\subsection{Sudoku} 






 
  
   

  \part{Software development}\label{PartIV}
    \input{./doc/Chapters/Part_IV/Software_development.tex}
    
  
    
   
    
    %______________________________________________________________________________________________________
    \backmatter
    
    %%%%%%%%%%%%%%%%%%%%%%%%%%%%%%%%%%%%%%%%%%%%%%%%%%%%%%%%%%%%%%%%%%%%%%%%%%%
    %%                              REFERENCES                               %%
    %%%%%%%%%%%%%%%%%%%%%%%%%%%%%%%%%%%%%%%%%%%%%%%%%%%%%%%%%%%%%%%%%%%%%%%%%%%
    %\nocite{*}          %Para que aparezcan todas las entradas de la bibliografía aunque no hayan sido mencionadas en el texto
                        %Sin embargo, toda entrada de la bibliografía DEBE ser mencionada en el texto...Con \nocite{<key>} incluyo una concreta.
  % \vspace{-2cm}                    
    \bibliography{./doc/Chapters/References}
    \addcontentsline{toc}{chapter}{References}
    
\end{document}