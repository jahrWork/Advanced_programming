\chapter*{Advanced Programming for Numerical Calculations: Climbing Python \& Fortran}
%addcontentsline{toc}{chapter}{Introduction}
%\thispagestyle{empty}
\vspace{-2.5cm}
This book is devoted to those who like math and  use  
programming languages such as: Python or Fortran to solve numerical problems.  
The main focus of this book is to approach math language to programming languages by 
paying attention to functional programming paradigm.
Through different numerical examples, 
the book tries to mimic mathematics by showing how  mathematical concepts 
are implemented by means of functional programming paradigm. 
 

This book is divided in three parts: 
\begin{itemize}
\setlength\itemsep{0cm}
\item[] Part I. Foundations. 
\item[] Part II. Computer operations with integers and reals. 
\item[] Part III. Advanced programming. 
\end{itemize}
\vspace{-0.2cm}
These three parts can be read  independently. The first part: Foundations is devoted to 
initiate the reader in implementing  numerical problems with the computer.
The second part: Computer operations with integers and reals. 
Real or integer numbers are represented with finite storage. When oerating those numbers, 
errors associated to the loss of precision is encountered. This second part tries to get a deep 
knowledge, by following different examples, of this round-off errors.   
The third part: Advanced programming gives tries to show the most important techniques 
focalizing  mainly in the functional programming paradigm with pure functions 
emulating math environments. 


This book can not be considered a full handbook of Python or Fortran. 
It is a modest approach of challenging numerical problems by means of
functional programming in Python and Fortran. 
This book is fully supported with the following software repository: 
\vspace{-0.2cm}
\begin{center} 
\link{Software repository: jahrWork}{https://github.com/jahrWork/Advanced_programming}
\end{center} 

\newpage 
\subsection*{Fortran menu}
The repository comprises a Fortran project and Python project
implemented in Visual Studio. However, examples can be run with any other different editor or 
Integrated
Development Environment (IDE). 
Once this repository  is downloaded or cloned, the following Fortran main program 
covers different examples for the three parts of this book:   
\vspace{0.5cm}
\lstfor
\renewcommand{\home}{./Fortran/sources} 
\listings{\home/main.f90}{Program main}{end program}{main.f90}

\newpage 
\subsection*{Python menu}
The following Python main program 
covers different examples for the three parts of this book:   
\vspace{0.5cm}
\lstpython
\renewcommand{\home}{./Python/sources} 
\listingsp{\home/main.py}{from}{option not implemented}{main.py}








