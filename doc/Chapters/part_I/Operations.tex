\chapter{Basic operations} 



\section{Data types} 



Georg Cantorgave the following definition: 

A set is a gathering together into a whole of definite, distinct objects of our perception or our thought—which are called elements of the 
set.

Roster or enumeration notation defines a set by listing
its elements between curly brackets, separated by commas:
$$
A = \{ 4, 2, 1, 3 \}
$$
$$
B = \{ blue, white, red \}.
$$
In a set, all that matters is whether each element is in it or not, 
so the ordering of the elements in roster notation is irrelevant 
(in contrast, in a sequence, a tuple, or a permutation of a set, 
the ordering of the terms matters).

 
For example, {2, 4, 6} and {4, 6, 4, 2} represent the same set.

For sets with many elements, especially those following an implicit pattern,
the list of members can be abbreviated using an ellipsis '...'.
For instance, the set of the first thousand positive integers 
may be specified in roster notation as
$$
\{ 1, 2, 3, ..., 1000 \}.
$$



\newpage 
\subsection*{Fortran code}


\newpage 
\subsection*{Python code}
The same function is presented below coded with Python. Essentially both codes are quite similar, 
however, now data types for variables are not explicitly declared because 
Python automatically declares them. In addition, indentation rules must be strictly followed. 
\vspace{0.5cm} 
\lstpython
\renewcommand{\home}{./Python/sources/Foundations/data_type} 
\listingsp{\home/data_type.py}{def}{format}{data_type.py}
%\lstfor
  

Explain  3 iterators with for Python: 

for v in V: 
for i in range(len(V)): 
for i, v in enumerate(V):

\newpage 
\section{Roots of a second degree equation} 
In this section, a program to obtain the roots of a second order equation is presented:  
$$
a x^2 + b x + c = 0, \qquad \forall \ a, b, c \in \mathbb{R}.
$$
The fundamental theorem of algebra states that every  Nth order polynomial has N complex roots. 
If the coefficients are real, then the roots are complex conjugate.
Dividing the above equation by $ a $ an looking for a perfect square, 
the following equation is obtained: 
$$
\left( x + \frac{b}{2a} \right)^2 - \frac{b^2 }{ 4 a^2} + \frac{c}{a} = 0. 
$$
Solving the unknown $x$, the well known formula for the roots is obtained: 
\begin{equation}
 x_{1,2} = \frac{ - b \pm \sqrt{ b^2 - 4 a c }  }{ 2 a  }  
 \label{x12}
\end{equation} 
If the discriminant $ d = b^2 - 4 a c $ is less than zero, roots become complex. 
In the following code, complex solutions given by (\ref{x12}) are implemented. 
Note that the discriminant $ d $ was defined as a complex variable to avoid math problems 
when the discriminant is negative. Whereas the root of a real negative number is not defined,
the root of a negative number has a value for complex numbers. 
 
\vspace{0.5cm}
\lstfor
\listings{\home/Roots.f90}{subroutine Roots_2th}{end subroutine}{Roots.f90}


\newpage 
\subsection*{Python code}
The same function is presented below coded with Python. Essentially both codes are quite similar, 
however, now data types for variables are not explicitly declared because 
Python automatically declares them. In addition, indentation rules must be strictly followed. 
\vspace{0.5cm} 
\lstpython
\renewcommand{\home}{./Python/sources/Foundations/Roots} 
\listingsp{\home/Roots.py}{from}{output}{Roots.py}
%\lstfor
  
  
 