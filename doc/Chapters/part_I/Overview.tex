\chapter{Python and Fortran foundations through examples} 
 
As a first approach to the use of Fortran to learn applied mathematics 
a chapter including basic operations will be presented. 
In it the reader can become familiar to the use of basic sentences in order to 
perform simple mathematical operations.

For each concept treated a Fortran code example guides the explanation and at the 
end of each chapter the corresponding Python codes are presented. Thus, the following
can be also used as a comparison between the foundations of both programming languages. 
 
The following menu schematizes the concepts treated in the following chapters, by choosing 
different options you can execute all the examples presented. 
 
    \newpage 
    \vspace{0.5cm}
    \renewcommand{\home}{./Fortran/sources/Foundations} 
  \listings{\home/Foundations.f90}{do while}{end select}{Foundations.f90}
   
    \newpage
    \vspace{0.5cm}
    \renewcommand{\home}{./Python/sources/Foundations} 
  \listings{\home/Foundations.py}{while}{Test_load_matrix()}{Foundations.py}  
   
 
 