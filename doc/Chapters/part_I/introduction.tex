
\chapter*{Programming menu for Foundations}
    %\newpage 
    \subsection*{Fortran menu}
    %\vspace{-0.3cm}
The following Fortran menu covers the different examples that are developed throughout the first part of this book.
By introducing a number on this menu you will execute a bunch of subroutines and functions. 
Notice that option \texttt{1} can be used to check that the programs are working properly in your computer, it just executes the well-known \texttt{Hello world} example. 
The option \texttt{2} includes all the programs treated in the chapter \ref{chap:basicop}, 
option \texttt{3} has the codes of chapter \ref{chap:impdec},
options \texttt{4} and \texttt{5} show the codes of chapter \ref{chap:matrices},
options \texttt{6} and \texttt{7} show the programs of chapter \ref{chap:opfuncs} and finally,
option \texttt{8} includes the code explained in chapter \ref{chap:readwrite}.
%\vspace{-0.3cm}

\newpage
\renewcommand{\home}{./Fortran/sources/Foundations} 
\lstfor
\listings{\home/Foundations.f90}{Basic programming}{end select}{Foundations.f90}
   
   
    \newpage
    \subsection*{Python menu}   
The following Python menu covers the numerical examples developed throughout the first part of this book in the same order that the previous menu. 
\vspace{0.5cm}
\renewcommand{\home}{./Python/sources/Foundations} 
\lstpython
\listingsp{\home/Foundations.py}{while}{Test_load_matrix()}{Foundations.py}  
   
 
 