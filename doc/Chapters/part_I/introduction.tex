
\chapter*{Programming menu for Foundations}
\vspace{-2.5cm}
%Parrafo de intro a Python y Fortran
Python and Fortran are both high-level programming languages. 
Certainly, Fortran was the first high-level programming language and it was invented by John Backus for IBM in 1954.
Python, developed by Guido van Rossum and first released in 1991, is a general-purpose programming language
meaning that it is not specialized for any specific kind of problem.
Python is usually used for data analytics, Artificial Intelligence and machine learning, to build websites or for data visualization among others. 
However, the language together with its standard library is not prepared for numerical computing.
Hence, mainly four libraries are used to extend its capabilities: \texttt{numpy}, \texttt{scipy}, \texttt{matplotlib} and \texttt{pandas}. 
Fortran, which stands for \textit{Formula Translation}, was specifically created for scientific computing.
Throughout the successive standards developed in its history, all the features needed to perform scientific computing are built in. 
From a general point of view, Fortran is a statically compiled language, strongly typed, natively parallel and it was projected for high performance computing.
On the contrary, Python is an interpreted, dynamically-typed and weakly-typed language.

The following chapters give a first approach to implement numerical 
problems in the computer.
For each concept treated a Fortran code example guides the explanation and at the 
end of the chapter the corresponding Python codes are presented. Thus, the following
can be also used as a comparison between the foundations of both programming languages.  


    \vspace{-0.3cm}
    \subsection*{Fortran menu}
    \vspace{-0.3cm}
The following Fortran menu covers the different examples that are developed throughout the first part of this book.
By introducing a number on this menu you will execute a bunch of subroutines and functions. 
Notice that option \texttt{1} can be used to check that the programs are working properly in your computer, it just executes the well-known \texttt{Hello world} example. 
The option \texttt{2} includes all the programs treated in the chapter \ref{chap:basicop}, 
option \texttt{3} has the codes of chapter \ref{chap:impdec},
options \texttt{4} and \texttt{5} show the codes of chapter \ref{chap:matrices},
options \texttt{6} and \texttt{7} show the programs of chapter \ref{chap:opfuncs} and finally,
option \texttt{8} includes the code explained in chapter \ref{chap:readwrite}.
%\vspace{-0.3cm}

\newpage
\renewcommand{\home}{./Fortran/sources/Foundations} 
\lstfor
\listings{\home/Foundations.f90}{Basic programming}{end select}{Foundations.f90}
   
   
    \newpage
    \subsection*{Python menu}   
The following Python menu covers the numerical examples developed throughout the first part of this book in the same order that the previous menu. 
\vspace{0.5cm}
\renewcommand{\home}{./Python/sources/Foundations} 
\lstpython
\listingsp{\home/Foundations.py}{while}{Test_load_matrix()}{Foundations.py}  
   
 
 