%--------------------------------------------------------------------------------------------------------------------------------------
 \renewcommand\home{./Fortran/sources/IEEE}

\chapter{Integers representation} 

    \addcontentsline{toc}{section}{Overview}
    \section*{Overview} 
    
It is well known that the programs are written with numbers in decimal form. However, the computer, 
in order to work with these constants and variables, is going to convert numbers to binary form. 
Using two voltage levels (representing 0`s and 1`s) as unique states the computer can perform calculations that, 
in order to show to the user the results obtained, will be translated again to decimal form. 

The computer translates decimal values to binary similarly to the way humans typically do (with some exceptions to be noticed). For 
positive integers the way to translate a value uses the principles of positional numeral systems, the value of each digit is totally determined by the 
digit itself, the position in the number and the base of the numeral system (10 and 2 for decimal and binary). In the case of negative integer values, 
while a person could write a sign (-) before the number, the machine uses only 0`s and 1`s and how this is done is treated in the chapter.

On one hand, not all the numbers that can be written in a program (actually, 
much less numbers than the infinite possibilities) are exactly represented 
in the binary translation that the computer performs. It is
essential for a programmer to understand this issue.
On the other hand, integers do have an exact representation in binary form 
with an integer variable  
but its maximum and minimum value depends on the memory size of this variable.  
The main problematic that motivates the study of the integer representation in 
computers is related to overflow, values out of the representable range. 

The following topics are covered in this chapter: 

\begin{enumerate} 
\setlength\itemsep{-0.1cm}
    \item Integer overflow example.
    \item Two's complement integer representation.
    \item Understanding integer overflows.
    \item Overflow of constants.
    \item Overflow by incorrect assignment.
    \item Declaring kind
    \item Two's complement converter
\end{enumerate} 
    
Throughout this chapter the reader will be able to understand and justify the corrupted values stored in a variable when an out-of-range integer is assigned by mistake. When this error occurs during compilation time, in many cases, the compiler does not abort the compilation, it only prints a warning. In the worst situation, the error appears during execution, the program is not working properly and this source of error could be a nightmare to trace.
 
 
 
   
\newpage
    \section{Integer overflow example}


The range of integer variables is limited by its memory size representation. 
When programming and by mistake is possible to assign a value out of the
range of some integer variable.
It is very important to understand how the programming language treats this exception 
to understand what is going on in our execution. 
In the following example, the integer variable \texttt{i} is stored in 1-byte 
 of memory 
whereas  \texttt{k} is stored in 2-bytes.

\vspace{0.5cm}

\listings{\home/Integer_representation.f90}{subroutine Integer_overflow}
{end subroutine}{Example of Integer overflow. Integer_representation.f90}



A single loop prints on screen 
the successive values of \texttt{k} and \texttt{i}. Once the value +127 
of \texttt{i} is reached, negative numbers of \texttt{i} appear even when we are
incrementing step by step the variable \texttt{k}.  
The successive  values of \texttt{i} after $+127$ are $-128, -127,  \dots$ 

To understand this fact, we should consider that integer numbers of 1-byte size 
are treated as numbers 
modulo $ 2^8 $ which means that if integer numbers were unsigned, their range would be 
from 0 to 255. 
If this unsigned integer variable were assigned with a number \texttt{k}
greater than 255, the resulting value would be k mod 255 (remainder of k divided by 255). 
Additionally, 
to hold signed integer variables (positive and negative values), 
a two's complement notation is generally used 
that restrict the integer values from -128 to +127 in case of 1-byte size variables. 
 



Another classical example that disconcerts to the novel programmer is the overflow with classical 
mathematical operations such as the  factorial of an integer value: 
$$
 {\displaystyle n \ !=n\cdot (n-1)\cdot (n-2)\cdot (n-3)\cdot \cdots \cdot 3\cdot 2\cdot 1\ }  
$$
In the following code, 
the factorial of 20 is obtained by multiplying the integer variable \texttt{f} 
sequentially from 1 to 20. The partial results are shown on screen. After the variable \texttt{k} 
has reached 13, the variable \texttt{f} shows wrong results. 
\vspace{0.5cm}
\listings{\home/Integer_representation.f90}{subroutine Factorial_overflow}
{end subroutine}{Factorial overflow. Integer_representation.f90}
Once the result has reached the maximum value the integer variable \texttt{f} can hold, even negative 
numbers appears being the result of the modular arithmetic with two's complement. 
It is important to notice, that in this case, the integer variable \texttt{f} is stored 
in 4 -bytes which is the default storage of an integer value. 


When an integer overflow occurs, the execution is not aborted 
and the programmer is not alerted of the malfunction. 
This is the main problems of integer overflow occurrences.  
Generally, and integer overflow is associated to the following situations: 
\begin{enumerate}
\setlength\itemsep{-0.1cm}
\item Counters to measure time or repetitive processes. 
     If these counters are used in a perpetual control system, they are  eventually overflowed.
\item Mathematical operations in the set of integers $ \mathbb{Z} $ such as the factorial of a number. 
\item Careless assignments between reals and integers.
\end{enumerate}

When program codes are used as controlling units in embedding systems, 
the overflow could origin terrible accidents such as the crash of the 1996 maiden 
flight of the Ariane 5 rocket. 

On May 2015, the European Aviation Safety Agency discovered that the embedded 
software of the Boeing 787 suffered after 248 days an integer overflow  of a  counter of 
a signed variable of 32-bits. 
This overflow required to reset periodically the system to avoid 
loss of electrical power and ram air turbine deployment. 



The two's complement integer representation is explained in the next section. 

 
\newpage
    \section{Two's complement integer representation} \label{sec:TwosComp}

An integer variable declared as 4-bytes (32 bits) size has $ 2^{32} $ possible different values. 
Some languages allow to work with unsigned integers. In that case, 
all bits are reserved for positive values and its range goes from 0 to $ 2^N - 1$. 
When specifying signed integers, one bit has to be reserved for the sign and then the range is divided by two.
Hence, an integer variable stored in 4-bytes has the range  [-2 147 483 648, 2 147 483 647]. 
The Table \ref{tab:propertiesint} summarizes the main parameters for signed integers.

\begin{table}[h]
    \centering
    \begin{tabular}{| r | c | c |}
        
        \hline
        Name  & Minimum ($-2^{N-1}$) & Maximum ($2^{N-1} - 1$)  \\ \hline
        
        \texttt{ Kind = 1}                         & $-128$ & $127$ \\ \hline
        \texttt{ Kind = 2}                      & $-32 768$ & $32 767$ \\ \hline
        \texttt{ Kind = 4}               & $-2 147 483 648$ & $2 147 483 647$ \\ \hline
        \texttt{ Kind = 8}   & $-9 223 372 036 854 775 808$ & $9 223 372 036 854 775 807$ \\ \hline
        
    \end{tabular}                                                       
    \caption{Main properties of the different signed integer kinds used (two's complement used for negative values).}
    \label{tab:propertiesint}
\end{table}

In general, the range of an integer variable stored in N bits is:   
\begin{equation}
    [ -2^{N-1},\: 2^{N-1}-1 ].  
\end{equation}



The Table \ref{bits} represents an integer storage with N-bits where $ b_i $ denotes the possible values 0 or 1. 

\begin{table}[h]
    \centering
    \begin{tabular}{| c | c | c | c | c | c | c | c |}
        \hline
        $ b_{N-1} $  & $ b_{N-2} $ &  $ b_{N-3} $ &\ldots  & \ldots & $ b_{2} $  & $ b_{1} $ & $ b_{0} $ \\ \hline 
    \end{tabular}                                                       
    \caption{Binary representation of an integer number stored in N bits.}
    \label{bits}
\end{table}

The general expression to convert this binary number to decimal when it is expressed in two's complement is given by: 
\begin{equation} 
 x = - b_{N-1} \ 2 ^{ N-1 } \ + \ \sum_{i=0} ^{ N-2} \ b_i \ 2^i. 
 \label{reconstruction}
\end{equation} 

As an example, let's consider an integer value equal to $425_{10}$ (subscripts $_{10}$ and $_2$ means decimal and binary value respectively) stored in 2-bytes of memory.
The binary form comprises 16 different bits to hold the integer number. As a signed positive integer 
it is represented with common binary numeral system: 
\begin{multline*}
    \texttt{ 0000000110101001}_{2} = 1\cdot 2^8 +1\cdot 2^7+ 0\cdot 2^6+ 1\cdot 2^5+ 0\cdot 2^4+ 1\cdot 2^3+  0\cdot 2^2+ \\ + 0\cdot 2^1+ 1\cdot 
    2^0 = 2^8 + 2^7 + 2^5 + 2^3 + 2 = 425_{10}
\end{multline*}
where the leading zeroes are not displayed when printing that number on your screen. 
The same number stored in a variable of 4-bytes (32bits) of memory size
has the same binary representation but with more leading zeroes. 
Notice that a signed integer variable of 1-byte size can not store this number 
since its range is  $[-128, 127]$. 




The following two questions arise:
\begin{enumerate} 
\setlength\itemsep{0cm}
\item How are negative integer numbers stored in the computer? 
\item How is overflow treated with integer numbers? 
\end{enumerate} 

To answer these two questions, we should understand how signed integer  
numbers are stored in the computer. 
There are two main ways to store signed integer variables in memory: 

\begin{enumerate}
\setlength\itemsep{0cm}
\item One's complement integer representation. 
\item Two's complement integer representation.
\end{enumerate}

In the following table, a signed integer variable of 3-bits of memory size is represented in binary form, 
unsigned integer, two's complement and one's complement. 

\begin{center} 
    \begin{tabular}{|c|c|c|c|}
        \hline
        Binary  & unsigned & two's comp.& one's comp.  \\
        \hline
        \texttt{000} & 0 & 0 & 0 \\
        \hline
        \texttt{001} &  1& 1 & 1 \\
        \hline
        \texttt{010} & 2 & 2 & 2 \\
        \hline
        \texttt{011} & 3 & 3 & 3 \\
        \hline
        \texttt{100} & 4 & -4 & -3 \\
        \hline
        \texttt{101} &  5 & -3 & -2 \\
        \hline
        \texttt{110} &  6 & -2 & -1 \\
        \hline
        \texttt{111} &  7 & -1 & -0 \\
        \hline
    \end{tabular}
\end{center} 


The two's complement notation is, generally, used in computers. 
If the leftmost bit is \texttt{1}, the integer variable holds a negative number.
From the above table, a procedure is deduced to obtain the binary form of a negative number. 
For example, the binary form of -2 is obtained from the binary form of 2 which is \texttt{010}. Then, 
these bits are inverted to give \texttt{101}. An finally, \texttt{001} is added to the resulting number to give: 
\texttt{110} which is -2 in the two's complement.

\begin{IN}
   To obtain the binary form in two's complement of a negative number, carry out the following steps: 
   \begin{enumerate}
       \item Obtain the binary form of the integer number without sign. 
       \item Invert all bits by using the bitwise NOT operation. 
       \item Add 1 to the resulting value. 
   \end{enumerate}
\end{IN}

The signed negative number $-425_{10}$, using two's complement, is translated in the machine like:

\begin{equation}
    \texttt{1111111001010111}_{2} = -425_{10}
\end{equation}

Notice that this technique to encode signed numbers defines the 'two's complement' of a specific number 
with N bits as its complement with respect to $2^N$ which means that adding a number to its complement 
results in $2^N$. From the different ways to represent signed numbers, this method performs addition, subtraction, and multiplication 
in the same way for signed and unsigned binary numbers and it lacks from the negative zero. 




Once the treatment of negative numbers is understood, the next step is to understand how the computer 
treats an overflow with integers.  
The most common result of an overflow is that the least significant representable digits 
of the result are stored. Imagine that integer value 5 is stored in a 4-bits x-variable:
$$
       x = 0101.
$$
If the content of this variable is assigned to a y-variable of 3-bits (which can only store the range [-4,3]),
only the least significant bits are stored and the 
resulting variable will store the number
$$
       y = 101,
$$
which represents the value -3 in a two's complement representation with 3 bits. The original value of x has value 5.  
Since the overflow of an integer variable never shows as an error, 
it is very important to avoid overflows of integer variables not to have mysterious 
programming problems very difficult to be identified. 


\begin{IN}
Overflow of integer variables are not usually identified as explicit errors and they 
can originate strange behaviors very difficult to discover in a programming code. 
\end{IN}




    \newpage 
    \section{Understanding integer overflows}  \label{sec:UndersIntOver}

Now, the examples at the beginning of this chapter are easily interpreted. In the first case, the 
values $+128, +129, \dots$ stored in \texttt{k} (2-bytes memory) size are assigned to the variable 
\texttt{i} with 1-byte memory size.
Then, the binary forms of $+128_{2-bytes}$ or $+129_{2-bytes}$:

\begin{equation*}
    +128_{2-bytes} = \texttt{00000000 10000000} ,\:  +129_{2-bytes} = \texttt{00000000 10000001}
\end{equation*}

are stored keeping the 8 least significant bits like:

\begin{equation*}
    \texttt{10000000} ,\:  \texttt{10000001}
\end{equation*}

which are the binary two's complement form of $-128_{1-byte}$ and $-127_{1-byte}$ respectively. 

In the second example the variable \texttt{f} does not overflow in one or two units but in a much 
higher quantity. 
The real result of $13!$ is $6227020800$ whose binary representation with enough memory size to store 
it (40 bits for example) is: 

\begin{equation*}
    6227020800_{40-bits} = \texttt{00000001 01110011 00101000 11001100 00000000} 
\end{equation*}

Once this variable is stored with the 32 least significant bits the result is:

\begin{equation*}
    \texttt{01110011 00101000 11001100 00000000} 
\end{equation*}

which is the binary two's complement representation of $+1932053504_{4-bytes}$, the result printed by the computer. 
From that value, all the following results of the factorial are wrong because of the overflow of the 4-bytes variable. 




In the following example, 
an integer variable of 1-byte size is overflowed with the value 130 and with 
-130. The resulting values are compared with an integer variable of 2-bytes size.  
The internal representation  in binary form according to two's complement is also shown. 
The following subroutine determines the internal binary form of an integer of any kind. 
Depending on the \texttt{kind\_type} of the integer variable, 
the maximum and minimum range is calculated by means of the subroutine \texttt{integer\_parameters}.
Later, the integer variable \texttt{xr} is written in binary form by means of a \texttt{write} sentence. 
And finally, the integer variable \texttt{xr} is reconstructed by means of equation  (\ref{reconstruction})
to show on screen and to check the binary form representation. 

\vspace{0.5cm}
\listings{\home/Integer_representation.f90}{subroutine Integer_representation_twos_complement}
{end subroutine}{Binary form representation. Integer_representation.f90}


To understand the binary internal representation of an integer overflow, 
the following lines gives the results: 
\begin{verbatim} 
call Integer_representation_twos_complement("kind=1",  130)
call Integer_representation_twos_complement("kind=2",  130)
call Integer_representation_twos_complement("kind=1", -130)
call Integer_representation_twos_complement("kind=2", -130)
\end{verbatim} 

\begin{verbatim} 
----------------------------------------------------------------
 Internal representation of x=130  with x integer  kind=1
    Integer  kind=8            x =                  130
    Same integer kind=1        x =                 -126
    Range integer kind=1         =                 -128      127
    Internal representation of x = 10000010
    Reconstruction sum b_i 2**k  =                 -126
----------------------------------------------------------------
----------------------------------------------------------------
 Internal representation of x=130  with x integer  kind=2
    Integer  kind=8            x =                  130
    Same integer kind=2        x =                  130
    Range integer kind=2         =               -32768    32767
    Internal representation of x = 0000000010000010
    Reconstruction sum b_i 2**k  =                  130
---------------------------------------------------------------
---------------------------------------------------------------
 Internal representation of x=-130  with x integer  kind=1
    Integer  kind=8            x =                 -130
    Same integer kind=1        x =                  126
    Range integer kind=1         =                 -128      127
    Internal representation of x = 01111110
    Reconstruction sum b_i 2**k  =                  126
----------------------------------------------------------------
----------------------------------------------------------------
 Internal representation of x=-130  with x integer kind=2
    Integer  kind=8            x =                 -130
    Same integer kind=2        x =                 -130
    Range integer kind=2         =               -3276     32767
    Internal representation of x = 1111111101111110
    Reconstruction sum b_i 2**k  =                 -130
----------------------------------------------------------------
\end{verbatim} 
It's shown that the value \texttt{130} overflows an integer variable of \texttt.
When this happens, no alert is shown by the compiler and non desirable results could destroy 
the performance of our programming code.    



    \newpage
    \section{Overflow of constants}
        
The issues and limits treated above also appear when, instead of integer variables, the program works 
with integer constants. 
The assignment of a constant which is out range of a \texttt integer variable 
will give rise to a compilation error.

In the following example an integer variable of size 8 bytes is assigned with a huge number: 
\begin{verbatim}
    integer (kind = 8) :: P
   
    !Example variable assignation out-of-range 8 bytes
    P = 9223372036854775808
\end{verbatim}
When trying to compile the above program, the compiler returns the following message: 
\begin{verbatim}
    Compilation Aborted (code 1)
    error #6901: The decimal constant was too large when 
        converting to an integer, and overflow occurred.
        [9223372036854775808]
\end{verbatim}
The compiler tries to store the number \texttt{9223372036854775808}, which is equal to \texttt{huge(P)}+1,
in an internal register of 8-bytes with two's complement representation. Since the constant is greater than 
the maximum allowed value of the maximum memory size available in the computer the compiler gives the above error. 


\newpage
    \section{Overflow by incorrect assignment}

If valid huge constants are assigned to lower size integer variables, only compiler warnings could alert 
of possible malfunctions.  
  
In the following example, an integer variable \texttt{Q} of 4-bytes is considered. 
The value \texttt{2147483648}, which is equal to \texttt{huge(Q)} +1, is assigned 
to \texttt{Q}. 
\vspace{0.5cm}
\listings{\home/Integer_representation.f90}{subroutine Assignment_overflow}
{end subroutine}{Assignment overflow. Integer_representation.f90}
When compiling the above program, no errors are obtained but the following warning appears: 
\begin{verbatim}
    warning #6384: The INTEGER(KIND=4) value is out-of-range.	
\end{verbatim}
Hence, the program  could be executed without taking care of warning messages giving the following result: 

\begin{verbatim}

   Intention: assign Q <- 2147483648
   Result:  Q = -2147483648 huge(Q) =   2147483647
\end{verbatim}
If warnings are not taken into account, non desirable results could be  obtained.  
In this case, variable \texttt{Q} stores the value  \texttt{-2147483648} and the 
programmer by an improper assignment tried to store the value \texttt{2147483648} which is greater 
the maximum value that \texttt{Q} can store.




    \section{Declaring kind}

A good practice is to write codes that could be used with different integer sizes: \texttt{kind=1, 2, 4, 8}. 
This can be done by not specifying the kind type of any integer variable. 
In this case, the compiler assumes default integer kind for all integer variables. 
Generally, this default can be easily changed  with a proper compilation option. 


The opposite approach is to declare specifically the kind type of any integer variable by means of 
sentences such as: 
 \begin{verbatim}
     integer (kind=1) :: i1
     integer (kind=2) :: i2
     integer (kind=4) :: i4
     integer (kind=8) :: i8
 \end{verbatim}
When declaring constants, the situation is different. A specific kind type is declared  by underscoring the 
constant with its kind type: 
 \begin{verbatim}
    integer (kind=8) :: i 
    
     i = 123_1 
     i = 123_2
     i = 123_4 
     i = 123_8
 \end{verbatim}
In the above example, the constant \texttt{123} is stored in a register of 1, 2, 4, or 8 bytes. Later, it is 
assigned to an integer variable of 8 bytes. 
When no kind type specifier is selected for integer constants (\texttt{i = 123}), 
then the compiler reserves an internal appropriate register of variable size
depending on the value of the constant. 

\begin{IN}
    A program can be independent of declared precisions so the following compiler option rules the behaviour of all integers:
    
    Default Integer KIND: $\texttt{/integer-size:{16\mid32\mid64}}$  
\end{IN}


%
%The compiler defines the memory size for each \textbf{constant} based on some rules:
%
%\begin{itemize}
%    \item If a kind parameter is specified, the integer has the kind specified. 
%    \item If a kind parameter is not specified, integer constants are interpreted as follows:
%        \subitem If the integer constant is within the default integer kind range, the kind is default integer (compiler options).
%        \subitem If the integer constant is outside the default integer kind range, the kind of the integer constant is the smallest integer 
%kind that holds the constant. 
%\end{itemize}
%
%According to \citet{IntelIntegers} the default integer size is affected by the compiler option \texttt{integer-size}, the \texttt{INTEGER} 
%compiler directive, and the \texttt{OPTIONS} statement. These configurations allow the user to change easily the execution of the code so, 
%for example, the algorithm can be checked for simple and double precision independently. 
%
%The memory size reserved for \textbf{variables} is slightly different:
%
%The first rule is also applied, the kind can be defined in the declaration. However, if it is not specified, the compiler is not permissive 
%and does not changes it automatically if finds that an out-of-range value is stored in a variable sized with default integer value.
% 
%    The compiler do not decide automatically, just conforms to what is defined by the programmer. Notice that every code file admits some 
%    compiler options where the default integer kind for variables and constants can be defined 
%    (option \texttt{integer-size}).
%    
    
\begin{IN}
    \begin{itemize} 
    \item  \textbf{Constants.}
    The compiler reserves memory size depending on the magnitude of the constant. 
    \item  \textbf{Variables.}
    The memory size of any variable is specified by its kind type. 
    If no kind type is specifically declared, the default integer kind is considered by the compiler. 
   
    \end{itemize} 
\end{IN}

%
%fixes automatically the size reserved for all constants and variables based 
%on the compiler options (which can be easily changed). 
%
%This allows you to run the program with different configurations depending on the needs of the moment. 
%
%However, codes are written with different purposes, choose the right option for each situation. 
%If, despite this, you still want to declare the kind for a specific variable just 
%include the value in the declaration by writing:
%
%\texttt{integer(kind = k) ::} where k is the bytes reserved: 1, 2, 4 or 8.
%
%%\texttt{integer(k) ::} or \texttt{integer*k ::}
%




%%OLD MATERIAL
%\newpage
%\listings{\home/Round_off.f90}{For CONSTANTS:}
%{--END OF EXAMPLE--}{Second example of Integer_representation - Round_off.f90} 
%
%The previous example shows a couple of Warnings that must be considered when managing with integers. The first one makes reference to the assignation of the value $130$ to a 1 byte integer constant, The second warning says the same for the variable Q, which is also out-of-range since it is declared as \texttt{kind = 4}. Notice that none of them interrupts the compilation of the code so if the warnings are not considered, the program will be executed with a different behaviour as expected. 
%
%\begin{verbatim}
%    warning #6047: The BYTE / LOGICAL(KIND=1) / INTEGER(KIND=1)
%    value is out-of-range.   [130]
%    
%    warning #6384: The INTEGER(KIND=4) value is out-of-range.
%\end{verbatim}


%\begin{itemize}
%    \item For a \textbf{constant} write the bytes reserved after the value preceded by an underscore (\texttt{n[\_k]}). 
%    \item For a \textbf{variable}, just include the kind in the declaration by writing \texttt{integer(k) ::}, \texttt{integer(kind = k) ::} or \texttt{integer*k ::} (where k is the bytes reserved; 1, 2, 4 or 8).
%\end{itemize}

%\newpage
%%Casos especiales
%Finally, take a look at these two especial cases treated by the compiler. 
%\begin{enumerate}
%    \item In the assignment of a lesser memory size constant to a higher size variable the variable imposes its size and the constant will be transformed to the higher value. 
%    \item If a corrupted constant (for example, \texttt{130\_1}) is assigned to a proper sized variable (for example \texttt{integer*2 :: R}), the compiler warns first about the out-of-range constant but makes the assignment properly assuming that the value expected for \texttt{R} is \texttt{130} and not the corrupted value.
%\end{enumerate}
%
%\listings{\home/Round_off.f90}{!1. R is 8}
%{--END OF EXAMPLE--}{Third example of Integer_representation - Round_off.f90}



    \section{Two's complement converter} 

A binary-to-decimal and decimal-to-binary converters are included in the codes that accompany this book.
A third function is also included, it checks if an integer value is not properly stored 
with an specific number of bits because of an overflow error. 
The converters will return the two's complement conversion for the introduced value, whether is an
integer value or a string characters of 0's and 1's. The specifications of each function are presented now.


        \subsection{\texttt{TwosCompl\_converter\_to\_decimal} function}

Returns the decimal integer value for the string of binary characters introduced according to the two's complement encoding.

Notice that the two's complement encoding is always referred to an specific number of bits which means that it make 
sense in a pre-defined memory size. Due to that, the conversion is performed according to the total length of bits 
introduced in the string, if your binary expression has leading zeros do not forget to include them. In addition, 
blank spaces are also treated as zeros so the string \texttt{' 01101'} is the same as \texttt{'001101'}.

If more than 64 bits or wrong characters are introduced the result will be automatically \texttt{0}. 
The table \ref{tab:specs1} shows the specifications of the input arguments of the converter.

    \texttt{TwosCompl\_converter\_to\_decimal( bits )}

\begin{table}[h]
    \centering
    \begin{tabular}{| c | c | c |}
        
        \hline
        Name       & Type                        & Limits  \\ \hline
        
        \begin{tabular}{@{}c@{}} \texttt{bits}  \\ (input)  \end{tabular}   & \texttt{character}    &  \begin{tabular}{@{}c@{}} Up to 64 characters with 0's, 1's  \\ or blank spaces (treated as 0's) \end{tabular}   \\ \hline
        
        Result   & \texttt{integer(kind=8)}   &  $[-9 223 372 036 854 775 808, 9 223 372 036 854 775 807]$  \\ \hline
       
        
    \end{tabular}                                                       
    \caption{Specifications of the \texttt{TwosCompl\_converter\_to\_decimal} function.}
    \label{tab:specs1}
\end{table}

Example:
\vspace{-0.5cm}
\begin{verbatim} 
write(*,*) "Two's complement conversion of binary: "
write(*,*) "010101010110111001110  =  "
write(*,*) TwosCompl_converter_to_decimal( '010101010110111001110' )
\end{verbatim} 
\vspace{-0.6cm}
\begin{verbatim} 
Two's complement conversion of binary:
010101010110111001110  =
                          699854
\end{verbatim} 

%Example 2:
%
%\begin{verbatim} 
%write(*,*) "Two's complement conversion of binary: "
%write(*,*) "110010110  =  "
%write(*,*) TwosCompl_converter_to_decimal( '110010110' )
%\end{verbatim}
%
%\begin{verbatim} 
%Two's complement conversion of binary:
%110010110  =
%                         -106
%\end{verbatim}
 

        \subsection{\texttt{TwosCompl\_converter\_to\_binary} function}

Returns the minimum length binary string that can store the decimal integer introduced according to the two's complement encoding.

Notice that the two's complement encoding is always referred to an specific number of bits which means that it make 
sense in a pre-defined memory size. This result is printed with the minimum bits that can store the number. 

Out-of-range \texttt integer values will not be compiled (see table \ref{tab:specs2}). In addition, take into account that the algorithm used for this converter is the 
one described in the section \ref{sec:TwosComp}. This algorithm will not return a proper value for the case \texttt{x = - huge(P) - 1} where \texttt{P} is a \texttt
variable. While that number can be perfectly stored in the computer, its absolute value can not:

\begin{equation*}
    \left|x\right| = \left|- huge(P) - 1\right|= huge(P) + 1
\end{equation*}

    \texttt{TwosCompl\_converter\_to\_binary( x )}

\begin{table}[h]
    \centering
    \begin{tabular}{| c | c | c |}
        
        \hline
        Name       & Type                        & Limits  \\ \hline
        
        \begin{tabular}{@{}c@{}} \texttt{x}  \\ (input)  \end{tabular}   & \begin{tabular}{@{}c@{}} \texttt{integer}  \\ (all kinds)  \end{tabular}  &  $[-9 223 372 036 854 775 807, 9 223 372 036 854 775 807]$  \\ \hline
        
        Result   & \texttt{character}  &  Up to 64 binary characters  \\ \hline
        
        
    \end{tabular}                                                       
    \caption{Specifications of the \texttt{TwosCompl\_converter\_to\_binary} function.}
    \label{tab:specs2}
\end{table}

Example 1:

\begin{verbatim} 
write(*,*) "Two's complement conversion of integer: ", -89544563
write(*,*) TwosCompl_converter_to_binary( -89544563 )
\end{verbatim} 

\begin{verbatim} 
Two's complement conversion of integer:    -89544563
1010101010011010100010001101
\end{verbatim} 

Example 2: This example shows that the number \texttt{1} must be represented with at least two bits. 
The following section explains this result. 

\begin{verbatim} 
write(*,*) "Two's complement conversion of integer: ", 1
write(*,*) TwosCompl_converter_to_binary( 1 )
\end{verbatim} 

\begin{verbatim} 
Two's complement conversion of integer:            1
01
\end{verbatim}



        \subsection{\texttt{Check\_overflowed\_int} function}

For an integer introduced it returns the actual decimal integer stored in the computer in a specific memory 
size according to the two's complement encoding. If the number is stored in more (or equal) binary digits than
the minimum length needed, the result is the same as the input.

Notice that an out-of-range \texttt integer value will not be compiled (see table \ref{tab:specs3}). 

    \texttt{Check\_overflowed\_int( int, n\_bits )}

\begin{table}[h]
    \centering
    \begin{tabular}{| c | c | c |}
        
        \hline
        Name       & Type                        & Limits  \\ \hline
        
        \begin{tabular}{@{}c@{}} \texttt{int}  \\ (input)  \end{tabular}   & \begin{tabular}{@{}c@{}} \texttt{integer}  \\ (all kinds)  \end{tabular}  &  $[-9 223 372 036 854 775 808, 9 223 372 036 854 775 807]$  \\ \hline
        
        \begin{tabular}{@{}c@{}} \texttt{n\_bits}  \\ (input)  \end{tabular}   & \begin{tabular}{@{}c@{}} \texttt{integer}  \\ (default kind)  \end{tabular}  &  $[1, 64]$  \\ \hline
        
        Result   & \texttt{integer(kind=8)}  &  $[-9 223 372 036 854 775 808, 9 223 372 036 854 775 807]$  \\ \hline
        
    \end{tabular}                                                       
    \caption{Specifications of the \texttt{Check\_overflowed\_int} function.}
    \label{tab:specs3}
\end{table}

A non-intuitive result is obtained from the experiment of representing the decimal integer \texttt{1} with just one bit. 
In a pure binary conversion (unsigned integer for example) it is clear that the number \texttt{1} is represented through 
the binary digit \texttt{1}. However, it can be checked that the two's complement conversion of the number \texttt{1} can only be 
performed with 2 or more binary digits (\texttt{01}), being the result of the following line \texttt{-1}.

\begin{verbatim}
    Check_overflowed_int( 1, 1 )
\end{verbatim}


Example:

\begin{verbatim} 
write(*,*) "Check overflow integer of 13! = ", 6227020800
write(*,*) "with 4-bytes memory size = ", 32, "bits"
write(*,*) "The actual value stored is: "
write(*,*) Check_overflowed_int( 6227020800, 32 )
\end{verbatim} 

\begin{verbatim} 
Check overflow integer of 13! =             6227020800
with 4-bytes memory size =           32 bits
The actual value stored is:
1932053504
\end{verbatim} 




%----------------------------OLD MATERIAL-----------------------
%
%A binary-to-decimal and decimal-to-binary converter is included in the codes that accompany this book.
%In order to perform a conversion specify the value to convert, whether is a binary or a decimal value, 
%and the number of memory size bits reserved to store the result. The two's complement conversion for that number of bits 
%is shown in the screen together with some useful information about the conversion.
%
%\texttt{Twos\_complement\_converter} subroutine:
%
%Displays a summarize of the conversion with:
%
%\begin{itemize}
%    \item Value to convert
%    \item Number of bits reserved to store the result
%    \item Decimal range available with that number of bits
%    \item Two's complement result of the conversion with that number of bits
%    \item Two's complement result of the conversion with 8-bytes (64 bits) memory storage 
%\end{itemize} 
%
%Notice that the result with the specified number of bits and the maximum allowed memory size (8-bytes) is the same 
%when no overflow has occurred. If an out-of-range number for the bits reserved is converted both results will differ. When this occurs, if the 
%conversion is performed from decimal to binary, the converter automatically shows the actual decimal value that is being stored due to the overflowing 
%value. The table \ref{tab:converterspecs} shows the specifications of the input arguments of the converter. 
%
%\texttt{call Twos\_complement\_converter( value, n\_bits )}
%
%\begin{table}[h]
%    \centering
%    \begin{tabular}{| c | c | c |}
%        
%        \hline
%        Name       & type                        & limits  \\ \hline
%        
%        \begin{tabular}{@{}c@{}} \texttt{value}  \\ (bin \rightarrow dec)  \end{tabular}   & \texttt{character}    &  \begin{tabular}{@{}c@{}} Up to 64 characters with 0's, 1's  \\ or blank spaces (treated as 0's) \end{tabular}   \\ \hline
%        
%        \begin{tabular}{@{}c@{}} \texttt{value}  \\ (dec \rightarrow bin)  \end{tabular}   & \texttt{integer}   &  $[-9 223 372 036 854 775 808, 9 223 372 036 854 775 807]$  \\ \hline
%        
%        \texttt{n\_bits}    & \texttt{integer}                     & $[1, 64]$                                                  \\ \hline
%        
%        \multicolumn{3}{l}{ \begin{tabular}{@{}c@{}} Notice that \texttt{n\_bits} does not need to coincide with the number of bits  \\  in the binary value to convert, it can be greater, equal or less. \end{tabular}    }   \\ \hline
%        
%    \end{tabular}                                                       
%    \caption{Input arguments of the \texttt{Twos\_complement\_converter} subroutine.}
%    \label{tab:converterspecs}
%\end{table}
%
%
%
%\newpage
%Examples of binary\rightarrow decimal use:
%
%\begin{verbatim} 
%call Twos_complement_converter( '0110100101101001', 30 )
%\end{verbatim} 
%
%\begin{verbatim} 
%--------------------------------------------------
%Conversion from two's complement binary to decimal
%                Value to convert =  0110100101101001
%                     Memory size =  30 bits
%Integer range = [          -536870912           536870911]
%     Two's complement conversion =                 26985
%Two's compl. 8-bytes conversion. =                 26985
%--------------------------------------------------
%\end{verbatim} 
%
%In this example the binary string is converted with enough memory size to hold the integer value, notice that 30 bits 
%is not a supported integer in most programming languages since this data type is always stored in 1, 2, 4 or 8 bytes of memory. 
%
%\begin{verbatim} 
%call Twos_complement_converter(                 &
%    '0000000101110011001010001100110000000000', &
%                                           32 )
%\end{verbatim} 
%
%\begin{verbatim} 
%--------------------------------------------------
%Conversion from two's complement binary to decimal
%Value to convert =  0000000101110011001010001100110000000000
%     Memory size =  32 bits
%Integer range = [         -2147483648          2147483647]
%     Two's complement conversion =            1932053504
%Two's compl. 8-bytes conversion. =            6227020800
%--------------------------------------------------
%\end{verbatim}
%
%In this example 4 bytes is not enough memory for this value, an overflow has occurred. The programmer should consider to use a 
%\texttt variable to hold it if this value is used in the program. 
%
%
%
%
%
%\newpage
%Examples of decimal\rightarrow binary use:
%
%\begin{verbatim} 
%call Twos_complement_converter( 735, 16 )
%\end{verbatim} 
%
%\begin{verbatim} 
%--------------------------------------------------
%Conversion from decimal to two's complement binary
%Value to convert =                   735
%Memory size =  16 bits
%Integer range = [              -32768               32767]
%Two's complement conversion =  0000001011011111
%Two's compl. 8-bytes conversion. =  0000000000000000000000
%                000000000000000000000000000000001011011111
%--------------------------------------------------
%\end{verbatim} 
%
%
%
%\begin{verbatim} 
%call Twos_complement_converter( 35000, 16 )
%\end{verbatim} 
%
%\begin{verbatim} 
%--------------------------------------------------
%Conversion from decimal to two's complement binary
%Value to convert =                 35000
%Memory size =  16 bits
%Integer range = [              -32768               32767]
%Two's complement conversion =  1000100010111000
%Two's compl. 8-bytes conversion. =  0000000000000000000000
%                000000000000000000000000001000100010111000
%--------------------------------------------------
%Overflow, not enough bits to store this number.
%The actual value stored in the computer is:
%--------------------------------------------------
%Conversion from two's complement binary to decimal
%                Value to convert =  1000100010111000
%                     Memory size =  16 bits
%Integer range = [              -32768               32767]
%     Two's complement conversion =                -30536
%Two's compl. 8-bytes conversion. =                 35000
%--------------------------------------------------
%\end{verbatim}
%
%In this last example notice how the number 35000 is an out-of-range value for a 2-bytes integer. 
%The converter warns that an overflow has occurred and the actual value stored in the computer is 
%$-30536$, which is the two's complement conversion of the 16 least significant bits involved in the 
%expected value. 






