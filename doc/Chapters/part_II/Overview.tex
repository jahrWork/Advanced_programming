%--------------------------------------------------------------------------------------------------------------------------------------
\chapter*{Operations with integers and reals} 

%\vspace{-1.3cm}
Being the base of any numerical calculation that a computer scientist wants to perform, 
both, integers and reals, must be a mastered subject.  
This part of the book covers the particularities of these two data types giving special
attention to the common errors that appear when programming maths in the computer. 

How the numbers are treated by computers is similar for most machines in the world. Although 
Fortran is used for the examples, any programming language will carry with the same issues 
in a similar way. Here you will find some useful notions to acquire a better understand of 
the behaviour of integers in a computer, the arithmetic behind different data types or the 
phenomena associated to the approximation of numbers by a finite precision. 

Make use during the reading of the chapter of the attached programs, there you will find the 
same examples explained here and the possibility to change and write your own programs 
using the existing codes. The following is the menu implemented in the program and also
serves as an overview of the topics covered in the following pages. 

\vspace{0.3cm}
 \renewcommand{\home}{./Fortran/sources/IEEE} 
  \listings{\home/integers_and_reals.f90}{select an option}{exception}{integers_and_reals.f90}

