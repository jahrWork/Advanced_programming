\chapter{Object Oriented Programming} 

\vspace{-1cm}
In the following sections we delve into the main principles of Object Oriented Programming through some examples: 
abstraction, encapsulation, inheritance and polymorphism. 


    \section{Encapsulation}
    
Let's continue with the example used for the overview of this chapter. 
Notice that the objects representing the droplets include everything they need (data and methods) in the same unit. 
Then, an object may share its state and interface so other objects can modify its state and use its methods 
or may not and keep them private, encapsulation lets the objects manage both. 
We could keep the state of each droplet private so it is not modified by other means than its own methods.  
If the interface is public then other objects can use the object but always by means of these methods provided.  
      
   
    \section{Abstraction}

As an extension of the encapsulation principle we can think of abstraction. 
Once the state and methods are encapsulated in the object, abstraction means that an object 
should hide any code and state that is not needed by the rest and 
should maintain public only a high-level interface to allow its use.
By exposing only the minimum necessary and hiding the implementation the code is less complex 
and increases reusability.


    

    \section{Inheritance}

Imagine that the simulation of evaporating droplets needs to include the combustion of spherical fuel droplets apart from the evaporation.
Fuel droplets present a similar state as the generic type: temperature, size, mass fraction of fuel at its surface, etc.
Also, they behave in a similar way: updating physical properties according to the temperature, decreasing size while it evaporates, etc. 
However, this new class of droplet includes a heat of combustion as state and has extra behaviour, burning and liberating energy. 

Inheritance allows the programmer to reuse all the logic behind the droplet class 
and create the new fuel droplet class with its unique particularities, hence, not blurring the 
definition of the droplet class.
A hierarchy has been created where a child class (fuel droplet) is derived from a parent class (droplet).




    \section{Polymorphism} 

Imagine now that the simulation requires non-spherical droplets, for example ellipsoid shaped. 
As we have seen, inheritance can be used to define this new class of droplets since they present a similar state and behaviour as spherical droplets.
However, the property ``size'' must take into account the three axes of an ellipsoid and not only the radius of a spherical shape. 
In addition, a method that decreases this size according to evaporation should also be consistent. 
Furthermore, we want the methods and properties to be used in the same way, whether the droplets in our simulation are spherical or ellipsoid shaped.

Polymorphism allows the use of the child classes in the same way as the parent classes with no need of mixing types. 
In our example, the size property or the updating size method have multiple shapes depending on the specific class where it is defined. 
The language decides at each case which implementation should be evaluated of the common method.
 
Mainly two types of polymorphism are usually distinguished: overloading and overriding. 
Overloading implies using methods with the same name in the same class, but 
having different number of parameters or, being the same number of parameters, having different types.
Overriding implies two or more classes with parent-child inheritance, using the same method, same number and types of parameters, 
but having the child class a specific implementation.
The method to use in a given situation for overloading is decided at compile-time,
for overriding is decided at runtime. 
Overloading is treated in depth in the chapter \ref{ch:overloading} for functions and operators. 


\newpage
In the following example the \texttt{polygon\_class} is created with 
the properties \texttt{color}, \texttt{N\_sides}, \texttt{length} and 
methods \texttt{perimeter()} and \texttt{polygon\_constructor()} as constructor. 
The classes \texttt{square\_class} and \texttt{circle\_class} are inherited from this one, 
the former modifies the definition of the constructor and 
the latter both the constructor and the perimeter method. 

Then, four objects are positioned in a variable called \texttt{Polygons}: a polygon, an square, a circle and the same polygon repeated. 
All of them are constructed using their respective classes and their perimeter obtained using the method \texttt{perimeter} indistinctly for each of them.


 
    \section*{Fortran} 
    
    Notice in the first place that the inherited classes \texttt{use} the module where the parent class is hosted. 
    Then, by using the \texttt{extends} keyword the child class can be defined as usual. 
    Also notice the use of \texttt{private} and \texttt{public} keywords to manage encapsulation and abstraction for these objects.  
    In this case, the \texttt{square\_class} redefines its constructor to build a four equal sides polygon.  
    \vspace{0.5cm}
    \renewcommand{\home}{./Fortran/sources/Advanced_programming/polymorphism} 
    \lstfor
    \listings{\home/polygon_class.f90}{module}{end type}{polygon_class.f90}
    
%    \listings{\home/square_class.f90}{module}{end module}{square_class.f90}

    \newpage
    \listings{\home/circle_class.f90}{module}{end module}{square_class.f90}
    
    
    
    \newpage
    In Fortran, the variable \texttt{polygons(:)} must be made with an array, which can only have elements of the same type.
    Since we need each element to be a different object, the variable is declared as an array of \texttt{pointer}, and 
    more specifically, an array of pointers with abstract class so their targets can be any type of object.
    When the objects \texttt{P}, \texttt{S} and \texttt{C} are constructed, 
    each of the four elements in the polymorphic array is pointed to each of them. 
    Finally, using the alias \texttt{Fi} for each object in the array, the perimeter is obtained and added to the total summation \texttt{L}.
    \vspace{0.3cm}
    \renewcommand{\home}{./Fortran/sources/Advanced_programming/polymorphism} 
    \lstfor
    \listings{\home/polymorphism.f90}{subroutine polymorphism_example}{end subroutine}{polymorphism.f90}
    
    
    
    
    
    
    
    
    \newpage
    \section*{Python}
    Python is much more straightforward for these purposes. 
    Notice that the keyword \texttt{self} is used substituting the instance of the class, 
    hence, all methods defined within the class can include references to the instance they are working with. 
    Among other purposes, this let a constructor method bind the arguments to the properties of the class. 
    
    The inheritance is achieved by passing the parent class as an argument of the child class. 
    Once done, the methods can be overrided like \texttt{parameter} in the example, which for both \texttt{square} and \texttt{circle} 
    needs to be defined with a new implementation. 
    While the \texttt{circle\_class} redefines the constructor from the beginning, notice that \texttt{square\_class} uses the inherited constructor 
    and imposes the need of having four equal length sides. To do that, the keyword \texttt{super} is used so a the methods of the parent class are accessible.     
    
    The function \texttt{polymorphism\_example} creates each object using their respective constructors and gathers the four instances 
    \texttt{P}, \texttt{S}, \texttt{C} and \texttt{P} again in a list. Since Python allows structures collecting different data types, 
    there is no need of using pointers. Finally, the perimeter of each object is computed and summed in \texttt{L}.
    
    \newpage
    \renewcommand{\home}{./Python/sources/Advanced_programming/polymorphism} 
    \lstpython
    \listingsp{\home/polymorphism.py}{class}{Total perimeter}{polymorphism.py}











%\newpage 
%\section{ODEs integration} 
% 
%\subsection*{Fortran}
%\renewcommand{\home}{./Fortran/sources/Advanced_programming/polymorphism} 
%\lstfor
%\listings{\home/polymorphic_ODES.f90}{def Euler}
%{polymorphic}{polymorphic_ODES.f90} 
% 
% 
%\subsection*{Python}
%\renewcommand{\home}{./Python/sources/Advanced_programming/polymorphism} 
%\lstpython
%\listingsp{\home/polymorphic_ODES.py}{def Euler}
%{polymorphic}{polymorphic_ODES.py}


