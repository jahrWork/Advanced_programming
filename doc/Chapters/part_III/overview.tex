
 
 
 
  
\chapter{Overview} 
One of the main characteristics to reuse code is generic programming.  
Generic programming is based on abstract variable types that are then instantiated when they are used for specific variable type.

Since Python is non typed language, 
generic programming in this language is straightforward. 
However, in Fortran the use of abstract  \lstinline{class(*)} 
allows to use different data types at run time. 
  
    
  \vspace{0.5cm}
   \renewcommand{\home}{./Fortran/sources/Advanced_programming} 
   
    \listings{\home/advanced_programming_techniques.f90}{while}{Wrapper}
             {advanced_programming_techniques.f90}
  
  \newpage 
  
   \vspace{0.5cm}
     \renewcommand{\home}{./Python/sources/Advanced_programming} 
     
      \listings{\home/advanced_programming_techniques.py}{while}{Wrappers}
               {advanced_programming_techniques.py}
               

            
\section{Scope} 

One of the most important matters that we need to understand when we begin to write our own codes in any programming language is the scope. 
The scope of variables, named constants, objects, functions or procedures is the area of the program where they can be used or modified, that is, the area where each object is visible.

The example that best illustrates this concept is a variable \texttt{c} declared inside a function \texttt{f( x )}. 
This variable named \texttt{c} can be initialized, used and modified inside the function. 
However, outside this area, \texttt{c} is not seen or the name \texttt{c} could refer to a totally different entity. 
In this case \texttt{c} is said to be a \textit{local variable}.
When a variable is declared at the beginning of a program, outside the scope of any function, it is said to be a \textit{global variable} and it can be used everywhere in your program unless you intentionally limit its scope. 

Notice that \textit{global} and \textit{local} sometimes refer to those entities that are seen by the whole program or just some parts of the code, respectively. 
On other occasions, \textit{global} and \textit{local} entities refer to one specific unit of program/subprogram, for instance, a variable declared at the beginning of a module is a \textit{global} variable for that module, so any subroutine or function defined inside will see it. However, this does not necessarily mean that the variable is seen by the main unit of program that uses this module. 

%Another example is the variable declared at the beginning of a subroutine that has a function declared inside, it is considered a global variable for the subroutine and functions nested but it is considered local according to the main program. 

Each programming language has its own set of rules to consider entities as global or local, however, sometimes the programmer can limit or enlarge the scope.
The statements \texttt{public} and \texttt{private} change the scope of entities in Fortran so they can be accessed or not outside a module.
Hence, private variables or procedures are specified explicitly. Otherwise, global module variables are visible outside by default.   

Scope is an essential concept when talking about modular programming and encapsulation. Some constants may be placed at the beginning of the code as global entities so every part of the code can see the same value (e.g. \texttt{pi = 4 * atan(1)}), in this case it is strongly recommend to use the \texttt{parameter} attribute so its value can not change during the execution. However, we want each subprogram unit to use their own set of variables and constants without interfering with the rest of parts of the program. Furthermore, we also want every subprogram to use only local variables so everything needed is either declared inside or passed via parameter. 
As a general rule, the use of global variables should be reduced as possible, these entities can be easily modified unintentionally leading to hard to fix bugs.  



\section{Functional Programming}

to mimic the mathematical functions 

\section{Overloading} 
\section{Object Oriented Programming}
\section{Pointers} 
\section{Wrappers} 
\section{Mixing Python and Fortran}
\section{Parallel programming} 


               