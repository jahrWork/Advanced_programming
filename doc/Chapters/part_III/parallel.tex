
 


\chapter{Parallel programming} 


\vspace{-1cm} 
\section{Introduction}
\vspace{-0.2cm} 
Computational fluid dynamics (CFD) 
uses numerical methods to simulate problems that involve fluid flows. 
Computers are used to perform the calculations required to simulate 
fluid motion subjected to boundary conditions associated to surfaces.  
High-speed supercomputers are  required to solve complex problems
and to achieve better solutions.
The idea of parallel programming is based on a bunch of CPU processors in a specific architecture 
to divide the required calculations into different tasks that could be performed in parallel.
Usually, space dimension divide the space domain in different regions, blocks or elements.
Time domain is usually difficult to divide because it constitutes the evolution line 
or the iterative counter. 
However, every time step, the  state vector $ U(t) $ is updated 
by vector operations. These operations  can be parallelized by splitting the the state 
vector in two or more parts. A profound knowledge of the numerical method is required to 
design a strategy to parallelize the programming code.
  
When talking about real parallel programs, two classic libraries are involved: MPI (Message Passing Interface)
and OpenMP (Open Multi-Processing).
MPI is a standardized and portable message-passing standard designed to function onparallel computing architectures
and OpenMP is an application programming interface (API) 
that supports multi-platform shared-memory multiprocessing programming. 
Both libraries are supported  in C, C++, and Fortran.
Besides, Fortran supports parallel programming natively without using any external library. 
Fortran defines a new type called Coarray Fortran (CAF) to implement parallel programming with a minimum effort. 
A CAF program is replicated a number of times and all copies were executed asynchronously. 
Each copy has its own set of data objects and is termed an image. 
Since the inclusion of coarrays in the Fortran 2008 standard, the number of implementations is growing.


\newpage
\section{Sum of series}
The idea of this example is to show how independent operations can be performed in parallel by using 
Coarrays in Fortran. 
The following program sums the first $ N $ terms of the series: 
\begin{equation}
S_N = \sum _{i=1} ^N \ \frac{1}{i^2}
\end{equation}
If $ Ni$  processors are available to perform calculations at the same time, 
the total sum can be divided in $Ni$  parts  with the following expression: 
\begin{equation}
S_N = S^1 + S^2 +  \ldots + S^{Ni}.
\end{equation}
Each partial sum $ S^j $  will add $ M = N/Ni $ terms and 
its  expression performed by each processor or image $ j $ is: 
\begin{equation}
S^j = \sum _{i=(j-1)M} ^{j M}  \ \frac{1}{i^2}.
\end{equation}
In the following example, a parallel Fortran program is written to 
perform this calculations. 
First,  the intrinsic function \verb|num_images()| determines the number of available processors. 
A real variable is specified to store the partial sums of each processor by the sentence 
\begin{verbatim}
        real (kind=8) :: S[*] 
\end{verbatim} 
Note that \verb|S| is a coarray and its dimension is determined at run-time.  
Once any \verb|image| is finished, the real variable \verb|SN| sums every contribution of different 
images \verb|j| with the following sentence: 
\begin{verbatim}
        SN = SN + S[j] 
\end{verbatim} 
It is important to note that this program is replicated exactly in each image. Hence, functions involved in 
the parallel program should be pure to avoid side effects. This is  one of the most important 
reasons why functional programming with pure functions is growing in popularity. 

\newpage 
\renewcommand{\home}{./Parallel_programming/sources} 
\listings{\home/Series_with_coarrays.f90}{program}
{CPU time}{Series_with_coarrays.f90}



%\section{Python} 








 