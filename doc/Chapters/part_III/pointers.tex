
 
\chapter{Pointers} 

\vspace{-1cm}
\section{Introduction}
\vspace{-0.2cm}
A pointer is a variable that stores a memory address of some other variable or function. 
For example, the sentence \verb|a = 10.4| creates the internal representation of the float 
 number \verb|10.4| and stores it in some part of the computer memory. After this number is stored, 
the label \verb|a| holds the memory direction where \verb|10.4| is stored. 
It is said that the variable \verb|a| points to the number \verb|10.4|.

Pointers are a very powerful concept to optimize memory, to write compact codes and to pass
array arguments to different functions. However, they are difficult to understand 
and very dangerous at the beginning and their use should be restricted to advanced users. 
Following this idea, Python language do not have pointer variables.
However, in Python everything is a pointer internally and 
it is necessary to understand how it works to avoid  misunderstandings and 
strange behaviors or errors.
Fortran language has pointers but they  have been restricted to  avoid
classical problems and to make them safer.
Pointers are also used when calling a function. Different inputs arguments are passed by reference or by value. 
If they are passed by reference, it means that only their address or pointer is passed. 
If they are passed by value, it means that a copy of the original argument is made
and a new local variable for this function is created. 
Once the function is abandoned, the variable is destroyed.  
Pointers can  also store the addresses of functions 
and can be used to compose or construct new functions. 

In the following sections, 
different examples will be shown from scalar variables to array or vector variables 
explaining the pointer concept and its relevance in numerical problems.
Some examples are chosen to lighten obscure hidden programming errors  even 
in languages like Python in which no explicit pointers are present. 



\newpage 
\subsection*{Fortran}
In the following example, scalar real variables (\verb|a|, \verb|b|) 
and real vector  variables are created (\verb|v|, \verb|U|) with the \verb|target| attribute. 
It means that these variables can be pointed by pointers.  
Additionally, pointers to scalar  variables (\verb|pa|) and 
pointers to real vector  variables are created (\verb|pv|, \verb|pU|) 
with the \verb|pointer| attribute.
This attribute allows to point these pointer variables to target variables. 
In Fortran, to point a pointer variable \verb|pa| to a target variable \verb|a|, 
the following sentence is used: \verb|pa =>a|.
Once \verb|pa| is pointed, \verb|pa| works as any other classical variables 
and any modification in \verb|a| alters the content of \verb|pa| and vice-versa because they both point
to the same address of memory. 

The same concepts are applied to vector variables with same or different rank. 
The two dimensional pointer \verb|pU| is pointed to
vector$ \verb|U| $ with $4$  components. The sentence \verb|pU(0:1,0:1) => U| points a matrix pointer 
of size $ 2 \times 2 $ to the vector \verb|U|. It is important to note, 
that in Fortran that \verb|U| holds the matrix \verb|pU|  by columns. 
\vspace{0.3cm}  
\renewcommand{\home}{./Fortran/sources/Advanced_programming/pointers} 
\lstfor
\listings{\home/pointers.f90}{pointer_examples}{end subroutine}{pointers.f90}


\newpage
\subsection*{Python}
As it was mentioned, Python has no pointer variables but all variables are treated like pointers
internally. The sentence \verb|a =3| transforms the decimal number \verb|3| to its binary representation
in some part of the memory. Later, label \verb|a| is pointed to the address where this integer is stored. 
The next sentence \verb|pa =a| copies the same address of variable \verb|a| for the new label
\verb|pa|. To check this fact, the intrinsic Python function \verb|id(a)| gives the identifier where 
\verb|a| is stored. The next sentence \verb|a=4| creates again a new integer number an stores 
it in some part of the computer memory. At this time, \verb|pa| is pointing to number \verb|3| and 
\verb|a| is pointing to number  \verb|4|.
If \verb|a| is modified, \verb|pa| is not modified because they are pointing 
to two different addresses of memory. 


The same concepts are applied to numpy arrays of same or different ranks.
The sentence \verb|a=array([1,2,3,4])| creates a one dimensional numpy array.
The sentence \verb|b=a| points varriable \verb|b| to the memory address of \verb|a|. 
Hence, by modifying \verb|a|, the variable \verb|b| is modified consistently.
However, the sentence \verb|a=a+1| does not work as expected. The reason 
is that this sentence creates another instance for \verb|a| which means
that \verb|pa| and  \verb|a| are stored in different positions of memory. Hence, 
the sentence \verb|a=a+1| does not modify the content of \verb|pa|. 
To fix this issue, the  sentence \verb|a[:]=a[:]+1| should be used 
to modify \verb|pa| at the same time.  
To point a two dimensional array to a one dimensional array, 
the sentence \verb|pa=reshape(a,(2,2))| assign a matrix pointer 
of size $ 2 \times 2 $ to the vector \verb|a|. It is important to note, 
that in Python that \verb|a| holds the matrix \verb|pa|  by rows. 
\vspace{0.4cm}  
\renewcommand{\home}{./Python/sources/Advanced_programming/pointers} 
\lstpython
\listingsp{\home/pointers.py}{def pointer_examples}{pa points to a}{pointers.py}







\section{Aliasing and cloning} 
Depending on the program needs, variables could share the same memory addresses than others 
to implement a clear and concise code or to optimize memory. 
Hence, it is important to know the mechanisms to achieve te desired result.
The process of pointing different labels to the same memory content is called aliasing.
If aliasing is considered, changes in different labels or variables are reflected in 
different variables. If this behavior is not desired, cloning creates
a duplicate independent object.
These concepts only apply to Python because in Fortran target and pointer variables are specified explicitly. 
In the following example, mechanisms for aliasing or  cloning  sets, tuples,lists or dictionaries
are shown. Usually, the sentence \verb|pa=a| is used to alias \verb|pa| to \verb|a|. 
If duplicate or copies are required, \verb|copy| function is used for sets, \verb|dict| function is used
for dictionaries and slicing  \verb|pa=a[:]| is used for lists. 
  

\subsection*{Python}

\vspace{0.4cm}  
\renewcommand{\home}{./Python/sources/Advanced_programming/pointers} 
\lstpython
\listingsp{\home/pointers.py}{def aliasing_and_cloning_examples}{Dictionaries, cloning}{pointers.py}



\newpage
\section{N-body problem}  
This example is used to show how pointers allow to write simulation codes 
where the same system variables are seen with different shapes. 
 
The N-body problem is the problem of predicting the individual motions 
of a group of mass objects under the influence of their gravitational field.
If  $ \bold{r}_i \in \mathbb{R}^3$ and $ \bold{v}_i \in \mathbb{R}^3$ are 
the position and the velocity vectors 
of each mass object $  i $, the equations of motion for each mass object writes:  
\begin{equation} 
\frac{d \bold{r}_i}{dt} = \bold{v}_i, \qquad i=1, \ldots,N  
\label{velocity} 
\end{equation}

\begin{equation} 
m_i \frac{d \bold{v}_i}{dt} =  \sum_{j=1, j \ne i} ^N G \ m_i \ m_j \ \frac{\bold{r}_j -\bold{r}_i}
                              {\|\bold{r}_j -\bold{r}_i \|^3}
                              \qquad i=1, \ldots,N 
\label{momentum}                              
\end{equation}
These ordinary differential equations together with initial conditions for the position 
and velocity of each mass point  constitute a Cauchy problem ($dU/dt = F(U,t)$) 
that can be integrated with any temporal scheme. 
To prepare the system to use any library for the numerical integration, a vector state comprising all 
degrees of freedom mus be defined. 
The state vector $ U(t) $ and its derivative $ F(U,t) $ can be  defined as: 
\begin{equation} 
U(t) =
\begin{pmatrix}
\begin{pmatrix}
  x_1 \\ y_1 \\ z_1 \\ \dot x_1 \\ \dot y_1 \\ \dot z_1 
\end{pmatrix}
\\
\begin{pmatrix}
  \vdots \\
\end{pmatrix}
\\
\begin{pmatrix}
  x_N \\ y_N \\ z_N \\ \dot x_N \\ \dot y_N \\ \dot z_N 
\end{pmatrix}
\end{pmatrix}, \qquad
F(U,t) =
\begin{pmatrix}
\begin{pmatrix}
  \dot x_1 \\ \dot y_1 \\ \dot z_1 \\ \ddot x_1 \\ \ddot y_1 \\ \ddot z_1 
\end{pmatrix}
\\
\begin{pmatrix}
  \vdots \\
\end{pmatrix}
\\
\begin{pmatrix}
  \dot x_N \\ \dot y_N \\ \dot z_N \\ \ddot x_N \\ \ddot y_N \\ \ddot z_N 
\end{pmatrix}
\end{pmatrix}.
\end{equation} 
where $ (x_i, y_i, z_i )$ and $ (\dot x_i, \dot y_i, \dot z_i )$ are the position and velocity 
of the  mass point $ i $. Note that the values $ (\ddot x_i, \ddot y_i, \ddot z_i )$
are obtained from equation (\ref{momentum}). The positions and the velocities of all points 
an be stored in two matrices ${r}_{ik} $ and  ${v}_{ik} $.
The first index represents the object $i$ and the second index represents the coordinate. 
%To save the two different arrays could be considered 
%to Since $ U(t) $ contains all degrees 
%of freedom of the problem, a

   

\newpage
\subsection*{Fortran}
In the following example, the function \verb|F_NBody| uses 
the state vector \verb|U| as an input and returns its derivative \verb|F|. Both arguments   
are considered  target variables. Pointer variables \verb|Us| and \verb|dUs| are created 
with rank three to be pointed to \verb|U| and \verb|F| respectively. The first index represents
the body, the second index the coordinate and the third index their position or velocity. 
The sentence \verb|Us(1:Nb,1:Nc,1:2)=>U| points \verb|Us| to \verb|U|. Now the components 
of \verb|Us| and \verb|U| are the same but with different shape. Another two pointer 
variables are specified \verb|r| and \verb|v| with rank two to be pointed to different parts of 
\verb|Us|. The first index of \verb|r| represents the body and the second index the coordinate 
position. The pointers \verb|r| and \verb|v| are pointed to different parts of \verb|Us|. The same 
idea is done for the derivative pointers \verb|drdt| and \verb|dvdt|. 
Once all pointers are assigned, equations (\ref{velocity}) and (\ref{momentum}) are written 
with no effort and the derivative of the vector state is modified automatically. 
This powerful technique based on pointers allows to integrate the N-body Cauchy problem 
by using standard libraries. 
\vspace{0.3cm}
\renewcommand{\home}{./Fortran/sources/Advanced_programming/odes} 
\lstfor
\listings{\home/n_body_problem.f90}{function F_NBody}{end function}{n_body_problem.f90}

\listings{\home/n_body_problem.f90}{Cauchy_ProblemS}{Cauchy_ProblemS}{n_body_problem.f90}


\newpage
\subsection*{Python}
In the following example, the function \verb|F_NBody| uses 
the state vector \verb|U| as an input and returns its derivative \verb|F|. 
The \verb|numpy| function \verb|reshape| allows to create pointers \verb|Us| and \verb|dUs|
with rank three  pointed to \verb|U|. The first index of \verb|Us| represents
the body, the second index the coordinate and the third index their position or velocity. 
Now the components 
of \verb|Us| and \verb|U| are the same but with different shape. 
Another two pointer \verb|r| and \verb|v| with rank two are created reshaping \verb|Us|. 
The first index of \verb|r| represents the body and the second index the coordinate 
position. The pointers \verb|r| and \verb|v| are pointed to different parts of \verb|Us|. 
The same idea is done for the derivative pointers \verb|drdt| and \verb|dvdt|. 
Once all pointers are assigned, equations (\ref{velocity}) and (\ref{momentum}) are written 
with no effort and the derivative of the vector state is modified automatically. 
This powerful technique based on pointers allows to integrate the N-body Cauchy problem 
by using standard libraries. In this example, the N-body problem is integrated 
the standard \verb|odeint| function from \verb|scipy| and with our own function 
\verb|Caucy_problem|.
\vspace{0.5cm}
\renewcommand{\home}{./Python/sources/Advanced_programming/odes} 
\lstpython
\listingsp{\home/n_body_problem.py}{def F_NBody}{return F}{n_body_problem.py}



%\listingsp{\home/n_body_problem.py}{def Initial_positions_and_velocities}{-0.4}{n_body_problem.py}
\listingsp{\home/n_body_problem.py}{F_NBody, U0}{F_NBody, Time}{n_body_problem.py}




