
% ________________________CODE LENGUAGE________________________________________
\usepackage{listings}

\lstdefinelanguage{JavaScript}{
	keywords={typeof, new, true, false, catch, function, return, null, catch, switch, var, if, in, while, do, else, case, break, undefined},
	keywordstyle=\color{blue}\bfseries,
	ndkeywords={class, export, boolean, throw, implements, import, this},
	ndkeywordstyle=\color{darkgray}\bfseries,
	identifierstyle=\color{black},
	sensitive=false,
	comment=[l]{//},
	morecomment=[s]{/*}{*/},
	commentstyle=\color{red}\ttfamily,
	stringstyle=\color{orange}\ttfamily,
	morestring=[b]',
	morestring=[b]"
}



%\lstset{literate=%
%	{0}{{{\color{Green}0}}}1
%	{1}{{{\color{Green}1}}}1
%	{2}{{{\color{Green}2}}}1
%	{3}{{{\color{Green}3}}}1
%	{4}{{{\color{Green}4}}}1
%	{5}{{{\color{Green}5}}}1
%	{6}{{{\color{Green}6}}}1
%	{7}{{{\color{Green}7}}}1
%	{8}{{{\color{Green}8}}}1
%	{9}{{{\color{Green}9}}}1
%}


\newcommand{\lsthtml}{
	\lstset{
		language=html,
		basicstyle=\ttfamily\small,
		frame=single,
		showspaces=false,
		showstringspaces=false,
		backgroundcolor=\color{white},
		keywordstyle=\color{blue},
		deletekeywords={type},
		commentstyle=\color{mygreen},
		stringstyle=\color{violet},
		breaklines=true,
		alsoletter={.},morekeywords={nav,/nav}
	}
}
\newcommand{\lstbash}{
	\lstset{
		language=bash,
		basicstyle=\ttfamily\small,
		frame=single,
		showspaces=false,
		showstringspaces=false,
		backgroundcolor=\color{white},
		keywordstyle=\color{blue},
		deletekeywords={type},
		commentstyle=\color{mygreen},
		stringstyle=\color{violet},
		breaklines=true,
		alsoletter={.},
		morekeywords={ssh,scp,server.py,runserver.sh,killserver.sh,killemail.sh,runemail.sh}
	}
}


%\newcommand{\lstpython}{
%	\lstset{
%		language=python,
%		basicstyle=\footnotesize\ttfamily,
%		frame=single,
%		showspaces=false,
%		showstringspaces=false,
%		backgroundcolor=\color{gray!2},
%		keywordstyle=\normalsize\bfseries,
%		deletekeywords={type},
%        commentstyle=\color{gray!50},
%        captionpos=b, % sets the caption-position to bottom
%		stringstyle=\color{gray},     % string literal style
%		breaklines=true,
%        breakatwhitespace=false, 
%		alsoletter={.,...},
%		morekeywords={},
%        tabsize=2, % sets default tabsize to 2 spaces
%        numberstyle=\tiny, % the size of the fonts that are used for the line-numbers     
%        numberstyle = \color{black},
%        rulesepcolor=\color{gray},
%        frame=trBL,
%        showtabs=false % show tabs within strings adding particular underscores
%	}
%}





\newcommand{\lstjs}{
	\lstset{
		language=JavaScript,
		extendedchars=true,
		basicstyle=\ttfamily\footnotesize,
		frame=single,
		showstringspaces=false,
		showspaces=false,
		numberstyle=\footnotesize,
		numbersep=9pt,
		tabsize=4,
		breaklines=true,
		showtabs=false,
		captionpos=b
	}
}




%
%
%\lstset{ %
%	backgroundcolor=\color{white},   % choose the background color; you must add \usepackage{color} or \usepackage{xcolor}
%	basicstyle=\footnotesize,%%        % the size of the fonts that are used for the code
%	breakatwhitespace=false,         % sets if automatic breaks should only happen at whitespace
%	breaklines=true,                 % sets automatic line breaking
%	captionpos=b,                    % sets the caption-position to bottom
%	commentstyle=\color{mygreen},    % comment style
%	deletekeywords={...},            % if you want to delete keywords from the given language
%	escapeinside={\%*}{*)},          % if you want to add LaTeX within your code
%	extendedchars=true,              % lets you use non-ASCII characters; for 8-bits encodings only, does not work with UTF-8
%	frame=single,                    % adds a frame around the code
%	keepspaces=true,                 % keeps spaces in text, useful for keeping indentation of code (possibly needs columns=flexible)
%	keywordstyle=\color{blue}\textbf,       % keyword style
%	language=Fortran,                 % the language of the code
%	otherkeywords={*,source},            % if you want to add more keywords to the
%	% set
%	numbers=none,                    % where to put the line-numbers; possible
%	% values are (none, left, right)
%	numbersep=5pt,                   % how far the line-numbers are from the code
%	numberstyle=\tiny\color{mygray}, % the style that is used for the line-numbers
%	rulecolor=\color{black},         % if not set, the frame-color may be changed on line-breaks within not-black text (e.g. comments (green here))
%	showspaces=false,                % show spaces everywhere adding particular underscores; it overrides 'showstringspaces'
%	showstringspaces=false,          % underline spaces within strings only
%	showtabs=false,                  % show tabs within strings adding particular underscores
%	stepnumber=2,                    % the step between two line-numbers. If it's 1, each line will be numbered
%	stringstyle=\color{mymauve},     % string literal style
%	tabsize=2,                       % sets default tabsize to 2 spaces
%	title=\lstname                   % show the filename of files included with \lstinputlisting; also try caption instead of title
%}
%
%
%


%\lstset{ %
%	backgroundcolor=\color{Goldenrod!10},   % choose the background color; you must add \usepackage{color} or \usepackage{xcolor}
%	basicstyle=\footnotesize\ttfamily,%%\small,               % the size of the fonts that are used for the code
%	breakatwhitespace=false,                % sets if automatic breaks should only happen at whitespace
%	breaklines=true,                 % sets automatic line breaking
%	captionpos=b,                    % sets the caption-position to bottom
%	commentstyle=\color{mygreen},    % comment style
%	deletekeywords={...},            % if you want to delete keywords from the given language
%	escapeinside={\%*}{*)},          % if you want to add LaTeX within your code
%	extendedchars=true,              % lets you use non-ASCII characters; for 8-bits encodings only, does not work with UTF-8
%	frame=single,                    % adds a frame around the code
%	keepspaces=true,                 % keeps spaces in text, useful for keeping indentation of code (possibly needs columns=flexible)
%	keywordstyle=\color{blue}\textbf,       % keyword style
%	language=Fortran,                 % the language of the code
%	otherkeywords={*,source},            % if you want to add more keywords to the
%	% set
%	numbers=none,                    % where to put the line-numbers; possible
%	% values are (none, left, right)
%	numbersep=5pt,                   % how far the line-numbers are from the code
%	numberstyle=\tiny\color{mygray}, % the style that is used for the line-numbers
%	rulecolor=\color{black},         % if not set, the frame-color may be changed on line-breaks within not-black text (e.g. comments (green here))
%	showspaces=false,                % show spaces everywhere adding particular underscores; it overrides 'showstringspaces'
%	showstringspaces=false,          % underline spaces within strings only
%	showtabs=false,                  % show tabs within strings adding particular underscores
%	stepnumber=2,                    % the step between two line-numbers. If it's 1, each line will be numbered
%	stringstyle=\color{mymauve},     % string literal style
%	tabsize=2,                       % sets default tabsize to 2 spaces
%%	title=\texttt{\detokenize{\lstname}}     % show the filename of files included with \lstinputlisting; also try caption instead of title
%%	title= \texttt{\lstname}
%%    caption =\texttt{\detokenize{\lstname}}
%%    caption =\detokenize{\lstname}
%}




%\newcommand{\lstfor}{
% \lstset{ 
%    language=Fortran, % choose the language of the code
%    % basicstyle=\footnotesize\ttfamily,
%    % keywordstyle=\color{black}\textit, % style for keywords
%    basicstyle=\footnotesize\ttfamily,
%    keywordstyle=\normalsize\bfseries,%\textit, %%
%    numbers=none, % where to put the line-numbers
%    numberstyle=\tiny, % the size of the fonts that are used for the line-numbers     
%    backgroundcolor=\color{gray!2},
%    showspaces=false, % show spaces adding particular underscores
%    showstringspaces=false, % underline spaces within strings
%    showtabs=false, % show tabs within strings adding particular underscores
%    frame=single, % adds a frame around the code
%    tabsize=2, % sets default tabsize to 2 spaces
%    rulesepcolor=\color{gray},
%    stringstyle=\color{gray},     % string literal style
%    captionpos=b, % sets the caption-position to bottom
%    breaklines=true, % sets automatic line breaking
%    breakatwhitespace=false, 
%    commentstyle=\color{gray!50},
%    numberstyle = \color{black},
%    frame=trBL,
%    morekeywords={procedure, findloc}
%}
%}


\newcommand\mycap[1]{ \texttt{ \detokenize{#1} }  }


\definecolor{mymauve}{rgb}{0.58,0,0.82}


\definecolor{codegreen}{rgb}{0,0.6,0}
\definecolor{codegray}{rgb}{0.5,0.5,0.5}
\definecolor{codepurple}{rgb}{0.58,0,0.82}
\definecolor{backcolour}{rgb}{0.95,0.95,0.92}

\newcommand{\lstfor}{
 \lstset{ 
     language=Fortran, % choose the language of the code
      morekeywords=[1]{,imag,procedure,abstract,findloc},
    % basicstyle=\footnotesize\ttfamily,
   %  keywordstyle=\color{black}\textit, % style for keywords
     basicstyle=\footnotesize\ttfamily,
     keywordstyle=\color{blue}\normalsize\bfseries,%\textit, %%
     numbers=none, % where to put the line-numbers
     numberstyle=\tiny, % the size of the fonts that are used for the line-numbers     
     backgroundcolor=\color{blue!1},
  %   backgroundcolor=\color{blue!5!white},
     showspaces=false, % show spaces adding particular underscores
     showstringspaces=false, % underline spaces within strings
     showtabs=false, % show tabs within strings adding particular underscores
     frame=single, % adds a frame around the code
  %   rulecolor=\color{black},
     rulecolor=\color{Blue!75!black},
     tabsize=2, % sets default tabsize to 2 spaces
  %   rulesepcolor=\color{blue},
     rulesepcolor=\color{Blue!75!black},
     stringstyle=\color{orange},     % string literal style
     captionpos=b, % sets the caption-position to bottom
     breaklines=true, % sets automatic line breaking
     breakatwhitespace=false, 
     commentstyle=\color{codegreen},
     numberstyle = \color{black},
     frame=trBL,
   %  morekeywords={imag,procedure, abstract, findloc}
  %    morekeywords=[1]{,imag,procedure,abstract,findloc}
 }
}



\newcommand{\lstpython}{
 \lstset{ 
     language=Python, % choose the language of the code
    % basicstyle=\footnotesize\ttfamily,
   %  keywordstyle=\color{black}\textit, % style for keywords
     basicstyle=\footnotesize\ttfamily,
     keywordstyle=\color{blue}\normalsize\bfseries,%\textit, %%
     numbers=none, % where to put the line-numbers
     numberstyle=\tiny, % the size of the fonts that are used for the line-numbers     
     backgroundcolor=\color{blue!1},
     showspaces=false, % show spaces adding particular underscores
     showstringspaces=false, % underline spaces within strings
     showtabs=false, % show tabs within strings adding particular underscores
     frame=single, % adds a frame around the code
   %  rulecolor=\color{black},
     rulecolor=\color{Blue!75!black},
     tabsize=2, % sets default tabsize to 2 spaces
   %  rulesepcolor=\color{blue},
     rulesepcolor=\color{Blue!75!black},
     stringstyle=\color{orange},     % string literal style
     captionpos=b, % sets the caption-position to bottom
     breaklines=true, % sets automatic line breaking
     breakatwhitespace=false, 
     commentstyle=\color{codegreen},
     numberstyle = \color{black},
     frame=trBL,
     morekeywords={}
 }
}
